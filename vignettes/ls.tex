\documentclass{article}

\title{
  ls: Least Squares Regression for Continuous
  Dependent Variables
}
\author{Matt Owen, Olivia Lau, Kosuke Imai, and Gary King}




\usepackage{bibentry}
\usepackage{graphicx}
\usepackage{natbib}
\usepackage{amsmath}
\usepackage{url}
\usepackage{Zelig}
\usepackage{Sweave}

%\VignetteIndexEntry{Least Squares Regression for Continuous Dependent Variables}
%\VignetteDepends{Zelig, stats}
%\VignetteKeyWords{model,least squares,continuous, regression}
%\VignettePackage{Zelig}

\begin{document}

\nobibliography*



\section{{\tt ls}: Least Squares Regression for Continuous
Dependent Variables}
\label{ls}

Use least squares regression analysis to estimate the best linear
predictor for the specified dependent variables.

\subsubsection{Syntax}

\begin{verbatim}
> z.out <- zelig(Y ~ X1 + X2, model = "ls", data = mydata)
> x.out <- setx(z.out)
> s.out <- sim(z.out, x = x.out)
\end{verbatim}

\subsubsection{Additional Inputs}  

In addition to the standard inputs, {\tt zelig()} takes the following
additional options for least squares regression:  
\begin{itemize}
\item {\tt robust}: defaults to {\tt FALSE}.  If {\tt TRUE} is
selected, {\tt zelig()} computes robust standard errors based on
sandwich estimators (see \cite{Zeileis04}, \cite{Huber81}, and
\cite{White80}).  The default type of robust standard error is
heteroskedastic consistent (HC), \emph{not} heteroskedastic and
autocorrelation consistent (HAC).  

In addition, {\tt robust} may be a list with the following options:  
\begin{itemize}
\item {\tt method}:  choose from 
\begin{itemize}
\item {\tt "vcovHC"}: (the default if {\tt robust = TRUE}), HC standard errors.
\item {\tt "vcovHAC"}: HAC standard errors without weights.  
\item {\tt "kernHAC"}: HAC standard errors using the weights given in
\cite{Andrews91}.   
\item {\tt "weave"}: HAC standard errors using the weights given in
\cite{LumHea99}.
\end{itemize} 
\item {\tt order.by}: only applies to the HAC methods above.  Defaults to
{\tt NULL} (the observations are chronologically ordered as in the
original data).  Optionally, you may specify a time index (either as
{\tt order.by = z}, where {\tt z} exists outside the data frame; or
as {\tt order.by = \~{}z}, where {\tt z} is a variable in the data
frame).  The observations are chronologically ordered by the size of
{\tt z}.
\item {\tt \dots}:  additional options passed to the functions
specified in {\tt method}.  See the {\tt sandwich} library and
\cite{Zeileis04} for more options.   
\end{itemize}
\end{itemize}

\subsubsection{Examples}\begin{enumerate}
\item Basic Example with First Differences

Attach sample data:
\begin{Schunk}
\begin{Sinput}
>  data(macro)
\end{Sinput}
\end{Schunk}
Estimate model:
\begin{Schunk}
\begin{Sinput}
>  z.out1 <- zelig(unem ~ gdp + capmob + trade, model = "ls", data = macro)
\end{Sinput}
\end{Schunk}
Summarize regression coefficients:
\begin{Schunk}
\begin{Sinput}
>  summary(z.out1)
\end{Sinput}
\end{Schunk}
Set explanatory variables to their default (mean/mode) values, with
high (80th percentile) and low (20th percentile) values for the trade variable:
\begin{Schunk}
\begin{Sinput}
>  x.high <- setx(z.out1, trade = quantile(macro$trade, 0.8))