\section{{\tt names}: Identifying the Structure of Zelig Objects}

\subsubsection{Description}
  The function \texttt{names()} returns the named elements within an
  object.  For objects created using \texttt{zelig()} and
  \texttt{sim()}, \texttt{names()} returns nested lists as
  object-level arguments.  Use the \texttt{\$} operator to extract 
elements
  from objects.  

\subsubsection{Syntax}
\begin{verbatim}
> names(x)                                            # Lists elements. 
> names(x) <- "value"                                 # Assigns names to x.  
\end{verbatim}

\subsubsection{Arguments}
\begin{itemize}
\item {\tt x}: any R object. 
\item {\tt "value"}: a character vector of up to the same length as
  \texttt{x} (or \texttt{NULL} to clear the names from the object).
\end{itemize}

\subsubsection{Output Values}
For the generic function \texttt{names}, \texttt{NULL} or a character
vector of the same length as \texttt{x}.

\subsubsection{Example}
Attach sample data:  
\begin{verbatim}
> data(turnout)
\end{verbatim}
Return the variables in the data frame:  
\begin{verbatim}
> names(turnout)
\end{verbatim}
Generate Zelig objects:
\begin{verbatim}
> z.out <- zelig(vote ~ age + income, model = "logit", data = turnout)
> x.out <- setx(z.out)
> s.out <- sim(z.out, x = x.out)
\end{verbatim}
Extract names from objects of class zelig:
\begin{verbatim}
> names(z.out)
> names(s.out)
> names(summary(s.out))
\end{verbatim}

\subsubsection{See Also}

Additional information is available in {\tt help(names)}.

%%% Local Variables: 
%%% mode: latex
%%% TeX-master: t
%%% End: 
