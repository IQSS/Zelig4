\chapter{Main Commands}\label{s:main}

Help for each command in Zelig and R is available through {\tt
  help.zelig()}.  For example, typing {\tt help.zelig(setx)} will
launch a web browser with the appropriate reference manual page for
the {\tt setx()} command.  (Occasionally, you may need to use, for example, {\tt
  help(print)} rather than {\tt help.zelig(print)}, to access the R
  help page instead of the default Zelig help page.)  

\documentclass[oneside,letterpaper,12pt]{book}
\usepackage{Rd}
%\usepackage{Sweave}
%\usepackage{/usr/lib64/R/share/texmf/Sweave}
%\usepackage{/usr/share/R/texmf/Sweave}
\usepackage{bibentry}
\usepackage{upquote}
\usepackage{graphicx}
\usepackage{natbib}
\usepackage[reqno]{amsmath}
\usepackage{amssymb}
\usepackage{amsfonts}
\usepackage{amsmath}
\usepackage{verbatim}
\usepackage{epsf}
\usepackage{url}
\usepackage{html}
\usepackage{dcolumn}
\usepackage{multirow}
\usepackage{fullpage}
\usepackage{lscape}
\usepackage[all]{xy}

\usepackage{csquotes}
% \usepackage[pdftex, bookmarksopen=true,bookmarksnumbered=true,
%   linkcolor=webred]{hyperref}
\bibpunct{(}{)}{;}{a}{}{,}
\newcolumntype{.}{D{.}{.}{-1}}
\newcolumntype{d}[1]{D{.}{.}{#1}}
\htmladdtonavigation{
  \htmladdnormallink{%
    \htmladdimg{http://gking.harvard.edu/pics/home.gif}}
  {http://gking.harvard.edu/}}
\newcommand{\MatchIt}{{\sc MatchIt}}
\newcommand{\hlink}{\htmladdnormallink}
\newcommand{\Sref}[1]{Section~\ref{#1}}
\newcommand{\fullrvers}{2.5.1}
\newcommand{\rvers}{2.5}
\newcommand{\rwvers}{R-2.5.1}
%\renewcommand{\bibentry}{\citealt}

\bodytext{ BACKGROUND="http://gking.harvard.edu/pics/temple.jpg"}
\setcounter{tocdepth}{2}

\usepackage{Sweave}
\title{Zelig: Everyone's Statistical Software\thanks{The current
    version of this software is available at
    \texttt{http://gking.harvard.edu/zelig/}, free of charge and
    open-source (under the terms of the GNU GPL, v. 2).}}
\author{Kosuke
  Imai\thanks{Assistant Professor, Department of Politics, Princeton
    University (Corwin Hall, Department of Politics, Princeton
    University, Princeton NJ 08544; \texttt{http://Imai.Princeton.Edu},
    \texttt{KImai@Princeton.Edu}).}
\and %
Gary King\thanks{Albert J. Weatherhead III University Professor, Harvard
  University (Institute for Quantitative Social Sciences, 1737 Cambridge 
Street, Harvard University, Cambridge MA 02138;
  \texttt{http://GKing.Harvard.Edu}, \texttt{King@Harvard.Edu}, (617)
  495-2027).}
\and %
Olivia Lau\thanks{Ph.D.\ Candidate, Department of Government, Harvard
  University (1737 Cambridge Street, Cambridge MA 02138;
  \texttt{http://www.people.fas.harvard.edu/\~\,olau},
  \texttt{OLau@Fas.Harvard.Edu}).}}

% rbuild: replace 'Version ' '\\' Version
\date{Version \\ \today}

\begin{document}
\maketitle
\begin{rawhtml}
  <p>
  [Also available is a downloadable <a href="/zelig/docs/zelig.pdf">PDF</a>
  version of this entire document]
\end{rawhtml}

\tableofcontents

\nobibliography*

\include{acknowledgments}

\input{intro}

\part[User's Guide]{User's Guide}

\input{install}
\input{syntax}
\input{Zcommands}
\input{graphs}

\part{Advanced Zelig Uses}

\chapter{R Objects}\label{a:R}

In R, objects can have one or more classes, consisting of the class of
the scalar value and the class of the data structure holding the
scalar value.  Use the {\tt is()} command to determine what an object
\emph{is}.  If you are already familiar with R objects, you may skip
to \Sref{load.data} for loading data, or \Sref{s:commands} for a
description of Zelig commands.

\section{Scalar Values}\label{variables}

R uses several classes of scalar values, from which it constructs 
larger data structures.  R is highly class-dependent: certain
operations will only work on certain types of values or certain types
of data structures.  We list the three basic types of scalar values
here for your reference:  

\begin{enumerate}
\item \textbf{Numeric} is the default value type for most numbers.  An
  \texttt{integer} is a subset of the \texttt{numeric} class, and may
  be used as a \texttt{numeric} value.  You can perform
  any type of math or logical operation on numeric values,
  including:
\begin{verbatim}
> log(3 * 4 * (2 + pi))         # Note that pi is a built-in constant, 
   [1] 4.122270                 #   and log() the natural log function.
> 2 > 3                         # Basic logical operations, including >,
   [1] FALSE                    #   <, >= (greater than or equals), 
                                #   <= (less than or equals), == (exactly 
                                #   equals), and != (not equals). 
> 3 >= 2 && 100 == 1000/10      # Advanced logical operations, including
   [1] TRUE                     #   & (and), && (if and only if), | (or), 
                                #   and || (either or).
\end{verbatim}
  Note that \texttt{Inf} (infinity), \texttt{-Inf} (negative
  infinity), \texttt{NA} (missing value), and \texttt{NaN} (not a
  number) are special numeric values on which most math operations
  will fail.  (Logical operations will work, however.)

\item \textbf{Logical} operations create logical values of either
  \texttt{TRUE} or \texttt{FALSE}.  To convert logical values to
  numerical values, use the \texttt{as.integer()} command:
\begin{verbatim}
> as.integer(TRUE)
   [1] 1 
> as.integer(FALSE)
   [1] 0
\end{verbatim}
\item \textbf{Character} values are text strings.  For example, 
\begin{verbatim}
> text <- "supercalafragilisticxpaladocious"
> text
[1] "supercalafragilisticxpaladocious"
\end{verbatim}
  assigns the text string on the right-hand side of the \texttt{<-} to
  the named object in your workspace.  Text strings are primarily used
  with data frames, described in the next section.  R always returns
  character strings in quotes.
\end{enumerate}

\section{Data Structures}

\subsection{Arrays}

Arrays are data structures that consist of only one type of scalar
value (e.g., a vector of character strings, or a matrix of numeric
values).  The most common versions, one-dimensional and
two-dimensional arrays, are known as \emph{vectors} and
\emph{matrices}, respectively.  

\subsubsection*{Ways to create arrays}
\begin{enumerate}
\item Common ways to create {\bf vectors} (or one-dimensional arrays)
include:  
\begin{verbatim}
> a <- c(3, 7, 9, 11)    # Concatenates numeric values into a vector
> a <- c("a", "b", "c")  # Concatenates character strings into a vector
> a <- 1:5               # Creates a vector of integers from 1 to 5 inclusive  
> a <- rep(1, times = 5) # Creates a vector of 5 repeated `1's
\end{verbatim}
To manipulate a vector:  
\begin{verbatim}
> a[10]                # Extracts the 10th value from the vector `a'
> a[5] <- 3.14         # Inserts 3.14 as the 5th value in the vector `a'
> a[5:7] <- c(2, 4, 7) # Replaces the 5th through 7th values with 2, 4, and 7
\end{verbatim}
\emph{Unlike} larger arrays, vectors can be extended without first
creating another vector of the correct length.  Hence, 
\begin{verbatim}
> a <- c(4, 6, 8) 
> a[5] <- 9       # Inserts a 9 in the 5th position of the vector,
                  #  automatically inserting an `NA' values position 4 
\end{verbatim}

\item \label{factors} A {\bf factor vector} is a special type of
vector that allows users to create $j$ indicator variables in one
vector, rather than using $j$ dummy variables (as in Stata or SPSS).  R
creates this special class of vector from a pre-existing vector {\tt x}
using the {\tt factor()} command, which separates {\tt x} into levels
based on the discrete values observed in {\tt x}.  These values may be
either integer value or character strings.  For example,
\begin{verbatim}
> x <- c(1, 1, 1, 1, 1, 2, 2, 2, 2, 9, 9, 9, 9)
> factor(x)
   [1] 1 1 1 1 1 2 2 2 2 9 9 9 9
   Levels: 1 2 9
\end{verbatim}
By default, {\tt factor()} creates unordered factors, which are
  treated as discrete, rather than ordered, levels.  Add the optional
  argument {\tt ordered = TRUE} to order the factors in the vector:
\begin{verbatim}  
> x <- c("like", "dislike", "hate", "like", "don't know", "like", "dislike")
> factor(x, levels = c("hate", "dislike", "like", "don't know"),
+        ordered = TRUE)
  [1] like    dislike    hate    like   don't know   like   dislike   
Levels: hate < dislike < like < don't know
\end{verbatim}
  The {\tt factor()} command orders the levels according to the order in
  the optional argument {\tt levels}.  If you omit the levels command,
  R will order the values as they occur in the vector.  Thus, omitting
  the {\tt levels} argument sorts the levels as {\tt like < dislike <
    hate < don't know} in the example above.  If you omit one or more
  of the levels in the list of levels, R returns levels values of {\tt
    NA} for the missing level(s):
\begin{verbatim}
> factor(x, levels = c("hate", "dislike", "like"), ordered = TRUE)
  [1] like    dislike hate    like    <NA>    like    dislike
Levels: hate < dislike < like
\end{verbatim}
Use factored vectors within data frames for plotting (see
\Sref{ss:draw}), to set the values of the explanatory variables using
{\tt setx} (see \Sref{s:main}) and in the ordinal logit and multinomial
logit models (see \Sref{s:models}).  

\item Build {\bf matrices} (or two-dimensional arrays) from vectors
(one-dimensional arrays).  You can create a matrix in two ways:   
\begin{enumerate}
\item From a vector: Use the command \texttt{matrix(vector, nrow
    =} $k$\texttt{, ncol =} $n$\texttt{)} to create a $k \times n$
    matrix from the vector by filling in the columns from left to
    right.  For example,
\begin{verbatim}
> matrix(c(1,2,3,4,5,6), nrow = 2, ncol = 3)
        [,1] [,2] [,3]       # Note that when assigning a vector to a
   [1,]    1    3    5       #  matrix, none of the rows or columns 
   [2,]    2    4    6       #  have names.  
\end{verbatim}
\item From two or more vectors of length $k$: Use \texttt{cbind()} to
  combine $n$ vectors vertically to form a $k \times n$ matrix, 
  or \texttt{rbind()} to combine $n$ vectors horizontally to form a
  $n \times k$ matrix.  For example:
\begin{verbatim}
> x <- c(11, 12, 13)         # Creates a vector `x' of 3 values.
> y <- c(55, 33, 12)         # Creates another vector `y' of 3 values.  
> rbind(x, y)                # Creates a 2 x 3 matrix.  Note that row
     [,1] [,2] [,3]          #  1 is named x and row 2 is named y, 
   x   11   12   13          #  according to the order in which the
   y   55   33   12          #  arguments were passed to rbind().
> cbind(x, y)                # Creates a 3 x 2 matrix.  Note that the
         x  y                #  columns are named according to the
   [1,] 11 55                #  order in which they were passed to
   [2,] 12 33                #  cbind().  
   [3,] 13 12
\end{verbatim}
\end{enumerate}
R supports a variety of matrix functions, including: \texttt{det()},
which returns the matrix's determinant; \texttt{t()}, which transposes
the matrix; \texttt{solve()}, which inverts the the matrix; and
\texttt{\%*\%}, which multiplies two matricies.  In addition, the
\texttt{dim()} command returns the dimensions of your matrix.  As with
vectors, square brackets extract specific values from a matrix and the
assignment mechanism \texttt{<-} replaces values.  For example:
\begin{verbatim}
> loo[,3]     		  # Extracts the third column of loo.  
> loo[1,]      		  # Extracts the first row of loo.  
> loo[1,3] <- 13          # Inserts 13 as the value for row 1, column 3.  
> loo[1,] <- c(2,2,3)     # Replaces the first row of loo.  
\end{verbatim}
If you encounter problems replacing rows or columns, make sure that
the \texttt{dims()} of the vector matches the \texttt{dims()} of the
matrix you are trying to replace.  

\item An \textbf{n-dimensional array} is a set of stacked matrices of identical
  dimensions.  For example, you may create a three dimensional array
  with dimensions $(x, y, z)$ by stacking $z$ matrices each with $x$
  rows and $y$ columns.  
\begin{verbatim}
> a <- matrix(8, 2, 3)       # Creates a 2 x 3 matrix populated with 8's.
> b <- matrix(9, 2, 3)       # Creates a 2 x 3 matrix populated with 9's.
> array(c(a, b), c(2, 3, 2)) # Creates a 2 x 3 x 2 array with the first
   , , 1                     #  level [,,1] populated with matrix a (8's),
                             #  and the second level [,,2] populated 
        [,1] [,2] [,3]       #  with matrix b (9's).  
   [1,]    8    8    8       
   [2,]    8    8    8       # Use square brackets to extract values.  For
                             #  example, [1, 2, 2] extracts the second
   , , 2                     #  value in the first row of the second level.
                             # You may also use the <- operator to 
        [,1] [,2] [,3]       #  replace values.  
   [1,]    9    9    9
   [2,]    9    9    9
\end{verbatim}
If an array is a one-dimensional vector or two-dimensional matrix, R
  will treat the array using the more specific method.  
\end{enumerate}

Three functions especially helpful for
arrays:  
\begin{itemize}
\item {\tt is()} returns both the type of scalar value that populates
the array, as well as the specific type of array (vector, matrix, or
array more generally).
\item {\tt dims()} returns the size of an array, where 
\begin{verbatim}
> dims(b) 
 [1]  33  5
\end{verbatim} 
indicates that the array is two-dimensional (a matrix), and has 33
rows and 5 columns.  
\item The single bracket \verb|[ ]| indicates specific values in the
array.  Use commas to indicate the index of the specific values you
would like to pull out or replace:  
\begin{verbatim}
> dims(a)
 [1]  14
> a[10]       # Pull out the 10th value in the vector `a'
> dims(b) 
 [1]  33  5
> b[1:12, ]   # Pull out the first 12 rows of `b' 
> c[1, 2]     # Pull out the value in the first row, second column of `c'
> dims(d)
 [1]  1000  4  5
> d[ , 3, 1]  # Pulls out a vector of 1,000 values 
\end{verbatim}
\end{itemize} 

\subsection{Lists}

Unlike arrays, which contain only one type of scalar value, lists
are flexible data structures that can contain heterogeneous value types
and heterogeneous data structures.  Lists are so flexible that one list
can contain another list.  For example, the list {\tt output} can
contain {\tt coef}, a vector of regression coefficients; {\tt
variance}, the variance-covariance matrix; and another list {\tt terms}
that describes the data using character strings.  Use the {\tt
names()} function to view the named elements in a list, and to extract
a named element, use
\begin{verbatim}
> names(output)
 [1] coefficients   variance   terms
> output$coefficients
\end{verbatim} %$
For lists where the elements are not named, use double square brackets
\verb|[[ ]]| to extract elements:  
\begin{verbatim}
> L[[4]]      # Extracts the 4th element from the list `L'
> L[[4]] <- b # Replaces the 4th element of the list `L' with a matrix `b'
\end{verbatim}

Like vectors, lists are flexible data structures that can be extended
without first creating another list of with the correct number of
elements:  
\begin{verbatim}
> L <- list()                      # Creates an empty list
> L$coefficients <- c(1, 4, 6, 8)  # Inserts a vector into the list, and 
                                   #  names that vector `coefficients' 
				   #  within the list
> L[[4]] <- c(1, 4, 6, 8)          # Inserts the vector into the 4th position
                                   #  in the list.  If this list doesn't 
                                   #  already have 4 elements, the empty 
                                   #  elements will be `NULL' values
\end{verbatim} %$
Alternatively, you can easily create a list using objects that already
exist in your workspace:  
\begin{verbatim}
> L <- list(coefficients = k, variance = v) # Where `k' is a vector and
                                            #   `v' is a matrix
\end{verbatim}

\subsection{Data Frames}

A data frame (or data set) is a special type of list in which each
variable is constrained to have the same number of observations.  A
data frame may contain variables of different types (numeric,
integer, logical, character, and factor), so long as each variable has
the same number of observations. 

Thus, a data frame can use both matrix commands and list commands to
manipulate variables and observations.  
\begin{verbatim}
> dat[1:10,]         # Extracts observations 1-10 and all associated variables  
> dat[dat$grp == 1,] # Extracts all observations that belong to group 1 
> group <- dat$grp   # Saves the variable `grp' as a vector `group' in
                     #   the workspace, not in the data frame
> var4 <- dat[[4]]   # Saves the 4th variable as a `var4' in the workspace
\end{verbatim} 

For a comprehensive introduction to data frames and recoding data, see
\Sref{load.data}.

\subsection{Identifying Objects and Data Structures}

Each data structure has several \emph{attributes} which describe it.
Although these attributes are normally invisible to users (e.g., not
printed to the screen when one types the name of the object), there are
several helpful functions that display particular attributes:  
\begin{itemize}
\item For arrays, {\tt dims()} returns the size of each dimension.  
\item For arrays, {\tt is()} returns the scalar value type and
specific type of array (vector, matrix, array).  For more complex data
structures, {\tt is()} returns the default methods (classes) for that object. 
\item For lists and data frames, {\tt names()} returns the variable
names, and {\tt str()} returns the variable names and a short
description of each element.  
\end{itemize}  
For almost all data types, you may use {\tt summary()} to get summary
statistics.  



%%% Local Variables: 
%%% mode: latex
%%% TeX-master: "zelig"
%%% End: 

\chapter{Programming Statements}

This chapter introduces the main programming commands.  These include
functions, if-else statements, for-loops, and special procedures for
managing the inputs to statistical models.  

\section{Functions}

Functions are either built-in or user-defined sets of encapsulated
commands which may take any number of arguments.  Preface a function
with the {\tt function} statement and use the {\tt <-}
operator to assign functions to objects in your workspace.  

You may use functions to run the same procedure on different objects
in your workspace.  For example, 
\begin{verbatim}
check <- function(p, q) { 
 result <- (p - q)/q
 result
 }
\end{verbatim}
is a simple function with arguments {\tt p} and {\tt q} which
calculates the difference between the $i$th elements of the vector
{\tt p} and the $i$th element of the vector {\tt q} as a proportion of
the $i$th element of {\tt q}, and returns the resulting vector.  For
example, {\tt check(p = 10, q = 2)} returns 4.  You may omit the
descriptors as long as you keep the arguments in the correct order:
{\tt check(10, 2)} also returns 4.  You may also use other objects as
inputs to the function.  If {\tt again = 10} and {\tt really = 2},
then {\tt check(p = again, q = really)} and {\tt check(again, really)}
also returns 4.

Because functions run commands as a set, you should make sure that
each command in your function works by testing each line of the
function at the R prompt.

\section{If-Statements}

Use {\tt if} (and optionally, {\tt else}) to control the flow of R
functions.  For example, let {\tt x} and {\tt y} be scalar numerical
values:  
\begin{verbatim}
if (x == y) {                # If the logical statement in the ()'s is true,  
  x <- NA                    #  then `x' is changed to `NA' (missing value). 
}
else {                       # The `else' statement tells R what to do if  
  x <- x^2                   #  the if-statement is false.  
} 
\end{verbatim}
As with a function, use {\tt \{} and {\tt \}} to define the set of commands associated
with each if and else statement.  (If you include if
statements inside functions, you may have multiple sets of nested
curly braces.)

\section{For-Loops}

Use {\tt for} to repeat (loop) operations.  Avoiding loops by using matrix
or vector commands is usually faster and more elegant, but loops are
sometimes necessary to assign values.  If you are using a loop to
assign values to a data structure, you must first initialize an empty
data structure to hold the values you are assigning.

Select a data structure compatible with the type of output your loop
will generate.  If your loop generates a scalar, store it in a vector
(with the $i$th value in the vector corresponding to the the $i$th run
of the loop).  If your loop generates vector output, store them as
rows (or columns) in a matrix, where the $i$th row (or column)
corresponds to the $i$th iteration of the loop.  If your output
consists of matrices, stack them into an array.  For list output (such
as regression output) or output that changes dimensions in each
iteration, use a list.  To initialize these data structures, use:
\begin{verbatim}
> x <- vector()                          # An empty vector of any length.
> x <- list()                            # An empty list of any length.  
\end{verbatim}
The {\tt vector()} and {\tt list()} commands create a vector or list
of any length, such that assigning {\tt x[5] <- 15} automatically
creates a vector with 5 elements, the first four of which are empty
values ({\tt NA}).  In contrast, the {\tt matrix()} and {\tt array()}
commands create data structures that are restricted to their original
dimensions.  
\begin{verbatim}
> x <- matrix(nrow = 5, ncol = 2)  # A matrix with 5 rows and 2 columns.
> x <- array(dim = c(5,2,3))       # A 3D array of 3 stacked 5 by 2 matrices.
\end{verbatim}
If you attempt to assign a value at $(100, 200, 20)$ to either of
these data structures, R will return an error message (``subscript is
out of bounds'').  R does not automatically extend the
dimensions of either a matrix or an array to accommodate additional
values.  

\paragraph{Example 1: Creating a vector with a logical statement} 
\begin{verbatim}
x <- array()             # Initializes an empty data structure.  
for (i in 1:10) {        # Loops through every value from 1 to 10, replacing
  if (is.integer(i/2)) { #  the even values in `x' with i+5.
    x[i] <- i + 5 
  }      
}                        # Enclose multiple commands in {}.  
\end{verbatim}
You may use {\tt for()} inside or outside of functions.  

\paragraph{Example 2: Creating dummy variables by hand}  

\label{dummy}You may also use a loop to create a matrix of dummy
variables to append to a data frame.  For example, to generate fixed
effects for each state, let's say that you have {\tt mydata} which
contains {\tt y}, {\tt x1}, {\tt x2}, {\tt x3}, and {\tt state}, with
{\tt state} a character variable with 50 unique values.  There are
three ways to create dummy variables: 1) with a built-in R command; 2)
with one loop; or 3) with 2 for loops.  
\begin{enumerate}
\item R will create dummy variables on the fly from a single variable
with distinct values.  
\begin{verbatim}
> z.out <- zelig(y ~ x1 + x2 + x3 + as.factor(state), 
                 data = mydata, model = "ls")
\end{verbatim}
This method returns $k - 1$ indicators for $k$ states.  

\item Alternatively, you can use a loop to create dummy variables by
hand.  There are two ways to do this, but both start with the same
initial commands. Using vector commands, first create an index of for
the states, and initialize a matrix to hold the dummy variables:
\begin{verbatim}  
idx <- sort(unique(mydata$state))
dummy <- matrix(NA, nrow = nrow(mydata), ncol = length(idx))
\end{verbatim}  %$
Now choose between the two methods.  
\begin{enumerate}
\item The first method is computationally inefficient, but more intuitive for users not
accustomed to vector operations.  The first loop uses {\tt i} as in
index to loop through all the rows, and the second loop uses {\tt j}
to loop through all 50 values in the vector {\tt idx}, which
correspond to columns 1 through 50 in the matrix {\tt dummy}.
\begin{verbatim}
for (i in 1:nrow(mydata)) {
  for (j in 1:length(idx)) {
    if (mydata$state[i,j] == idx[j]) {
      dummy[i,j] <- 1
    }
    else {
      dummy[i,j] <- 0
    }
  }
}
\end{verbatim}  % $
Then add the new matrix of dummy variables to your data frame:
\begin{verbatim}
names(dummy) <- idx
mydata <- cbind(mydata, dummy) 
\end{verbatim}  

\item As you become more comfortable with vector operations, you can
replace the double loop procedure above with one loop:   
\begin{verbatim}
for (j in 1:length(idx)) { 
  dummy[,j] <- as.integer(mydata$state == idx[j])
}
\end{verbatim} %$
The single loop procedure evaluates each element in {\tt idx} against the vector {\tt mydata\$state}.  This creates a vector of $n$ {\tt
    TRUE}/{\tt FALSE} observations, which you may transform to {\tt
    1}'s and {\tt 0}'s using {\tt as.integer()}.  Assign the resulting
  vector to the appropriate column in {\tt dummy}.  Combine the {\tt
dummy} matrix with the data frame as above to complete the procedure.
\end{enumerate}
\end{enumerate}

\paragraph{Example 3: Weighted regression with subsets}  

Selecting the {\tt by} option in {\tt zelig()} partitions the data
frame and then automatically loops the specified model through each
partition.  Suppose that {\tt mydata} is a data frame with variables
{\tt y}, {\tt x1}, {\tt x2}, {\tt x3}, and {\tt state}, with {\tt
state} a factor variable with 50 unique values.  Let's say that you
would like to run a weighted regression where each observation is
weighted by the inverse of the standard error on {\tt x1}, estimated
for that observation's state.  In other words, we need 
to first estimate the model for each of the 50 states, calculate 1 /
{\sc se}({\tt x1}$_j$) for each state $j = 1, \dots, 50$, and then
assign these weights to each observation in {\tt mydata}.    
\begin{itemize}
\item Estimate the model separate for each state using the {\tt by}
option in {\tt zelig()}:  
\begin{verbatim}
z.out <- zelig(y ~ x1 + x2 + x3, by = "state", data = mydata, model = "ls")
\end{verbatim}
Now {\tt z.out} is a list of 50 regression outputs.  
\item Extract the standard error on {\tt x1} for each of the state
level regressions.  
\begin{verbatim}
se <- array()                          # Initalize the empty data structure.
for (i in 1:50) {                      # vcov() creates the variance matrix
  se[i] <- sqrt(vcov(z.out[[i]])[2,2]) # Since we have an intercept, the 2nd 
}                                      # diagonal value corresponds to x1.
\end{verbatim}
\item Create the vector of weights.  
\begin{verbatim}
wts <- 1 / se
\end{verbatim}
This vector {\tt wts} has 50 values that correspond to the 50 sets of
state-level regression output in {\tt z.out}.  
\item To assign the vector of weights to each observation, we need to
match each observation's state designation to the appropriate state.
For simplicity, assume that the states are numbered 1 through 50.
\begin{verbatim}
mydata$w <- NA            # Initalizing the empty variable
for (i in 1:50) { 
  mydata$w[mydata$state == i] <- wts[i]
} 
\end{verbatim} %$
We use {\tt mydata\$state} as the index (inside the square brackets)
to assign values to {\tt mydata\$w}.  Thus, whenever state equals 5
for an observation, the loop assigns the fifth value in the vector
{\tt wts} to the variable {\tt w} in {\tt mydata}.  If we had 500
observations in {\tt mydata}, we could use this method to match each
of the 500 observations to the appropriate {\tt wts}.

If the states are character strings instead of integers, we can use a
slightly more complex version
\begin{verbatim}
mydata$w <- NA
idx <- sort(unique(mydata$state))
for (i in 1:length(idx) { 
  mydata$w[mydata$state == idx[i]] <- wts[i]
}
\end{verbatim}   

\item Now we can run our weighted regression:  
\begin{verbatim}
z.wtd <- zelig(y ~ x1 + x2 + x3, weights = w, data = mydata, 
               model = "ls")
\end{verbatim}

\end{itemize}  


\input{newModels}
\chapter{Adding Models and Methods to Zelig}
\label{c:addingmodels}

Zelig is highly modular.  You can add methods to Zelig \emph{and}, if
you wish, release your programs as a stand-alone package.  By making
your package compatible with Zelig, you will advertise your package
and help it achieve a widespread distribution.

This chapter assumes that your model is written as a function that
takes a user-defined formula and data set (see Chapter \ref{s:new}),
and returns a list of output that includes (at the very least) the
estimated parameters and terms that describe the data used to fit the
model.  You should choose a class (either S3 or S4 class) for this
list of output, and provide appropriate methods for generic functions
such as {\tt summary()}, {\tt print()}, {\tt coef()} and {\tt vcov()}.

To add new models to Zelig, you need to provide six R functions,
illustrated in Figure \ref{add}.  Let {\tt mymodel} be a new model
with class {\tt "myclass"}. 

\begin{figure*}[h!]
\caption{Six functions (solid boxes) to implement a new Zelig model}
\label{add}
\begin{center}
\setlength{\unitlength}{0.5mm}
\begin{picture}(160,170)(0,0)
\linethickness{0.75pt}

\put(0,166){Estimate}

\put(70,162){\line(0,-1){42}}
\put(50,162){\dashbox{2}(40,12){{\tt zelig()}}}
\multiput(70,144)(0,-24){2}{\line(1,0){9}}
\put(80,138){\framebox(83,12){(1) {\tt zelig2mymodel()}}}
\put(80,114){\framebox(57,12){(2) {\tt mymodel()}}}

\put(0,96){Interpret}

\put(70,92){\line(0,-1){42}}
\put(50,92){\dashbox{2}(40,12){{\tt sim()}}}
\multiput(70,74)(0,-24){2}{\line(1,0){9}}
\put(80,68){\framebox(83,12){(3) {\tt param.myclass()}}}
\put(80,44){\framebox(69,12){(4) {\tt qi.myclass()}}}

\put(0,26){Plot}

\put(50,0){\framebox(105,12){(6) {\tt plot.zelig.mymodel()}}}

\end{picture}
\end{center}
\end{figure*}

These functions are as follows:  
\begin{enumerate}
\item {\tt zelig2mymodel()} translates {\tt zelig()} arguments into
the arguments for {\tt mymodel()}.
\item {\tt mymodel()} estimates your statistical procedure.
\item {\tt param.myclass()} simulates parameters for your model.
Alternatively, if your model's parameters consist of one vector with a
correspondingly observed variance-covariance matrix, you may write
\emph{two} simple functions to substitute for {\tt param.myclass()}:  
\begin{enumerate}
\item {\tt coef.myclass()} to extract the coefficients from your model
output, and
\item {\tt vcov.myclass()} to extract the variance-covariance matrix
from your model.  
\end{enumerate}
\item {\tt qi.myclass()} calculates expected values, simulates
predicted values, and generates other quantities of interest for your
model (applicable only to models that take explanatory variables).  
\item {\tt plot.zelig.mymodel()} to plot the simulated quantities of
interest from your model.  
\item A {\bf reference manual page} to document the model.
  (See~\Sref{s:format})
\item A function ({\tt describe.mymodel()}) describing the inputs to
your model, for use with a graphical user interface.  (See \Sref{describe.mymodel}).  
\item An optional {\bf demo script} {\tt mymodel.R} which contains commented code for
  the models contained in the example section of your reference manual
  page.
\end{enumerate}

\section{Making the Model Compatible with Zelig}\label{compatible}

Developers can develop a model, write the model-fitting function, and test it within the Zelig framework without explicit intervention from the Zelig team.  This modularity relies on two R programming conventions:

\begin{enumerate}
	\item{{\bf wrappers}, which pass arguments from R functions to other R functions or foreign function calls (such as in C, C++, or Fortran).  This step is facilitated by - as will be explained in detail in the upcoming chapter - the {\tt zelig2} functions.}
	\item{{\bf classes}, which tell generic functions how to handle objects of a given class.  For a statistical model to be compliant with Zelig, the model-fitting function \emph{must} return a classed object.}
\end{enumerate}

Zelig implements a unique and simple method for incorporating existing statistical models which lets developers test \emph{within} the Zelig framework \emph{without} any modification of both their own code or the {\tt zelig} function itself.  The heart of this procedure is the {\tt zelig2} function, which acts as an interface between the {\tt zelig} function and the existing statistical model.  That is, the {\tt zelig2} function maps the user-input from the {\tt zelig} function into input for the existing statistical model's constructor function.  Specifically, a Zelig model requires:

\begin{enumerate}
	\item{An existing statistical model, which is invoked through a function call and returns an object}
	\item{A {\tt zelig2} function which maps user-input from the {\tt zelig} function to the existing statistical model}
	\item{A name for the {\tt \bf zelig} model, which can differ from the original name of the statistical model.}
\end{enumerate}

%% MOVED 1:39

% THE ZELIG2 FUNCTION
\subsubsection{The \emph{zelig2} Function}
The following sections explain how to write a {\tt zelig2} function, given an arbitrary statistical model.  In the illustrative examples, the following conventions are used:

\begin{description}
	\item[model]{will refer to the name of the \emph{Zelig} model, not the name of the existing model - though these two names are not necessarily different.  If the developer names his model ``logit'' then model refers to ``logit''.}
	\item[model\_function]{will refer to the name of function that produces the existing statistical model.  If the developer is writing a wrapper for R's built-in logit function, then \emph{model\_function} refers to ``glm''.}
	\item[zelig2model]{will refer to the name of the {\tt zelig2} function.  If the developer names his model ``logit'', then \emph{zelig2model} refers to ``zelig2logit''.}
\end{description}



% WRITING THE ZELIG2 FUNCTION
\subsubsection{Writing the \emph{zelig2} Function}

The {\tt zelig2} function should follow several specific conventions:

\begin{enumerate}
	\item{The {\tt zelig2model} function should be simply named \emph{zelig2model}, where \emph{model} is the chosen name for the zelig package}
	\item{The {\tt zelig2model} function itself should have arguments that list entirety of possible inputs to the {\tt model\_function}}
	\item{The {\tt zelig2model} function should return a list of key-value pairs that represent the map from {\tt zelig} input to {\tt model\_function} input}
\end{enumerate}


% EXAMPLE USING zelig2logit
\subsubsection{Example of a \emph{zelig2} Function}

\begin{verbatim}
zelig2logit <- function(model, formula, ..., data, weights=NULL)
  alist(glm,                             # [1]
        "formula",                       # [2]
        "data",
        "weights",
        family = binomial(link="logit"), # [3]
        model  = FALSE
        )
\end{verbatim}

% EXPLANATION OF zelig2logit code
The bracketed numbers in the previous example correspond to the numbers below:

\begin{enumerate}
	\item{The first entry of the returned list in a {\tt zelig2} call represents the name of the function call.  That is, it is the name of the function that the {\tt zelig2} function is wrapping.}
	\item{Any character-string that does not have a key associated with it will simply forward user-input to that specified parameter.  That is, it will expect the {\tt zelig} function call to supply the value of character-string.}
	\item{Any key-value pair will explicitly set the parameter specified \emph{by} the key \emph{to} its value.  That is, the left side of the equal specifies a parameter of the existing model and the value specifies that parameter's value.}
\end{enumerate}

% Equivalent code to the example
% ...

\subsubsection{Equivalent Function Call to the Example \emph{zelig2} Function}
The above code is create an interface for the glm function (which returns a generalized linear model, specified by the {\tt family} parameter.  In effect, the above code maps user-input into the {\tt zelig} function to the following call to the {\tt glm} function.

\begin{verbatim}
glm(formula = formula,
    data = data,
    weights = weights,
    family = binomial("logit"),
    model = FALSE
    )
\end{verbatim}

% Explanation of Above Code
% ....
\subsubsection{Explanation of the Example Code}

\subsubsection{Return Values of a \emph{zelig2} Function}

A {\tt zelig2model} function must always return either a list or an alist as its return value.  In most situations, alist is preferred, since it will place the results symbolically within the list.  This typically results in much cleaner looking print methods.

The entries of the returned list have the following format:
\begin{description}
	\item[first entry]{is a function or character string specifying the {\tt model\_function}}
	\item[character-strings with no key]{represent parameters that are simply forwarded to the {\tt model\_function} from the {\tt zelig} function.  Nothing will be computed, etc. on these values}
	\item[key-value pairs]{Represent an explicitly set value for the parameter that matches ``key''.  This is useful for parameters that do not require user-input, or user-input that must be changed to interface correctly with the existing model}
\end{description}


% setx function
%
\subsubsection{The \emph{setx} Function}
The Zelig software suite can typically work with the arbitrary {\tt zelig} object, so long as several functions are defined with the following names and outputs:

\begin{description}
	\item[terms]{returns a list of terms used to interpret {\tt data.frame} objects.  \emph{See {\tt help(terms)} for information concerning creating a {\tt terms} function}.}
	\item[formula]{returns the formula passed into the existing statistical model.  This should typically be the exact value of {\tt formula} passed into the {\tt zelig} call.  This will not be the case, if extra parsing or editing of the {\tt formula} argument is done within the {\tt zelig2} function.}
\end{description}

If a model is failing to work with the {\tt setx} function, then the developer must either create valid versions of the the missing or undefined functions; or entirely rewrite the {\tt setx.model} function (where {\tt model} is the name of the Zelig model being written) to perform the correct type of operations.  Please note that the {\tt setx.model} function must return an object of class ``setx'' and the original class of the statistical model.  Please see the source code of {\tt setx.default} for reference.

% SIM function
\subsubsection{The \emph{sim} Function}

Simulating \emph{quantities of interest} is an integral part of interpreting model results. To use the functionality built into the Zelig {\tt sim} procedure, you need to provide a way to simulate parameters (called a {\tt param} function), and a method for calculating or drawing \emph{quantities of interest} from the simulated parameters (called a {\tt qi} function).

% boot is probably not working right now
% \section{The \emph{boot} Function}

% param function
% ...
\subsubsection{The \emph{param} Function}
Whether you choose to use the default method, or write a model- specific method for simulating parameters, these functions require the same two inputs:

\begin{itemize}
	\item{{\tt object}, the object returned by the {\tt zelig} function}
	\item{{\tt num}, the number of simulations to perform}
\end{itemize}

When creating the return value for the function, several options are available which are explored in detail in the following subsections.

% FIRST METHOD TO SIMULATE PARAMETERS
\subsubsection{Simplest Method: Returning a Vector/Matrix of Samples}

Simply returning a vector of samples from the fitted model is by far the simplest approach to constructing a ``param'' function.  While it is quick-and-easy to implement, it lacks the descriptiveness of other methods, and restricts the ability of the developer to make use of API functions.  While this my be an insignificant loss to some, it typically will result in the developer writing harder-to-read code.  It is recommended that the developer that the developer use the slightly more sophisticated methods when possible.

\subsubsection{Example Code for Simple Vector/Matrix Method}
\begin{verbatim}
param.default <- function(z, num) {
  mvrnorm(n=num, mu=coef(z), Sigma=vcov(z))
}
\end{verbatim}

\subsubsection{Explanation}
``mvrnorm'' is the function that takes samples from a multivariate-normal distribution.  In the above example, the \emph{param.default} function returns a matrix with row size \emph{num} and column size equivalent to that of the variance-covariance matrix of the fitted model.  Zelig knows how to convert matrices of this type to a parameters object.

% SECOND METHOD (LIST METHOD)
% LIST METHODS
% ------------
% LOGIT EXAMPLE
\subsubsection{List Method: Returning an Indexed List of Parameters}

While the simple method of returning a vector or matrix from a \emph{param} function is extremely simple, it has no method for setting link or link-inverse functions for use within the actual simulation process.  That is, it does not provide a clear, easy-to-read method for simulating \emph{quantities of interest}.  By returning an indexed list - or a parameters object - the developer can provide clearly labeled and stored link and link-inverse functions, as well as, ancillary parameters.


\subsubsection{Example of Indexed List Method with \emph{fam} Object Set}

\begin{verbatim}
param.logit <- function(z, num)
  list(
       coef  = mvrnorm(n=num, mu=coef(z), Sigma=vcov(z)),
       alpha = NULL,
       fam   = binomial(link="logit")
       )
\end{verbatim}


\subsubsection{Explanation of Indexed List with \emph{fam} Object Set Example}

The above example shows how link and link-inverse functions (for a ``logit'' model) can be set using a ``family'' object.  Family objects exist for most statistical models - logit, probit, normal, Gaussian, et cetera - and come preset with values for link and link-inverses.  This method does not differ immensely from the simple, vector-only method; however, it allows for the use of several API functions - \emph{link}, \emph{linkinv}, \emph{coef}, \emph{alpha} - that improve the readability and simplicity of the model's implementation.

The \emph{param} function and the \emph{parameters} class offer methods for automating and simplifying a large amount of repetitive and cumbersome code that may come with building the arbitrary statistical model.  While both are in principle entirely optional - so long as the \emph{qi} function is well-written - they serve as a means to quickly and elegantly implement Zelig models.


% ANOTHER METHOD
\subsubsection{Example of Indexed List Method (with \emph{link} Function) Set}

\begin{verbatim}
param.poisson <- function(z, num) {
  list(
       coef = mvrnorm(n=num, mu=coef(z), Sigma=vcov(z)),
       link = log,
             
       # because ``link'' is set,
       # the next line is purely optional
       linkinv = exp
       )
}
\end{verbatim}


\subsubsection{Explanation of Indexed List (with \emph{link} Function) Example}

The above example shows how a \emph{parameters} object can be created with by explicitly setting the statistical model's link function.  The \emph{linkinv} parameter is purely optional, since Zelig will create a numerical inverse if it is undefined.  However, the computation of the inverse is typically slower than non-iterative methods.  As a result of this, if the link-inverse is known, it should be set, using the \emph{linkinv} parameter.

The above example can also contain an \emph{alpha} parameter, in order to store important ancillary parameters - mean, standard deviation, gamma-scale, etc. - that would be necessary in the computation of \emph{quantities of interest}.


% QUANTITIES OF INTEREST
\subsection{The \emph{qi} Function}

% Introduction Material
For any Zelig model, the \emph{qi} function is ultimately the most important piece of code that must be written; it describes the actual process which simulates the \emph{quantities of interest}.  Because of the nature of this process - and the gamut of statistical packages and their underlying statistical model - it is rare that the simulation process can be generalized for arbitrary fitted models.  Despite this, it is possible to break down the simulation process into smaller steps.

% Calculating
\subsubsection{Calculating Quantities of Interest}
All models require a model-specific method for calculating quantities of interest from the simulated parameters. For a model of class {\tt model}, the appropriate {\tt qi} function is {\tt qi.model}. This function should calculate, at the bare minimum, the following quantities of interest:

\begin{description}
	\item[E(Y:X)]{the expected values, calculated from the analytic solution for the expected value as a function of the systematic component and ancillary parameters.}
	\item[Y:X]{the predicted values, drawn from a distribution defined by the predicted values. If R does not have a built-in random generator for your function, you may take a random draw from the uniform distribution and use the inverse CDF method to calculate predicted values.}
	\item[E(Y:X1) - E(Y:X)]{first differences in the expected value, calculated by subtracting the expected values given the specified {\tt x} from the expected values given {\tt x1}.}
\end{description}


% QI FUNCTION SIGNATURE
\subsection{The Function's Signature}


% quick intro
The \emph{qi} function's signature accepts 4 parameters:


%
\begin{description}
	\item[@object:]{An object of type ``\emph{zelig}''.  This wraps the fitted model in the slot ``result''.}
	\item[@x:]{An object of type ``\emph{setx}''.  This object is used to compute important coefficients, parameters, and features of the data.frame passed to the function call.}
	\item[@x1:]{Also an object of type ``\emph{setx}''.  This object is used in a similar fashion, however its presence allows a variety of \emph{quantities of interest} to be computed.  Notably, this is a necessary parameter to compute first-differences.}
	\item[@num:]{The number of simulations to compute}
\end{description}


% code example
\subsubsection{Code Example: \emph{qi} Function Signature}
\begin{verbatim}
qi.model <- function(z, x=NULL, x1=NULL, num=1000) {
  # start typing your code here
  # ...
  # ...
}
\end{verbatim}


Note: In the above example, the function name ``qi.\_model'' is merely a placeholder.  In order to register a \emph{qi} function with zelig, the developer must follow the naming convention qi.\emph{your mode name}, where \emph{your\_model\_name} is the name of the developer's module.  For example, if a developer titled his or her zelig module ``logit'', then the corresponding \emph{qi} function is titled ``\emph{qi.logit}''.


% QI FUNCTION RETURN VALUES
\subsubsection{The Return Value}
In order for Zelig to process the simulations, they must be returned in the following format:

\begin{verbatim}
  list(
       "TITLE OF QI 1" = val1,
       "TITLE OF QI 2" = val2,

       # any number of title-val pairs
       "TITLE OF QI N" = val.n
       )
\end{verbatim}


In the above example,\emph{val1, val2} are data.frames, matrices, or lists representing the simulations of the \emph{quantities of interests}, and \emph{title1, title2} - and any number of titles - are character-strings that will act as human-readable descriptions of the \emph{quantities of interest}.  Once results are returned in this format, Zelig will convert the results into a machine-readable format and summarize the simulations into a comprehensible format.




% EXAMPLE CODE
%

\pagebreak

\subsubsection{Example \emph{qi} function (qi.ls.R)}


\begin{verbatim}
qi.logit <- function(z, x=NULL, x1=NULL, num=1000, param=NULL) {
  coef <- coef(param)             # [1]
  link.inverse <- linkinv(param)

  eta <- coef %*% t(x)            # [2]
  theta <- matrix(link.inverse(eta), nrow = nrow(coef))

  ev <- theta                     # [3]
  pr <- matrix(nrow=nrow(theta), ncol=ncol(theta))

  ev2 <- pr2 <- fd <- NA          # [4]

  for (i in 1:ncol(theta))        # [5]
    pr[,i] <- as.character(rbinom(length(ev[,i]), 1, ev[,i]))


  if (!is.null(x1)) {             # [6]
    eta2 <- coef %*% t(x1)
    theta2 <- matrix(link.inverse(eta2), nrow = nrow(coef))

    ev2 <- theta2
    pr2 <- matrix(nrow=nrow(theta2), ncol=ncol(theta2))

    for (i in 1:ncol(theta))
      pr2[,i] <- as.character(rbinom(length(ev[,i]), 1, ev[,i]))

    fd <- ev2-ev
  }

  # return                          [7]
  list("Expected Values: E(Y|X)" = ev,
       "Predicted Values: Y|X" = pr,
       "Expected Values (for X1)" = ev2,
       "Predicted Values (for X1" = pr2,
       "First Differences: E(Y|X1) - E(Y|X)" = fd
       )
}

\end{verbatim}


% EXPLANATION OF ABOVE CODE
\subsubsection{\emph{qi.logit} Code Explanation}

The following list corresponds to the numbered comments in the {\tt qi.logit} example:

\begin{enumerate}
	\item{Extract ancillary parameters and the inverse function from the {\tt parameters} object.  These are used later in simulating the \emph{quantities of interest}}
	\item{Compute $\eta$ (eta) for the given statistical model, and apply the inverse link function to compute $\theta$ (theta)}
	\item{Store $\theta$ in a matrix}
	\item{Set our finite-differences, et cetera to {\tt NA}.  If the parameter {\tt x1} is {\tt NULL}, this will prevent unavailable \emph{quantities of interest} from being computed}
	\item{Use the ancillary parameters that have been simulated to take random draws from a binomial distribution, this simulates the predicted values of our data-set}
	\item{If the {\tt x1} parameter is not {\tt NULL}, we want to compute the expected values and predicted values for x1, as well.  This allows for that condition to be possible, while not exclusively depending on it occuring}
	\item{The return value matches titles of the \emph{quantities of interest} with their actual values}
\end{enumerate}



%\section{Getting Ready for the GUI}  
%
%Zelig can work with a variety of graphical user interfaces (GUIs).  GUIs
%work by knowing {\it a priori} what a particular model
%accepts, and presenting only those options to the user in some sort of
%graphical interface.  Thus, in order for your model to work with a
%GUI, you must describe your model in terms that the GUI can
%understand.  For models written using the guidelines in Chapter
%\ref{s:new}, your model will be compatible with (at least)
%the \hlink{Virtual Data Center}{http://thedata.org} GUI.  For
%pre-existing models, you will need to create a {\tt describe.*()}
%function for your model following the examples in
%\Sref{describe.mymodel}.  

\section{Formatting Reference Manual Pages}  \label{s:format}

One of the primary advantages of Zelig is that it fully documents the
included models, in contrast to the programming-orientation of R
documentation which is organized by function.  Thus, we ask that Zelig
contributors provide similar documentation, including the syntax and
arguments passed to {\tt zelig()}, the systematic and stochastic
components to the model, the quantities of interest, the output
values, and further information (including references).  There are
several ways to provide this information:  
\begin{itemize}
\item If you have an existing package documented using the .Rd help
  format, {\tt help.zelig()} will automatically search R-help in
  addition to Zelig help.
\item If you have an existing package documented using on-line HTML
  files with static URLs (like Zelig or MatchIt), you need to provide
  a {\tt PACKAGE.url.tab} file which is a two-column table containing
  the name of the function in the first column and the url in the
  second.  (Even though the file extension is {\tt .url.tab}, the file
  should be a tab- or space-delimited text file.)  For example:  
\begin{verbatim}
command       http://gking.harvard.edu/zelig/docs/Main_Commands.html
model         http://gking.harvard.edu/zelig/docs/Specific_Models.html
\end{verbatim}
If you wish to test to see if your {\tt .url.tab} files works, simply
place it in your R library/Zelig/data/ directory.  (You do not need to
reinstall Zelig to test your {\tt .url.tab} file.)
\item Preferred method:  You may provide a \LaTeXe\ {\tt .tex} file.  This document uses
  the book style and supports commands from the following packages:
  {\tt graphicx}, {\tt natbib}, {\tt amsmath}, {\tt amssymb}, {\tt
    verbatim}, {\tt epsf}, and {\tt html}.  Because model pages are
  incorporated into this document using {\tt $\backslash$include\{\}},
  you should make sure that your document compiles before submitting
  it.  Please adhere to the following conventions for your model page: 
  \begin{enumerate}
  \item All mathematical formula should be typeset using the {\tt
      equation*} and {\tt array}, {\tt eqnarray*}, or {\tt align}
    environments.  Please avoid {\tt displaymath}.  (It looks funny in
    html.)
  \item All commands or R objects should use the {\tt texttt}
    environment.
  \item The model begins as a subsection of a larger document, and
    sections within the model page are of sub-subsection level.
\item For stylistic consistency, please avoid using the {\tt
    description} environment.
\end{enumerate}

Each \LaTeX\ model page should include the following elements.  Let
{\tt contrib} specify the new model.

\subsubsection*{Help File Template}
\begin{verbatim}
\subsection{{\tt contrib}: Full Name for [type] Dependent Variables}
\label{contrib}

\subsubsection{Syntax}

\subsubsection{Examples}
\begin{enumerate}
\item First Example
\item Second Example
\end{enumerate}

\subsubsection{Model}
\begin{itemize}
\item The observation mechanism, if applicable.
\item The stochastic component.
\item The systematic component.
\end{itemize}

\subsubsection{Quantities of Interest}
\begin{itemize}
\item The expected value of your distribution, including the formula
  for the expected value as a function of the systemic component and
  ancillary paramters.  
\item The predicted value drawn from the distribution defined by the
       corresponding expected value.  
\item The first difference in expected values, given when x1 is specified.  
\item Other quantities of interest.
\end{itemize}

\subsubsection{Output Values}
\begin{itemize}
\item From the {\tt zelig()} output stored in {\tt z.out}, you may
  extract:
   \begin{itemize}
   \item 
   \item 
   \end{itemize}
\item From {\tt summary(z.out)}, you may extract: 
   \begin{itemize}
   \item 
   \item 
   \end{itemize}
\item From the {\tt sim()} output stored in {\tt s.out}:
   \begin{itemize}
   \item 
   \item 
   \end{itemize}
\end{itemize}

\subsubsection{Further Information}

\subsubsection{Contributors}
\end{verbatim}
\end{itemize} 



%%% Local Variables: 
%%% mode: latex
%%% TeX-master: "~/zelig/docs/commands/zelig"
%%% End: 


\part{Reference Manual}

\chapter{Main Commands}\label{s:main}

Help for each command in Zelig and R is available through {\tt
  help.zelig()}.  For example, typing {\tt help.zelig(setx)} will
launch a web browser with the appropriate reference manual page for
the {\tt setx()} command.  (Occasionally, you may need to use, for example, {\tt
  help(print)} rather than {\tt help.zelig(print)}, to access the R
  help page instead of the default Zelig help page.)  

\include{zinput}
\usepackage{Sweave}
\title{Zelig: Everyone's Statistical Software\thanks{The current
    version of this software is available at
    \texttt{http://gking.harvard.edu/zelig/}, free of charge and
    open-source (under the terms of the GNU GPL, v. 2).}}
\author{Kosuke
  Imai\thanks{Assistant Professor, Department of Politics, Princeton
    University (Corwin Hall, Department of Politics, Princeton
    University, Princeton NJ 08544; \texttt{http://Imai.Princeton.Edu},
    \texttt{KImai@Princeton.Edu}).}
\and %
Gary King\thanks{Albert J. Weatherhead III University Professor, Harvard
  University (Institute for Quantitative Social Sciences, 1737 Cambridge 
Street, Harvard University, Cambridge MA 02138;
  \texttt{http://GKing.Harvard.Edu}, \texttt{King@Harvard.Edu}, (617)
  495-2027).}
\and %
Olivia Lau\thanks{Ph.D.\ Candidate, Department of Government, Harvard
  University (1737 Cambridge Street, Cambridge MA 02138;
  \texttt{http://www.people.fas.harvard.edu/\~\,olau},
  \texttt{OLau@Fas.Harvard.Edu}).}}

% rbuild: replace 'Version ' '\\' Version
\date{Version \\ \today}

\begin{document}
\maketitle
\begin{rawhtml}
  <p>
  [Also available is a downloadable <a href="/zelig/docs/zelig.pdf">PDF</a>
  version of this entire document]
\end{rawhtml}

\tableofcontents

\nobibliography*

\include{acknowledgments}

\input{intro}

\part[User's Guide]{User's Guide}

\input{install}
\input{syntax}
\input{Zcommands}
\input{graphs}

\part{Advanced Zelig Uses}

\input{robjects}
\input{programmingstatements}
\input{newModels}
\input{addingModels}

\part{Reference Manual}

\input{refman}

\appendix

\part[Appendices]{Appendices}

%\input{distributions}

\chapter{What's New?  What's Next?}
\input{whatsnew}

\input{faq}

\bibliographystyle{asa}
\bibliography{gk,gkpubs}

\end{document}

%%% Local Variables: 
%%% mode: LaTeX
%%% TeX-master: t
%%% End:


\include{commandsRd/setx}
\include{commandsRd/sim}
\section{{\tt summary}: Summarizing Zelig Output}
\label{ss:summary}

\subsubsection{Description}

Summarize R objects using using the generic function {\tt summary()}.
R automatically recognizes the type of object (lists, data frames,
variables, etc.) and selects the appropriate {\tt summary()} function.

\subsubsection{Syntax}

\begin{verbatim}
> summary(z.out, subset = NULL, ...)
> summary(s.out, subset = NULL, CI = 95, 
          stats = c("mean", "sd", "min", "max"), ...)
\end{verbatim}
To set the desired number of digits in the summary output, use {\tt
  options(digits = x)}, where {\tt x} is the desired number of digits,
prior to using the {\tt summary()} command.

\subsubsection{Arguments for multiply-imputed {\tt zelig()} output}
For {\tt zelig()} output created with $m$ multiply-imputed data sets,
the {\tt z.out} object contains one set of output for each imputed
data set.  Options for multiply-imputed {\tt zelig()} output include:
   \begin{itemize}
   \item {\tt subset}: specifies which of the $m$ sets of output to
     summarize.  Possible arguments include:
       \begin{itemize}
       \item {\tt NULL}: (default) which produces one summary by
         averaging the coefficients and standard errors from all the
         imputed data sets using the Rubin rules (\hlink{King,
           Honaker, Joseph, Scheve
           (2000)}{http://gking.harvard.edu/files/abs/evil-abs.shtml},
         p. 53).
       \item a numeric vector: consisting of any set of numbers in
         $[1,m]$.  For example, you may use {\tt summary(z.out,
           subset = 2:4)} to view the output from the second, third,
         and fourth imputed data sets only.  
       \end{itemize}
     \item{\dots}: additional options passed to {\tt print()}.  
    \end{itemize}

\subsubsection{Arguments for subsetted {\tt zelig()} output}
If you selected the {\tt by} option, {\tt zelig()} creates an {\tt
  z.out} object with one set of regression output for each of the
unique values in the {\tt by} variable.  Options for subsetted
  analyses are:  
\begin{itemize}
\item {\tt subset}: specifies which regression output to summarize.
  You cannot summarize multiple analyses generated using the {\tt
    "by"} option in one summary, as each of the levels in the {\tt by}
  variable may have a different number of associated observations.  (A
  weighted summary would also be inappropriate, as the weighted
  coefficients would be identical to the coefficients generated
  without subsetting.)
\begin{itemize}
\item numeric vector: specifying which sets of regression output to
  summarize.  By default, {\tt summary()} for subsetted output
  sequentially summarizes the regression output for each unique value
  in the {\tt by} variable.  You may choose to summarize just the
  fifteenth unique value by typing:  {\tt summary(z.out, subset =
    15)}.  
\end{itemize}
\item {\tt \dots}: Additional arguments passed to {\tt print()}.
\end{itemize}
   
\subsubsection{Arguments for {\tt sim()} output}
Summaries of {\tt sim()} output includes these additional options:
   \begin{itemize}
     \item {\tt subset}:  Only valid for {\tt sim()} output created
       using more than one observation in {\tt x}.  You may select to
       summarize the quantities of interest created by each
       observation in {\tt x} using one of the following options: 
       \begin{itemize}
       \item {\tt NULL}: (default) produces one summary per quantity
         of interest by combining the simulations from each
         observation into a single set and then summarizing.
         \item {\tt all}:  summarizes the quantity of interest
           produced by each observation separately.  
         \item a numeric vector: specifying the observations to summarize 
       \end{itemize}
     \item {\tt CI}: A value between 0 and 100, for the percentage
       confidence interval.  The default value is 95, for a 95 percent
       confidence interval.  
     \item {\tt stats}: A vector specifying the statistics to
       calculate for each quantity of interest.  The default values
       are {\tt c("mean", "sd", "min", "max")}.  
   \end{itemize}
Using {\tt summary()} on {\tt sim()} output from multiple analyses
will automatically weight the quantities of interest according to the
proportion of observations in each strata.  

\subsubsection{Output Values}
\begin{itemize}
\item For {\tt zelig()} output objects, {\tt summary()} returns a
  formatted summary of the regression output, including the usual
  table of coefficients, standard errors, and $t$-values.  Additional
  output depends on the statistical model, the names of which may
  be viewed using the {\tt names(summary(z.out))} command.
\item For {\tt sim()} output objects, {\tt summary()} returns summary
  statistics for each quantity of interest, including a default 95\%
  confidence interval.  Use {\tt names(summary(s.out))} to see the
  available output.  Note that {\tt qi.stats} contains a useful matrix
  of summary statistics for each of the quantities of interest.  For
  example, {\tt summary(s.out)\$qi.stats\$ev}, extracts the summary
  matrix of expected values.
\end{itemize}

\subsubsection{See Also}

Advanced users may wish to refer to {\tt help.zelig("print")},
as well as {\tt help(summary)}, {\tt help(summary.lm)}, {\tt
  help(summary.glm)}, and {\tt help(anova)} in the base package.

\subsubsection{Contributors}

Kosuke Imai, Gary King, and Olivia Lau added summary methods for
certain {\tt zelig()} models, and for {\tt sim()} output.


%%% Local Variables: 
%%% mode: latex
%%% TeX-master: t
%%% End: 


\include{commandsRd/plot.zelig}
\section{{\tt print}: Printing Quantities of Interest}
\label{ss:print}

\subsubsection{Description}
The {\tt print} command formats summary output and allows users to
specify the number of decimal places as an optional argument.  

\subsubsection{Syntax}
\begin{verbatim}
> print(x, digits = 3, print.x = FALSE)
\end{verbatim}

\subsubsection{Arguments}

\begin{itemize}
\item {\tt x}: the object to be printed may be {\tt z.out} output
  from {\tt zelig()}, {\tt x.out} output from {\tt setx()}, {\tt s.out}
  output from {\tt sim()}, or other R data structures.
\item {\tt digits}: the minimum number of significant digits to return
  for all elements of $x < 0$.  By default, {\tt print()} avoids
  scientific notation, but setting the number of digits to 1 will
  frequently force output in scientific notation.  The number of {\tt
    digits} is not the number of significant digits for all output
  values, but the minimum number of significant digits for the
  smallest value in {\tt x} between -1 and 1; this governs the number
  of significant digits in the rest of the values with decimal output.
\item {\tt print.x}: a logical value for {\tt sim()} output, which
  specifies whether to print a summary ({\tt print.x = FALSE}, the
  default) of the {\tt x} and {\tt x1} \emph{inputs} to {\tt sim()},
  or the complete set of inputs (optionally, {\tt print.x = TRUE}).
\end{itemize}

\subsubsection{Examples}
\begin{verbatim}
> print(summary(z.out), digits = 2)
> print(summary(s.out), digits = 3, print.x = TRUE)
\end{verbatim}

\subsubsection{See Also}

Advanced users may wish to refer to {\tt help(print)}.

\subsubsection{Contributors}

Kosuke Imai, Gary King, and Olivia Lau added print methods for {\tt
  sim()} output, and {\tt summary()} output for Zelig objects.   

%%% Local Variables: 
%%% mode: latex
%%% TeX-master: t
%%% End: 

\include{commandsRd/repl}

\chapter{Supplementary Commands}

\include{commands/matchit}
% author: Matt Owen
% documentation for using the 'mi' class
\documentclass[a4paper,11pt]{article}

% Packages
\usepackage{listings}
\lstset{language=R}


\begin{document}


\section*{Fitting Multiple Statistical Models Simultaneously}

It is often important, especially when computing data based on counterfactuals, to have a side-by-side comparison of the different fitted models and their \emph{quantities of interest}.  Typically, the statistician interested in comparing multiple models would:
\begin{itemize}
	\item{Prepare Several Different Datasets, by:}
	\begin{itemize}
		\item{Splitting a Dataset into Several Categorized by Some Particular Quality}
		\item{Preparing Counterfactual Data Based on a Pre-existing Dataset}
		\item{Collecting Several Different Data-sets Analyzing Similar Data}
	\end{itemize}
	\item{Compute a Fitted Model for Each Dataset}
	\item{Simulate Quantities of Interest for Each Fitted Model}
	\item{Organize the Results in a Logical Fashion}
\end{itemize}

This process can be quite arduous, and the programming syntax associated with these sorts of processes is often unintuitive and cumbersome.  Resulting from the necessity to simplify this process, the Zelig software package offers an intuitive and simple syntax to produce these results.

\section{The \emph{mi} class: An Easy Method for Multiply Imputed Data}

Zelig offers a simple way to fit models for multiply imputed data.  By submitting an \emph{mi} object to the data parameter of Zelig, the statistician can prepare analysis of multiply imputed data quickly and easily.

\subsection{Parameters and Return-values of the ``mi'' Function}


% Describe arguments/return-values
\begin{description}
	\item[@...:]{A series of \emph{data.frame} objects.  Note: if a non-\emph{data.frame} object is submitted, the \emph{mi} class will raise an error.}
	\item[Return Value:]{An \emph{mi} object that the \emph{zelig} function will use to run analysis on the datasets.}
\end{description}


% clear page, and begin the section on mi
\pagebreak
\subsection{Examples of Multiply Imputed Data}


% CODE EXAMPLE
% ---- -------
\begin{lstlisting}
library(Zelig)

data(turnout)

# split the turnout data into 2 sets
turnout1 <- turnout[1:1000,]
turnout2 <- turnout[1001:2000, ]

# create the mi object
split.turnout <- mi(turnout1, turnout2)

# fit the models
z.out <- zelig(vote ~ race + educate, model="logit", data=split.turnout)

# set a counterfactual on both data-sets
x.out <- setx(z.out, age=65)

# simulate quantities of interest for both models
s.out <- sim(z.out,  x = x.out)

# output results
summary(s.out)
\end{lstlisting}

% some spacing between sections
\vspace{4mm}

\subsection*{Important Notes}
\begin{itemize}
	\item{When \emph{data} is submitted to the zelig function via the \emph{mi} class, Zelig returns an object of both class \emph{MI} and \emph{zelig}.  This is in contrast to usually simply being of the class \emph{zelig}.}
	\item{...}
\end{itemize}

%
\subsection{Developing Models to Interact with \emph{mi} Objects}
For the most part, the Zelig developer should not need any additional code to have their model work with the \emph{mi} class.  However, an API exists for the managing and printing of results from this class.


\section{The \emph{by} Keyword: Subsetting Datasets by Factors}

For some models, a the significance of a factor variable - a variable whose range is a finite, specific number of values - is best understood by subsetting a dataset by each of its potential values.  For example, when trying to analyze voting patterns, it may make sense to fit your model separately by race, religion, or birth-state.  Typically, this would require creating a new data-set for each of the possible combinations of race, religion, and birth-state.  Zelig, via the emph{by} keyword, employs a straightforward method to create such models.

\subsection{How-to Use the \emph{by} Keyword}

\begin{itemize}
	\item{\emph{by} must be a column name of the dataset (or datasets) being passed to the model}
	\item{The formula must not contain any of the values passed into \emph{by}}
	\item{\emph{by} should be a column describing factors - not numerical data.  Otherwise, results may be too numerous or nonsensical}
\end{itemize}

\subsection{Code Example}

\begin{lstlisting}
library(Zelig)

data(turnout)

z.out <- zelig(
	vote ~ age + income + educate,
	model = "logit",
	data  = turnout,
	by    = "race"
	)

x.out <- setx(z.out)

s.out <- sim(z.out, x.out)

summary(s.out)
\end{lstlisting}

% some spacing between sections
\vspace{4mm}

\subsection{Explanation of the Above Code}

The above code individually fits, analyzes, and simulates a model for voter turnout based on age, income, and race.  By passing the \emph{by} keyword the value ``race'', our model knows to split the turnout data object into separate components: one for each different in the data column labeled ``race''.  The beauty in this script is the simplicity in which it does this process.  That is, it differs in no way from a typical call to \emph{zelig} with the exception that the \emph{by} keyword is specified.

\section{Using the \emph{mi} Class and the \emph{by} Keyword Together}

No special programming is necessary to use both the \emph{mi} class along with the \emph{by} keyword!  Just simply pass an \emph{mi} object to data, and pass the names of columns to subset the data by to the \emph{by} parameter, and \emph{zelig} will handle the rest.  In fact, the printout that Zelig will produce will automatically categorize the data according to which dataset was submitted and which factor was subsetted.  For complex operations an API is available for the savvy developer.

\end{document}






























\include{commandsRd/network}
\include{commandsRd/plot.ci}
\include{commandsRd/plot.surv}
\include{commandsRd/rocplot}
\include{commandsRd/ternaryplot}
\include{commandsRd/ternarypoints}

\chapter{Models Zelig Can Run}\label{s:model.details}

This section describes the mathematical components of the models
supported by Zelig, using whenever possible the classification and
notation of \cite{King89}.  Most models have a \emph{stochastic
  component} (probability density given certain parameters) and
a \emph{systematic component} (deterministic functional form that
specifies how one or more of the parameters varies over the observed
values $y_i$ as a function of the explanatory variables $x_i$).

Let $Y_i$ be a random outcome variable, realized as $i = 1, \dots, n$
observations $y_i$.  For the probability density $f(\cdot)$ with
systematic feature $\theta_i$ varying over $i$ and a scalar ancillary
parameter $\alpha$ (constant over $i$), the stochastic component is
given by
\begin{equation*}
Y_i \sim f(y_i \mid \theta_i, \alpha).
\end{equation*}

For a functional form $g(\cdot)$, $k$ explanatory variables $X_i$, and
effect parameters $\beta$, the systematic component is:  
\begin{equation*}
\theta_i = g(x_i, \beta).
\end{equation*}

Using the definitions of \hlink{King, Tomz, and Wittenberg,
    2000}{http://gking.harvard.edu/files/abs/making-abs.shtml},
  \nocite{KinTomWit00}
Zelig generates at least two quantities of interest:   
\begin{itemize}
\item The predicted value is a random draw from the stochastic component
  given random draws of $\beta$ and $\alpha$ from their sampling (or
  posterior) distribution.  
\item The expected value is the \emph{mean} of the stochastic component
  given random draws of $\beta$ and $\alpha$ from their sampling (or
  posterior) distributions.  For computational efficiency, Zelig
  deterministically calculates the expected values from the simulated
  parameters whenever possible. 
\end{itemize}

Both the predicted values and expected values produced by Zelig can be
displayed as histograms or density estimates (to summarize the full
sampling or posterior density), or summarized with confidence
intervals (by sorting the simulations and taking the 5th and 95th
percentile values for a 90\% confidence interval for example),
standard errors (by taking the standard deviation of the simulations),
or point estimates (by averaging the simulations).  The point estimate
of predicted and expected values are the same only in linear models.
In almost all situations, simulations from predicted values have more
variance than expected values.  As the number of simulations increases
the distribution of the expected values tends toward a constant; the
distribution of the predicted values does not collapse as the number
of simulations increases.

\include{aov}
\include{arima}
\include{blogit}
\include{bprobit}
\include{chopit}
\include{cloglog.net}
\include{coxph}
\include{ei.dynamic}
\include{ei.hier}
\include{ei.RxC}
\include{exp}
\include{factor.bayes}
\include{factor.mix}
\include{factor.ord}
\include{gamma}
\include{gamma.gee}
\include{gamma.mixed}
\include{gamma.net}
\include{gamma.survey}
\include{irt1d}
\include{irtkd}
\include{logit}
\include{logit.bayes}
\include{logit.gam}
\include{logit.gee}
\include{logit.gee}
\include{logit.mixed}
\include{logit.mixed}
\include{logit.net}
\include{logit.survey}
\include{lognorm}
\documentclass{article}

\title{
  ls: Least Squares Regression for Continuous
  Dependent Variables
}
\author{Matt Owen, Olivia Lau, Kosuke Imai, and Gary King}




\usepackage{bibentry}
\usepackage{graphicx}
\usepackage{natbib}
\usepackage{amsmath}
\usepackage{url}
\usepackage{Zelig}
\usepackage{Sweave}

%\VignetteIndexEntry{Least Squares Regression for Continuous Dependent Variables}
%\VignetteDepends{Zelig, stats}
%\VignetteKeyWords{model,least squares,continuous, regression}
%\VignettePackage{Zelig}

\begin{document}

\nobibliography*



\section{{\tt ls}: Least Squares Regression for Continuous
Dependent Variables}
\label{ls}

Use least squares regression analysis to estimate the best linear
predictor for the specified dependent variables.

\subsubsection{Syntax}

\begin{verbatim}
> z.out <- zelig(Y ~ X1 + X2, model = "ls", data = mydata)
> x.out <- setx(z.out)
> s.out <- sim(z.out, x = x.out)
\end{verbatim}

\subsubsection{Additional Inputs}  

In addition to the standard inputs, {\tt zelig()} takes the following
additional options for least squares regression:  
\begin{itemize}
\item {\tt robust}: defaults to {\tt FALSE}.  If {\tt TRUE} is
selected, {\tt zelig()} computes robust standard errors based on
sandwich estimators (see \cite{Zeileis04}, \cite{Huber81}, and
\cite{White80}).  The default type of robust standard error is
heteroskedastic consistent (HC), \emph{not} heteroskedastic and
autocorrelation consistent (HAC).  

In addition, {\tt robust} may be a list with the following options:  
\begin{itemize}
\item {\tt method}:  choose from 
\begin{itemize}
\item {\tt "vcovHC"}: (the default if {\tt robust = TRUE}), HC standard errors.
\item {\tt "vcovHAC"}: HAC standard errors without weights.  
\item {\tt "kernHAC"}: HAC standard errors using the weights given in
\cite{Andrews91}.   
\item {\tt "weave"}: HAC standard errors using the weights given in
\cite{LumHea99}.
\end{itemize} 
\item {\tt order.by}: only applies to the HAC methods above.  Defaults to
{\tt NULL} (the observations are chronologically ordered as in the
original data).  Optionally, you may specify a time index (either as
{\tt order.by = z}, where {\tt z} exists outside the data frame; or
as {\tt order.by = \~{}z}, where {\tt z} is a variable in the data
frame).  The observations are chronologically ordered by the size of
{\tt z}.
\item {\tt \dots}:  additional options passed to the functions
specified in {\tt method}.  See the {\tt sandwich} library and
\cite{Zeileis04} for more options.   
\end{itemize}
\end{itemize}

\subsubsection{Examples}\begin{enumerate}
\item Basic Example with First Differences

Attach sample data:
\begin{Schunk}
\begin{Sinput}
>  data(macro)
\end{Sinput}
\end{Schunk}
Estimate model:
\begin{Schunk}
\begin{Sinput}
>  z.out1 <- zelig(unem ~ gdp + capmob + trade, model = "ls", data = macro)
\end{Sinput}
\end{Schunk}
Summarize regression coefficients:
\begin{Schunk}
\begin{Sinput}
>  summary(z.out1)
\end{Sinput}
\end{Schunk}
Set explanatory variables to their default (mean/mode) values, with
high (80th percentile) and low (20th percentile) values for the trade variable:
\begin{Schunk}
\begin{Sinput}
>  x.high <- setx(z.out1, trade = quantile(macro$trade, 0.8))
\documentclass{article}

\title{
  ls: Least Squares Regression for Continuous
  Dependent Variables
}
\author{Matt Owen, Olivia Lau, Kosuke Imai, and Gary King}




\usepackage{bibentry}
\usepackage{graphicx}
\usepackage{natbib}
\usepackage{amsmath}
\usepackage{url}
\usepackage{Zelig}
\usepackage{Sweave}

%\VignetteIndexEntry{Least Squares Regression for Continuous Dependent Variables}
%\VignetteDepends{Zelig, stats}
%\VignetteKeyWords{model,least squares,continuous, regression}
%\VignettePackage{Zelig}

\begin{document}

\nobibliography*



\section{{\tt ls}: Least Squares Regression for Continuous
Dependent Variables}
\label{ls}

Use least squares regression analysis to estimate the best linear
predictor for the specified dependent variables.

\subsubsection{Syntax}

\begin{verbatim}
> z.out <- zelig(Y ~ X1 + X2, model = "ls", data = mydata)
> x.out <- setx(z.out)
> s.out <- sim(z.out, x = x.out)
\end{verbatim}

\subsubsection{Additional Inputs}  

In addition to the standard inputs, {\tt zelig()} takes the following
additional options for least squares regression:  
\begin{itemize}
\item {\tt robust}: defaults to {\tt FALSE}.  If {\tt TRUE} is
selected, {\tt zelig()} computes robust standard errors based on
sandwich estimators (see \cite{Zeileis04}, \cite{Huber81}, and
\cite{White80}).  The default type of robust standard error is
heteroskedastic consistent (HC), \emph{not} heteroskedastic and
autocorrelation consistent (HAC).  

In addition, {\tt robust} may be a list with the following options:  
\begin{itemize}
\item {\tt method}:  choose from 
\begin{itemize}
\item {\tt "vcovHC"}: (the default if {\tt robust = TRUE}), HC standard errors.
\item {\tt "vcovHAC"}: HAC standard errors without weights.  
\item {\tt "kernHAC"}: HAC standard errors using the weights given in
\cite{Andrews91}.   
\item {\tt "weave"}: HAC standard errors using the weights given in
\cite{LumHea99}.
\end{itemize} 
\item {\tt order.by}: only applies to the HAC methods above.  Defaults to
{\tt NULL} (the observations are chronologically ordered as in the
original data).  Optionally, you may specify a time index (either as
{\tt order.by = z}, where {\tt z} exists outside the data frame; or
as {\tt order.by = \~{}z}, where {\tt z} is a variable in the data
frame).  The observations are chronologically ordered by the size of
{\tt z}.
\item {\tt \dots}:  additional options passed to the functions
specified in {\tt method}.  See the {\tt sandwich} library and
\cite{Zeileis04} for more options.   
\end{itemize}
\end{itemize}

\subsubsection{Examples}\begin{enumerate}
\item Basic Example with First Differences

Attach sample data:
\begin{Schunk}
\begin{Sinput}
>  data(macro)
\end{Sinput}
\end{Schunk}
Estimate model:
\begin{Schunk}
\begin{Sinput}
>  z.out1 <- zelig(unem ~ gdp + capmob + trade, model = "ls", data = macro)
\end{Sinput}
\end{Schunk}
Summarize regression coefficients:
\begin{Schunk}
\begin{Sinput}
>  summary(z.out1)
\end{Sinput}
\end{Schunk}
Set explanatory variables to their default (mean/mode) values, with
high (80th percentile) and low (20th percentile) values for the trade variable:
\begin{Schunk}
\begin{Sinput}
>  x.high <- setx(z.out1, trade = quantile(macro$trade, 0.8))
\documentclass{article}

\title{
  ls: Least Squares Regression for Continuous
  Dependent Variables
}
\author{Matt Owen, Olivia Lau, Kosuke Imai, and Gary King}




\usepackage{bibentry}
\usepackage{graphicx}
\usepackage{natbib}
\usepackage{amsmath}
\usepackage{url}
\usepackage{Zelig}
\usepackage{Sweave}

%\VignetteIndexEntry{Least Squares Regression for Continuous Dependent Variables}
%\VignetteDepends{Zelig, stats}
%\VignetteKeyWords{model,least squares,continuous, regression}
%\VignettePackage{Zelig}

\begin{document}

\nobibliography*



\section{{\tt ls}: Least Squares Regression for Continuous
Dependent Variables}
\label{ls}

Use least squares regression analysis to estimate the best linear
predictor for the specified dependent variables.

\subsubsection{Syntax}

\begin{verbatim}
> z.out <- zelig(Y ~ X1 + X2, model = "ls", data = mydata)
> x.out <- setx(z.out)
> s.out <- sim(z.out, x = x.out)
\end{verbatim}

\subsubsection{Additional Inputs}  

In addition to the standard inputs, {\tt zelig()} takes the following
additional options for least squares regression:  
\begin{itemize}
\item {\tt robust}: defaults to {\tt FALSE}.  If {\tt TRUE} is
selected, {\tt zelig()} computes robust standard errors based on
sandwich estimators (see \cite{Zeileis04}, \cite{Huber81}, and
\cite{White80}).  The default type of robust standard error is
heteroskedastic consistent (HC), \emph{not} heteroskedastic and
autocorrelation consistent (HAC).  

In addition, {\tt robust} may be a list with the following options:  
\begin{itemize}
\item {\tt method}:  choose from 
\begin{itemize}
\item {\tt "vcovHC"}: (the default if {\tt robust = TRUE}), HC standard errors.
\item {\tt "vcovHAC"}: HAC standard errors without weights.  
\item {\tt "kernHAC"}: HAC standard errors using the weights given in
\cite{Andrews91}.   
\item {\tt "weave"}: HAC standard errors using the weights given in
\cite{LumHea99}.
\end{itemize} 
\item {\tt order.by}: only applies to the HAC methods above.  Defaults to
{\tt NULL} (the observations are chronologically ordered as in the
original data).  Optionally, you may specify a time index (either as
{\tt order.by = z}, where {\tt z} exists outside the data frame; or
as {\tt order.by = \~{}z}, where {\tt z} is a variable in the data
frame).  The observations are chronologically ordered by the size of
{\tt z}.
\item {\tt \dots}:  additional options passed to the functions
specified in {\tt method}.  See the {\tt sandwich} library and
\cite{Zeileis04} for more options.   
\end{itemize}
\end{itemize}

\subsubsection{Examples}\begin{enumerate}
\item Basic Example with First Differences

Attach sample data:
\begin{Schunk}
\begin{Sinput}
>  data(macro)
\end{Sinput}
\end{Schunk}
Estimate model:
\begin{Schunk}
\begin{Sinput}
>  z.out1 <- zelig(unem ~ gdp + capmob + trade, model = "ls", data = macro)
\end{Sinput}
\end{Schunk}
Summarize regression coefficients:
\begin{Schunk}
\begin{Sinput}
>  summary(z.out1)
\end{Sinput}
\end{Schunk}
Set explanatory variables to their default (mean/mode) values, with
high (80th percentile) and low (20th percentile) values for the trade variable:
\begin{Schunk}
\begin{Sinput}
>  x.high <- setx(z.out1, trade = quantile(macro$trade, 0.8))
\include{mlogit}
\include{mlogit.bayes}
\include{negbin}
\include{normal}
\include{normal.bayes}
\include{normal.gam}
\include{normal.gee}
\include{normal.net}
\include{normal.survey}
\include{ologit}
\include{oprobit}
\include{oprobit.bayes}
\include{poisson}
\include{poisson.bayes}
\include{poisson.gam}
\include{poisson.gee}
\include{poisson.mixed}
\include{poisson.net}
\include{poisson.survey}
\documentclass{article}

\title{
  probit: Probit Regression for Dichotomous
  Dependent Variables
}
\author{Matt Owen, Olivia Lau, Kosuke Imai, and Gary King}




\usepackage{bibentry}
\usepackage{graphicx}
\usepackage{natbib}
\usepackage{amsmath}
\usepackage{url}
\usepackage{Zelig}
\usepackage{Sweave}

%\VignetteIndexEntry{Probit Regression for Dichotomous Dependent Variables}
%\VignetteDepends{Zelig, stats}
%\VignetteKeyWords{model,least squares,continuous, regression}
%\VignettePackage{Zelig}

\begin{document}

\nobibliography*



\section{{\tt probit}: Probit Regression for Dichotomous Dependent Variables}\label{probit}

Use probit regression to model binary dependent variables
specified as a function of a set of explanatory variables.  

\subsubsection{Syntax}
\begin{verbatim}
> z.out <- zelig(Y ~ X1 + X2, model = "probit", data = mydata)
> x.out <- setx(z.out)
> s.out <- sim(z.out, x = x.out, x1 = NULL)
\end{verbatim}

\subsubsection{Additional Inputs} 

In addition to the standard inputs, {\tt zelig()} takes the following
additional options for probit regression:  
\begin{itemize}
\item {\tt robust}: defaults to {\tt FALSE}.  If {\tt TRUE} is
selected, {\tt zelig()} computes robust standard errors via the {\tt
sandwich} package (see \cite{Zeileis04}).  The default type of robust
standard error is heteroskedastic and autocorrelation consistent (HAC),
and assumes that observations are ordered by time index.

In addition, {\tt robust} may be a list with the following options:  
\begin{itemize}
\item {\tt method}:  Choose from 
\begin{itemize}
\item {\tt "vcovHAC"}: (default if {\tt robust = TRUE}) HAC standard
errors. 
\item {\tt "kernHAC"}: HAC standard errors using the
weights given in \cite{Andrews91}. 
\item {\tt "weave"}: HAC standard errors using the
weights given in \cite{LumHea99}.  
\end{itemize}  
\item {\tt order.by}: defaults to {\tt NULL} (the observations are
chronologically ordered as in the original data).  Optionally, you may
specify a vector of weights (either as {\tt order.by = z}, where {\tt
z} exists outside the data frame; or as {\tt order.by = \~{}z}, where
{\tt z} is a variable in the data frame).  The observations are
chronologically ordered by the size of {\tt z}.
\item {\tt \dots}:  additional options passed to the functions 
specified in {\tt method}.   See the {\tt sandwich} library and
\cite{Zeileis04} for more options.   
\end{itemize}
\end{itemize}

\subsubsection{Examples}
Attach the sample turnout dataset:
\begin{Schunk}
\begin{Sinput}
>  data(turnout)
\end{Sinput}
\end{Schunk}
Estimate parameter values for the probit regression:
\begin{Schunk}
\begin{Sinput}
>  z.out <- zelig(vote ~ race + educate,  model = "probit", data = turnout) 
\end{Sinput}
\end{Schunk}
\begin{Schunk}
\begin{Sinput}
>  summary(z.out)
\end{Sinput}
\end{Schunk}
Set values for the explanatory variables to their default values.
\begin{Schunk}
\begin{Sinput}
>  x.out <- setx(z.out)
\documentclass{article}

\title{
  probit: Probit Regression for Dichotomous
  Dependent Variables
}
\author{Matt Owen, Olivia Lau, Kosuke Imai, and Gary King}




\usepackage{bibentry}
\usepackage{graphicx}
\usepackage{natbib}
\usepackage{amsmath}
\usepackage{url}
\usepackage{Zelig}
\usepackage{Sweave}

%\VignetteIndexEntry{Probit Regression for Dichotomous Dependent Variables}
%\VignetteDepends{Zelig, stats}
%\VignetteKeyWords{model,least squares,continuous, regression}
%\VignettePackage{Zelig}

\begin{document}

\nobibliography*



\section{{\tt probit}: Probit Regression for Dichotomous Dependent Variables}\label{probit}

Use probit regression to model binary dependent variables
specified as a function of a set of explanatory variables.  

\subsubsection{Syntax}
\begin{verbatim}
> z.out <- zelig(Y ~ X1 + X2, model = "probit", data = mydata)
> x.out <- setx(z.out)
> s.out <- sim(z.out, x = x.out, x1 = NULL)
\end{verbatim}

\subsubsection{Additional Inputs} 

In addition to the standard inputs, {\tt zelig()} takes the following
additional options for probit regression:  
\begin{itemize}
\item {\tt robust}: defaults to {\tt FALSE}.  If {\tt TRUE} is
selected, {\tt zelig()} computes robust standard errors via the {\tt
sandwich} package (see \cite{Zeileis04}).  The default type of robust
standard error is heteroskedastic and autocorrelation consistent (HAC),
and assumes that observations are ordered by time index.

In addition, {\tt robust} may be a list with the following options:  
\begin{itemize}
\item {\tt method}:  Choose from 
\begin{itemize}
\item {\tt "vcovHAC"}: (default if {\tt robust = TRUE}) HAC standard
errors. 
\item {\tt "kernHAC"}: HAC standard errors using the
weights given in \cite{Andrews91}. 
\item {\tt "weave"}: HAC standard errors using the
weights given in \cite{LumHea99}.  
\end{itemize}  
\item {\tt order.by}: defaults to {\tt NULL} (the observations are
chronologically ordered as in the original data).  Optionally, you may
specify a vector of weights (either as {\tt order.by = z}, where {\tt
z} exists outside the data frame; or as {\tt order.by = \~{}z}, where
{\tt z} is a variable in the data frame).  The observations are
chronologically ordered by the size of {\tt z}.
\item {\tt \dots}:  additional options passed to the functions 
specified in {\tt method}.   See the {\tt sandwich} library and
\cite{Zeileis04} for more options.   
\end{itemize}
\end{itemize}

\subsubsection{Examples}
Attach the sample turnout dataset:
\begin{Schunk}
\begin{Sinput}
>  data(turnout)
\end{Sinput}
\end{Schunk}
Estimate parameter values for the probit regression:
\begin{Schunk}
\begin{Sinput}
>  z.out <- zelig(vote ~ race + educate,  model = "probit", data = turnout) 
\end{Sinput}
\end{Schunk}
\begin{Schunk}
\begin{Sinput}
>  summary(z.out)
\end{Sinput}
\end{Schunk}
Set values for the explanatory variables to their default values.
\begin{Schunk}
\begin{Sinput}
>  x.out <- setx(z.out)
\documentclass{article}

\title{
  probit: Probit Regression for Dichotomous
  Dependent Variables
}
\author{Matt Owen, Olivia Lau, Kosuke Imai, and Gary King}




\usepackage{bibentry}
\usepackage{graphicx}
\usepackage{natbib}
\usepackage{amsmath}
\usepackage{url}
\usepackage{Zelig}
\usepackage{Sweave}

%\VignetteIndexEntry{Probit Regression for Dichotomous Dependent Variables}
%\VignetteDepends{Zelig, stats}
%\VignetteKeyWords{model,least squares,continuous, regression}
%\VignettePackage{Zelig}

\begin{document}

\nobibliography*



\section{{\tt probit}: Probit Regression for Dichotomous Dependent Variables}\label{probit}

Use probit regression to model binary dependent variables
specified as a function of a set of explanatory variables.  

\subsubsection{Syntax}
\begin{verbatim}
> z.out <- zelig(Y ~ X1 + X2, model = "probit", data = mydata)
> x.out <- setx(z.out)
> s.out <- sim(z.out, x = x.out, x1 = NULL)
\end{verbatim}

\subsubsection{Additional Inputs} 

In addition to the standard inputs, {\tt zelig()} takes the following
additional options for probit regression:  
\begin{itemize}
\item {\tt robust}: defaults to {\tt FALSE}.  If {\tt TRUE} is
selected, {\tt zelig()} computes robust standard errors via the {\tt
sandwich} package (see \cite{Zeileis04}).  The default type of robust
standard error is heteroskedastic and autocorrelation consistent (HAC),
and assumes that observations are ordered by time index.

In addition, {\tt robust} may be a list with the following options:  
\begin{itemize}
\item {\tt method}:  Choose from 
\begin{itemize}
\item {\tt "vcovHAC"}: (default if {\tt robust = TRUE}) HAC standard
errors. 
\item {\tt "kernHAC"}: HAC standard errors using the
weights given in \cite{Andrews91}. 
\item {\tt "weave"}: HAC standard errors using the
weights given in \cite{LumHea99}.  
\end{itemize}  
\item {\tt order.by}: defaults to {\tt NULL} (the observations are
chronologically ordered as in the original data).  Optionally, you may
specify a vector of weights (either as {\tt order.by = z}, where {\tt
z} exists outside the data frame; or as {\tt order.by = \~{}z}, where
{\tt z} is a variable in the data frame).  The observations are
chronologically ordered by the size of {\tt z}.
\item {\tt \dots}:  additional options passed to the functions 
specified in {\tt method}.   See the {\tt sandwich} library and
\cite{Zeileis04} for more options.   
\end{itemize}
\end{itemize}

\subsubsection{Examples}
Attach the sample turnout dataset:
\begin{Schunk}
\begin{Sinput}
>  data(turnout)
\end{Sinput}
\end{Schunk}
Estimate parameter values for the probit regression:
\begin{Schunk}
\begin{Sinput}
>  z.out <- zelig(vote ~ race + educate,  model = "probit", data = turnout) 
\end{Sinput}
\end{Schunk}
\begin{Schunk}
\begin{Sinput}
>  summary(z.out)
\end{Sinput}
\end{Schunk}
Set values for the explanatory variables to their default values.
\begin{Schunk}
\begin{Sinput}
>  x.out <- setx(z.out)
\documentclass{article}

\title{
  probit: Probit Regression for Dichotomous
  Dependent Variables
}
\author{Matt Owen, Olivia Lau, Kosuke Imai, and Gary King}




\usepackage{bibentry}
\usepackage{graphicx}
\usepackage{natbib}
\usepackage{amsmath}
\usepackage{url}
\usepackage{Zelig}
\usepackage{Sweave}

%\VignetteIndexEntry{Probit Regression for Dichotomous Dependent Variables}
%\VignetteDepends{Zelig, stats}
%\VignetteKeyWords{model,least squares,continuous, regression}
%\VignettePackage{Zelig}

\begin{document}

\nobibliography*



\section{{\tt probit}: Probit Regression for Dichotomous Dependent Variables}\label{probit}

Use probit regression to model binary dependent variables
specified as a function of a set of explanatory variables.  

\subsubsection{Syntax}
\begin{verbatim}
> z.out <- zelig(Y ~ X1 + X2, model = "probit", data = mydata)
> x.out <- setx(z.out)
> s.out <- sim(z.out, x = x.out, x1 = NULL)
\end{verbatim}

\subsubsection{Additional Inputs} 

In addition to the standard inputs, {\tt zelig()} takes the following
additional options for probit regression:  
\begin{itemize}
\item {\tt robust}: defaults to {\tt FALSE}.  If {\tt TRUE} is
selected, {\tt zelig()} computes robust standard errors via the {\tt
sandwich} package (see \cite{Zeileis04}).  The default type of robust
standard error is heteroskedastic and autocorrelation consistent (HAC),
and assumes that observations are ordered by time index.

In addition, {\tt robust} may be a list with the following options:  
\begin{itemize}
\item {\tt method}:  Choose from 
\begin{itemize}
\item {\tt "vcovHAC"}: (default if {\tt robust = TRUE}) HAC standard
errors. 
\item {\tt "kernHAC"}: HAC standard errors using the
weights given in \cite{Andrews91}. 
\item {\tt "weave"}: HAC standard errors using the
weights given in \cite{LumHea99}.  
\end{itemize}  
\item {\tt order.by}: defaults to {\tt NULL} (the observations are
chronologically ordered as in the original data).  Optionally, you may
specify a vector of weights (either as {\tt order.by = z}, where {\tt
z} exists outside the data frame; or as {\tt order.by = \~{}z}, where
{\tt z} is a variable in the data frame).  The observations are
chronologically ordered by the size of {\tt z}.
\item {\tt \dots}:  additional options passed to the functions 
specified in {\tt method}.   See the {\tt sandwich} library and
\cite{Zeileis04} for more options.   
\end{itemize}
\end{itemize}

\subsubsection{Examples}
Attach the sample turnout dataset:
\begin{Schunk}
\begin{Sinput}
>  data(turnout)
\end{Sinput}
\end{Schunk}
Estimate parameter values for the probit regression:
\begin{Schunk}
\begin{Sinput}
>  z.out <- zelig(vote ~ race + educate,  model = "probit", data = turnout) 
\end{Sinput}
\end{Schunk}
\begin{Schunk}
\begin{Sinput}
>  summary(z.out)
\end{Sinput}
\end{Schunk}
Set values for the explanatory variables to their default values.
\begin{Schunk}
\begin{Sinput}
>  x.out <- setx(z.out)
\documentclass{article}

\title{
  probit: Probit Regression for Dichotomous
  Dependent Variables
}
\author{Matt Owen, Olivia Lau, Kosuke Imai, and Gary King}




\usepackage{bibentry}
\usepackage{graphicx}
\usepackage{natbib}
\usepackage{amsmath}
\usepackage{url}
\usepackage{Zelig}
\usepackage{Sweave}

%\VignetteIndexEntry{Probit Regression for Dichotomous Dependent Variables}
%\VignetteDepends{Zelig, stats}
%\VignetteKeyWords{model,least squares,continuous, regression}
%\VignettePackage{Zelig}

\begin{document}

\nobibliography*



\section{{\tt probit}: Probit Regression for Dichotomous Dependent Variables}\label{probit}

Use probit regression to model binary dependent variables
specified as a function of a set of explanatory variables.  

\subsubsection{Syntax}
\begin{verbatim}
> z.out <- zelig(Y ~ X1 + X2, model = "probit", data = mydata)
> x.out <- setx(z.out)
> s.out <- sim(z.out, x = x.out, x1 = NULL)
\end{verbatim}

\subsubsection{Additional Inputs} 

In addition to the standard inputs, {\tt zelig()} takes the following
additional options for probit regression:  
\begin{itemize}
\item {\tt robust}: defaults to {\tt FALSE}.  If {\tt TRUE} is
selected, {\tt zelig()} computes robust standard errors via the {\tt
sandwich} package (see \cite{Zeileis04}).  The default type of robust
standard error is heteroskedastic and autocorrelation consistent (HAC),
and assumes that observations are ordered by time index.

In addition, {\tt robust} may be a list with the following options:  
\begin{itemize}
\item {\tt method}:  Choose from 
\begin{itemize}
\item {\tt "vcovHAC"}: (default if {\tt robust = TRUE}) HAC standard
errors. 
\item {\tt "kernHAC"}: HAC standard errors using the
weights given in \cite{Andrews91}. 
\item {\tt "weave"}: HAC standard errors using the
weights given in \cite{LumHea99}.  
\end{itemize}  
\item {\tt order.by}: defaults to {\tt NULL} (the observations are
chronologically ordered as in the original data).  Optionally, you may
specify a vector of weights (either as {\tt order.by = z}, where {\tt
z} exists outside the data frame; or as {\tt order.by = \~{}z}, where
{\tt z} is a variable in the data frame).  The observations are
chronologically ordered by the size of {\tt z}.
\item {\tt \dots}:  additional options passed to the functions 
specified in {\tt method}.   See the {\tt sandwich} library and
\cite{Zeileis04} for more options.   
\end{itemize}
\end{itemize}

\subsubsection{Examples}
Attach the sample turnout dataset:
\begin{Schunk}
\begin{Sinput}
>  data(turnout)
\end{Sinput}
\end{Schunk}
Estimate parameter values for the probit regression:
\begin{Schunk}
\begin{Sinput}
>  z.out <- zelig(vote ~ race + educate,  model = "probit", data = turnout) 
\end{Sinput}
\end{Schunk}
\begin{Schunk}
\begin{Sinput}
>  summary(z.out)
\end{Sinput}
\end{Schunk}
Set values for the explanatory variables to their default values.
\begin{Schunk}
\begin{Sinput}
>  x.out <- setx(z.out)
\documentclass{article}

\title{
  probit: Probit Regression for Dichotomous
  Dependent Variables
}
\author{Matt Owen, Olivia Lau, Kosuke Imai, and Gary King}




\usepackage{bibentry}
\usepackage{graphicx}
\usepackage{natbib}
\usepackage{amsmath}
\usepackage{url}
\usepackage{Zelig}
\usepackage{Sweave}

%\VignetteIndexEntry{Probit Regression for Dichotomous Dependent Variables}
%\VignetteDepends{Zelig, stats}
%\VignetteKeyWords{model,least squares,continuous, regression}
%\VignettePackage{Zelig}

\begin{document}

\nobibliography*



\section{{\tt probit}: Probit Regression for Dichotomous Dependent Variables}\label{probit}

Use probit regression to model binary dependent variables
specified as a function of a set of explanatory variables.  

\subsubsection{Syntax}
\begin{verbatim}
> z.out <- zelig(Y ~ X1 + X2, model = "probit", data = mydata)
> x.out <- setx(z.out)
> s.out <- sim(z.out, x = x.out, x1 = NULL)
\end{verbatim}

\subsubsection{Additional Inputs} 

In addition to the standard inputs, {\tt zelig()} takes the following
additional options for probit regression:  
\begin{itemize}
\item {\tt robust}: defaults to {\tt FALSE}.  If {\tt TRUE} is
selected, {\tt zelig()} computes robust standard errors via the {\tt
sandwich} package (see \cite{Zeileis04}).  The default type of robust
standard error is heteroskedastic and autocorrelation consistent (HAC),
and assumes that observations are ordered by time index.

In addition, {\tt robust} may be a list with the following options:  
\begin{itemize}
\item {\tt method}:  Choose from 
\begin{itemize}
\item {\tt "vcovHAC"}: (default if {\tt robust = TRUE}) HAC standard
errors. 
\item {\tt "kernHAC"}: HAC standard errors using the
weights given in \cite{Andrews91}. 
\item {\tt "weave"}: HAC standard errors using the
weights given in \cite{LumHea99}.  
\end{itemize}  
\item {\tt order.by}: defaults to {\tt NULL} (the observations are
chronologically ordered as in the original data).  Optionally, you may
specify a vector of weights (either as {\tt order.by = z}, where {\tt
z} exists outside the data frame; or as {\tt order.by = \~{}z}, where
{\tt z} is a variable in the data frame).  The observations are
chronologically ordered by the size of {\tt z}.
\item {\tt \dots}:  additional options passed to the functions 
specified in {\tt method}.   See the {\tt sandwich} library and
\cite{Zeileis04} for more options.   
\end{itemize}
\end{itemize}

\subsubsection{Examples}
Attach the sample turnout dataset:
\begin{Schunk}
\begin{Sinput}
>  data(turnout)
\end{Sinput}
\end{Schunk}
Estimate parameter values for the probit regression:
\begin{Schunk}
\begin{Sinput}
>  z.out <- zelig(vote ~ race + educate,  model = "probit", data = turnout) 
\end{Sinput}
\end{Schunk}
\begin{Schunk}
\begin{Sinput}
>  summary(z.out)
\end{Sinput}
\end{Schunk}
Set values for the explanatory variables to their default values.
\begin{Schunk}
\begin{Sinput}
>  x.out <- setx(z.out)
\documentclass{article}

\title{
  probit: Probit Regression for Dichotomous
  Dependent Variables
}
\author{Matt Owen, Olivia Lau, Kosuke Imai, and Gary King}




\usepackage{bibentry}
\usepackage{graphicx}
\usepackage{natbib}
\usepackage{amsmath}
\usepackage{url}
\usepackage{Zelig}
\usepackage{Sweave}

%\VignetteIndexEntry{Probit Regression for Dichotomous Dependent Variables}
%\VignetteDepends{Zelig, stats}
%\VignetteKeyWords{model,least squares,continuous, regression}
%\VignettePackage{Zelig}

\begin{document}

\nobibliography*



\section{{\tt probit}: Probit Regression for Dichotomous Dependent Variables}\label{probit}

Use probit regression to model binary dependent variables
specified as a function of a set of explanatory variables.  

\subsubsection{Syntax}
\begin{verbatim}
> z.out <- zelig(Y ~ X1 + X2, model = "probit", data = mydata)
> x.out <- setx(z.out)
> s.out <- sim(z.out, x = x.out, x1 = NULL)
\end{verbatim}

\subsubsection{Additional Inputs} 

In addition to the standard inputs, {\tt zelig()} takes the following
additional options for probit regression:  
\begin{itemize}
\item {\tt robust}: defaults to {\tt FALSE}.  If {\tt TRUE} is
selected, {\tt zelig()} computes robust standard errors via the {\tt
sandwich} package (see \cite{Zeileis04}).  The default type of robust
standard error is heteroskedastic and autocorrelation consistent (HAC),
and assumes that observations are ordered by time index.

In addition, {\tt robust} may be a list with the following options:  
\begin{itemize}
\item {\tt method}:  Choose from 
\begin{itemize}
\item {\tt "vcovHAC"}: (default if {\tt robust = TRUE}) HAC standard
errors. 
\item {\tt "kernHAC"}: HAC standard errors using the
weights given in \cite{Andrews91}. 
\item {\tt "weave"}: HAC standard errors using the
weights given in \cite{LumHea99}.  
\end{itemize}  
\item {\tt order.by}: defaults to {\tt NULL} (the observations are
chronologically ordered as in the original data).  Optionally, you may
specify a vector of weights (either as {\tt order.by = z}, where {\tt
z} exists outside the data frame; or as {\tt order.by = \~{}z}, where
{\tt z} is a variable in the data frame).  The observations are
chronologically ordered by the size of {\tt z}.
\item {\tt \dots}:  additional options passed to the functions 
specified in {\tt method}.   See the {\tt sandwich} library and
\cite{Zeileis04} for more options.   
\end{itemize}
\end{itemize}

\subsubsection{Examples}
Attach the sample turnout dataset:
\begin{Schunk}
\begin{Sinput}
>  data(turnout)
\end{Sinput}
\end{Schunk}
Estimate parameter values for the probit regression:
\begin{Schunk}
\begin{Sinput}
>  z.out <- zelig(vote ~ race + educate,  model = "probit", data = turnout) 
\end{Sinput}
\end{Schunk}
\begin{Schunk}
\begin{Sinput}
>  summary(z.out)
\end{Sinput}
\end{Schunk}
Set values for the explanatory variables to their default values.
\begin{Schunk}
\begin{Sinput}
>  x.out <- setx(z.out)
\include{relogit}
\include{sur}
\include{threesls}
\include{tobit}
\include{tobit.bayes}
\include{twosls}

\include{zinput}
%\VignetteIndexEntry{Weibull Regression for Duration Dependent Variables}
%\VignetteDepends{Zelig, survival}
%\VignetteKeyWords{model, weibull,regression,bounded, duration}
%\VignettePackage{Zelig}
\usepackage{Sweave}
\begin{document}
\nobibliography*


\section{{\tt weibull}: Weibull Regression for Duration
Dependent Variables}\label{weibull}

Choose the Weibull regression model if the values in your dependent
variable are duration observations.  The Weibull model relaxes the
exponential model's (see \Sref{exp}) assumption of constant hazard,
and allows the hazard rate to increase or decrease monotonically with
respect to elapsed time.

\subsubsection{Syntax}

\begin{verbatim}
> z.out <- zelig(Surv(Y, C) ~ X1 + X2, model = "weibull", data = mydata)
> x.out <- setx(z.out)
> s.out <- sim(z.out, x = x.out)
\end{verbatim}
Weibull models require that the dependent variable be in the form {\tt
  Surv(Y, C)}, where {\tt Y} and {\tt C} are vectors of length $n$.
For each observation $i$ in 1, \dots, $n$, the value $y_i$ is the
duration (lifetime, for example), and the associated $c_i$ is a binary
variable such that $c_i = 1$ if the duration is not censored ({\it
  e.g.}, the subject dies during the study) or $c_i = 0$ if the
duration is censored ({\it e.g.}, the subject is still alive at the
end of the study).  If $c_i$ is omitted, all Y are assumed to be
completed; that is, time defaults to 1 for all observations.

\subsubsection{Input Values} 

In addition to the standard inputs, {\tt zelig()} takes the following
additional options for weibull regression:  
\begin{itemize}
\item {\tt robust}: defaults to {\tt FALSE}.  If {\tt TRUE}, {\tt
zelig()} computes robust standard errors based on sandwich estimators
(see \cite{Huber81} and \cite{White80}) based on the options in {\tt
cluster}.
\item {\tt cluster}:  if {\tt robust = TRUE}, you may select a
variable to define groups of correlated observations.  Let {\tt x3} be
a variable that consists of either discrete numeric values, character
strings, or factors that define strata.  Then
\begin{verbatim}
> z.out <- zelig(y ~ x1 + x2, robust = TRUE, cluster = "x3", 
                 model = "exp", data = mydata)
\end{verbatim}
means that the observations can be correlated within the strata defined by
the variable {\tt x3}, and that robust standard errors should be
calculated according to those clusters.  If {\tt robust = TRUE} but
{\tt cluster} is not specified, {\tt zelig()} assumes that each
observation falls into its own cluster.  
\end{itemize}  

\subsubsection{Example}

Attach the sample data: 
\begin{Schunk}
\begin{Sinput}
> data(coalition)
\end{Sinput}
\end{Schunk}
Estimate the model: 
\begin{Schunk}
\begin{Sinput}
> z.out <- zelig(Surv(duration, ciep12) ~ fract + numst2, model = "weibull", 
+     data = coalition)
\end{Sinput}
\end{Schunk}
View the regression output:  
\begin{Schunk}
\begin{Sinput}
> summary(z.out)
\end{Sinput}
\end{Schunk}
Set the baseline values (with the ruling coalition in the minority)
and the alternative values (with the ruling coalition in the majority)
for X:
\begin{Schunk}
\begin{Sinput}
> x.low <- setx(z.out, numst2 = 0)
> x.high <- setx(z.out, numst2 = 1)
\end{Sinput}
\end{Schunk}
Simulate expected values ({\tt qi\$ev}) and first differences ({\tt qi\$fd}):
\begin{Schunk}
\begin{Sinput}
> s.out <- sim(z.out, x = x.low, x1 = x.high)


\chapter{Commands for Programmers and Contributors}

\section{{\tt describe}: Describe a model's systematic and stochastic 
parameters}
\label{describe.mymodel}

\subsubsection{Description}

In order to use {\tt parse.formula()}, {\tt parse.par()}, and the {\tt
model.*.multiple()} commands, you must write a {\tt describe.mymodel()}
function where {\tt mymodel} is the name of your modeling function.
(Hence, if your function is called {\tt normal.regression()}, you need
to write a {\tt describe.normal.regression()} function.)  Note that
{\tt describe()} is \emph{not} a generic function, but is called by
{\tt parse.formula(\dots, model = "mymodel")} using a combination of
{\tt paste()} and {\tt exists()}.  You will never need to call {\tt
describe.mymodel()} directly, since it will be called from {\tt
parse.formula()} as that function checks the user-input formula or
list of formulas.  

\subsubsection{Syntax}
\begin{verbatim}
describe.mymodel()
\end{verbatim}

\subsubsection{Arguments}\label{categories}
The {\tt describe.mymodel()} function takes no arguments.  

\subsubsection{Output Values}
The {\tt describe.mymodel()} function returns a list with the
following information:  
\begin{itemize}
\item {\tt category}: a character string, consisting of one of the
following: 
\begin{itemize}
\item {\tt "continuous"}: the dependent variable is continuous, numeric, and
unbounded (e.g., normal regression), but may be censored with an associated censoring 
indicator (e.g., tobit regression).  
\item {\tt "dichotomous"}: the dependent variable takes two discrete integer
values, usually 0 and 1 (e.g., logistic regression).  
\item {\tt "ordinal"}: the dependent variable is an ordered factor
response, taking 3 or more discrete values which are arranged in an
ascending or descending manner (e.g., ordered logistic regression).  
\item {\tt "multinomial"}: the dependent variable is an unordered
factor response, taking 3 or more discrete values which are arranged
in no particular order (e.g., multinomial logistic regression).  
\item {\tt "count"}: the dependent variable takes integer values
greater than or equal to 0 (e.g., Poisson regression).  
\item {\tt "bounded"}: the dependent variable is a continuous numeric variable, that 
is restricted (bounded within) some range (e.g., $(0, \infty)$).  The variable may 
also be censored either on the left and/or right side, with an associated censoring 
indicator (e.g., Weibull regression).
\item {\tt "mixed"}: the dependent variables are a mix of the above
categories (usually applies to multiple equation models).  
\end{itemize}
Selecting the category is particularly important since it sets certain
interface parameters for the entire GUI.

\item {\tt package}: (optional) a list with the following elements

  \begin{itemize} 

   \item {\tt name}: a characters string with the name of the package
   containing the {\tt mymodel()} function.

   \item {\tt version}: the minimum version number that works with
   Zelig.

   \item {\tt CRAN}: if the package is not hosted on CRAN mirrors,
   provide the URL here as a character string.  You should be able
   to install your package from this URL using {\tt name}, {\tt
version}, and {\tt CRAN}:
\begin{verbatim}
install.packages(name, repos = CRAN, installWithVers = TRUE)
\end{verbatim}  
By default, {\tt CRAN = "http://cran.us.r-project.org/"}.  
\end{itemize}

\item {\tt parameters}: For each systematic and stochastic parameter
(or set of parameters) in your model, you should create a list (named
after the parameters as given in your model's notation, e.g., {\tt
mu}, {\tt sigma}, {\tt theta}, etc.; not literally {\tt myparameter})
with the following information:
\begin{itemize}

\item {\tt equations}: an integer number of equations for the
parameter.  For parameters that can take an undefined number of
equations (for example in seemingly unrelated regression), use {\tt
c(2, Inf)} or {\tt c(2, 999)} to indicate that the parameter can take
a minimum of two equations up to a theoretically infinite number of
equations.  

\item {\tt tagsAllowed}: a logical value ({\tt TRUE}/{\tt FALSE})
specifying whether a given parameter allows constraints.  If there is
only one equation for a parameter (for example, {\tt mu} for the
normal regression model has {\tt equations = 1}), then {\tt
tagsAllowed = FALSE} by default.  If there are two or more equations
for the parameter (for example, {\tt mu} for the bivariate probit
model has {\tt equations = 2}), then {\tt tagsAllowed = TRUE} by
default.  

\item {\tt depVar}: a logical value ({\tt TRUE}/{\tt FALSE})
specifying whether a parameter requires a corresponding dependent
variable.  

\item {\tt expVar}: a logical value ({\tt TRUE}/{\tt FALSE})
specifying whether a parameter allows explanatory variables.  If {\tt
depVar = TRUE} and {\tt expVar = TRUE}, we call the parameter a
``systematic component'' and {\tt parse.formula()} will fail if
formula(s) are not specified for this parameter.  If {\tt
depVar = FALSE} and {\tt expVar = TRUE}, the parameter is estimated as
a scalar ancillary parameter, with default formula \verb|~ 1|, if the
user does not specify a formula explicitly.  If {\tt depVar = FALSE}
and {\tt expVar = FALSE}, the parameter can only be estimated as a
scalar ancillary parameter.  

\item {\tt specialFunction}: (optional) a character string giving the
name of a function that appears on the left-hand side of the formula.
Options include {\tt "Surv"}, {\tt "cbind"}, and {\tt "as.factor"}. 

\item {\tt varInSpecial}: (optional) a scalar or vector giving the
number of variables taken by the {\tt specialFunction}.  For example,
{\tt Surv()} takes a minimum of 2 arguments, and a maximum of 4
arguments, which is represented as {\tt c(2, 4)}.   

\end{itemize}
If you have more than one parameter (or set of parameters) in
your model, you will need to produce a {\tt myparameter} list for each
one.  See examples below for details.  
\end{itemize}

\subsubsection{Examples}
For a Normal regression model with mean {\tt mu} and scalar variance
parameter {\tt sigma2}, the minimal {\tt describe.*()} function is as
follows:  
\begin{verbatim}
describe.normal.regression <- function() {
  category <- "continuous"
  mu <- list(equations = 1,              # Systematic component
             tagsAllowed = FALSE, 
             depVar = TRUE, 
             expVar = TRUE)
  sigma2 <- list(equations = 1,          # Scalar ancillary parameter
                 tagsAllowed = FALSE, 
                 depVar = FALSE, 
                 expVar = FALSE)
  pars <- list(mu = mu, sigma2 = sigma2)
  model <- list(category = category, parameters = pars)
}
\end{verbatim}
See \Sref{normal.regression} for full code to execute this model from
scratch in R with Zelig.  

Now consider a bivariate probit model with parameter vector {\tt mu} and
correlation parameter {\tt rho} (which may or may not take explanatory
variables).  Since the bivariate probit function uses the {\tt pmvnorm()} 
function from the mvtnorm library, we list this under {\tt package}.   
\begin{verbatim}
describe.bivariate.probit <- function() {
  category <- "dichotomous"
  package <- list(name = "mvtnorm", 
                  version = "0.7")
  mu <- list(equations = 2,               # Systematic component 
             tagsAllowed = TRUE,          
             depVar = TRUE, 
             expVar = TRUE) 
  rho <- list(equations = 1,              # Optional systematic component
             tagsAllowed = FALSE,         #   Estimated as an ancillary
             depVar = FALSE,              #   parameter by default
             expVar = TRUE) 
  pars <- list(mu = mu, rho = rho)
  list(category = category, package = package, parameters = pars)
}
\end{verbatim}  
See \Sref{bivariate.probit} for the full code to write this model from
scratch in R with Zelig. 

For a multinomial logit model, which takes an undefined number of
equations (corresponding to each level in the response variable):  
\begin{verbatim}
describe.multinomial.logit <- function() { 
  category <- "multinomial"
  mu <- list(equations = c(1, Inf), 
             tagsAllowed = TRUE, 
             depVAR = TRUE, 
             expVar = TRUE, 
             specialFunction <- "as.factor", 
             varInSpecial <- c(1, 1))
  list(category = category, parameters = list(mu = mu))
}
\end{verbatim}
(This example does not have corresponding executable sample code.)

\subsubsection{See Also}
\begin{itemize}
\item \Sref{s:new} for an overview of how the {\tt describe.*()}
function works with {\tt parse.formula()}.  
\item \Sref{parse.formula} for information on {\tt parse.formula()}.
\end{itemize}

\subsubsection{Contributors}

Kosuke Imai, Gary King, Olivia Lau, and Ferdinand Alimadhi.

%%% Local Variables: 
%%% mode: latex
%%% TeX-master: "~/zelig/docs/zelig"
%%% End: 

\include{commands/model.end}
\include{commands/model.frame.multiple}
\include{commands/model.matrix.multiple}
\documentclass{article}

\title{Specification for {\tt parse.formula}}
\author{Matt Owen}

\begin{document}

\maketitle


\section{Introduction}

\section{Types of Formula}

There are three basic ways to specify formulae in Zelig.

\begin{enumerate}

  \item A single formula with a single outcome term and one or more 
    response terms. For example, 
    \begin{itemize}
      \item {\tt y \~{} 1}
      \item {\tt y \~{} x1 + x2 + x3}
    \end{itemize}

  \item A single formula with a multiple outcome terms specified with 
    {\tt cbind} or {\tt list} and one or more response terms. For example,
    \begin{itemize}
      \item {\tt cbind(y1, y2) \~{} x1 + x2}
      \item {\tt list(y1, y2) \~{} x1 + x2}
    \end{itemize}

  \item A list of formulas of the first type - single outcome terms with one or
    more response terms. For example, 
    \begin{itemize}
      \item \begin{verbatim}
list(
     y1 ~ x1,
     y2 ~ x1 + x2
     )
        \end{verbatim}

      \item \begin{verbatim}
list(
     mu1 = y1 ~ x1,
     mu2 = y2 ~ 1
     )
        \end{verbatim}
    \end{itemize}

\end{enumerate}



% How to parse a formula
\section{{\tt parse.formula}}



% How a model matrix should be constructed
\section{{\tt model.matrix}}

This should be given a Zelig-style formula. As output, it will produce a valid
model matrix. Care should be taken that matrices of simulations have the same
column order of this model.matrix. Otherwise, simulations will produce invalid 
results.


\end{document}

\include{commands/parse.par}
\include{commands/put.start}
\include{commands/set.start}
\section{{\tt tag}: Constrain parameter effects across equations}
\label{tag}

\subsubsection{Description}
Use {\tt tag()} to identify parameters and constrain their effects
across equations in multiple-equation models.  
  
\subsubsection{Syntax}
\begin{verbatim}
tag(x, label)
\end{verbatim}

\subsubsection{Arguments}
\begin{itemize}
\item {\tt x}: the variable to be constrained.
\item {\tt label}: the name that the constrained variable takes.  
\end{itemize}

\subsubsection{Output Values}
While there is no specific output from {\tt tag()} itself, {\tt
parse.formula()} uses {\tt tag()} to identify parameter constraints
across equations, when a model takes more than one systematic
component.  

\subsubsection{Examples}

\subsubsection{See Also}
\begin{itemize}
\item \Sref{ui} for an overview of the multiple-equation user-interface.
\item \Sref{parse.formula} for more examples of acceptable uses for
{\tt tag()} in formulas.  
\end{itemize}

\subsubsection{Contributors}

Kosuke Imai, Gary King, Olivia Lau, and Ferdinand Alimadhi.


%%% Local Variables: 
%%% mode: latex
%%% TeX-master: t
%%% End: 















%%% Local Variables: 
%%% mode: latex
%%% TeX-master: "zelig"
%%% End: 


\appendix

\part[Appendices]{Appendices}

%\input{distributions}

\chapter{What's New?  What's Next?}

\section{What's New:  Zelig Release Notes}\label{release.notes}

Releases listed as ``stable releases'' have been tested against prior
versions of Zelig for consistency and accuracy.  Testing distributions
may contain bugs, but are usually replaced by stable releases within a
few days. 

\begin{itemize}
\item{3.4-8} (Jan 1, 2010): Stable release for R 2.10

Fixed problem with survival regressions and robust standard erros (assuming survival >= 2.35-8)\\
Fixed vignette documentation\\

\item {3.4-7} (Oct 23, 2009): Stable release for R 2.9.2

Fixed .net models \\

\item {3.4-6} (May 22, 2009): Stable release for R 2.9

Added quantile regression model (quantile) \\
Changed the digits option (thanks to Matthias Studer) \\
Removed the old mloglm model (Multinomial Log-Linear Regression)\\

\item {3.4-5} (Mar 13, 2009): A bug fixed in {\tt plot.ci()} (thanks to Ken Benoit) \\

\item {3.4-4} (Mar 4, 2009): weights are incorporated into {\tt ologit}, {\tt oprobit}, and {\tt negbin} models (thanks to Casey Klofstad) \\

\item {3.4-2} (Feb 10, 2009): Small fixes in the Rd files as required 
by new check in CRAN \\

\item {3.4-0} (Jan 2, 2009): Bug-fix release for R 2.8.0
Changed the Zelig citation \\
Fixed zelig() signature to ensure that the formals() work properly 
and all arguments remain documented. "save.data" and "cite" were not 
documented (thanks to Micah Altman)\\
Fixed some typos in model family names (thanks to Kevin Condon)\\
Fixed the predicted values in gam.* models \\
Fixed the plot functions in gam.* models \\

\item {3.4-0} (Oct 27, 2008): Stable release for R 2.8.0.
zelig() now takes a "citation" argument. If "citation" is "true" (default)
the model citation is printed in each zelig run \\
Introduced two new elemetns on the describe.mymodel function: authors and year \\
Fixed the problems with lme4 package. Note that there is still a problem with 
simulation step of "gamma.mixed" model. We are still working on that. \\
Fixed the bug with "gam" models (wrong predicted values) \\
Fixed the bug with when zelig model name was provided from a variable (reported from 
Jeroen)

  
\item {\bf 3.3-1} (June 12, 2008): Bug-fix release for R.2.7.0. A bug
  fix for {\rm plot.ci()} so that it works with mixed effects models
  (thanks to Keith Schnakenberg).
  
\item {\bf 3.3-0} (June 03, 2008): Stable release for R.2.7.0. Updated
  {\rm coxph} so that it handles time-varying covariates (contributed
  by Patrick Lam). A new plot function for survival models
  (contributed by John Graves). First version dependencies are as
  follows: \newline
\begin{tabular}{ll}\label{table.compact}
  "MASS"     & "7.2-41" \\
  "nlme"     & "3.1-87" \\
  "survival" & "2.34"    \\
  "coda"     & "0.13-1"  \\
  "sna"      & "1.5"     \\
  "boot"     & "1.2-31"  \\
  "nnet"     & "7.2-41"  \\
  "zoo"      & "1.5-0"   \\
  "sandwich" & "2.1-0"   \\
  "lme4"     & "0.99875-9" \\
  "systemfit" & "1.0-2" \\
  "VGAM"      & "0.7-5" \\
  "MCMCpack"  & "0.8-2" \\
  "mvtnorm"   & "0.8-3" \\
  "gee"       & "4.13-13" \\
  "mgcv"      & "1.3-29" \\
  "anchors"   & "1.9-2" \\
  "survey"    & "3.6-13" 
\end{tabular}
  
\item {\bf 3.2-1} (April 10, 2008): Bug-fix release for R.2.6.0-2.6.2.
  Fixed the {\rm setx()} bug for multiply imputed data sets. (Thanks
  to Steve Shewfelt and Keith Schnakenberg.)

\item {\bf 3.2} (April 3, 2008): Stable release for R 2.6.0-2.6.2.
  Adding models for survey data -- {\tt normal.survey}, {\tt
    logit.survey}, {\tt probit.survey}, {\tt poisson.survey}, {\tt
    gamma.survey}.  First version dependencies are as
  follows:\newline
\begin{tabular}{ll}\label{table.compat}
survey    & 3.6-13 \\
MASS      & 7.2-34 \\
nlme      & 3.1-86 \\
survival  & 2.34 \\
boot      & 1.2-30 \\
nnet      & 7.2-34 \\
zoo       & 1.4-0 \\
sandwich  & 2.0-2 \\
sna       & 1.4 \\
lme4      & 0.99875-9 \\
coda      & 0.12-1 \\
systemfit & 0.8-5 \\
VGAM      & 0.7-4 \\
MCMCpack  & 0.8-2 \\
mvtnorm   & 0.8-1 \\
gee       & 4.13-13 \\
mgcv      & 1.3-29 \\
anchors   & 2.0 
\end{tabular}

\item {\bf 3.1-1} (January 10, 2008): Bug-fix release for R
  2.6.0-2.6.1. Fixed bugs, improved the code and the documentation for
  mixed effects models. Thanks to Gregor Gorjanc. Fixed systemfit models 
  due to some API changes in systemfit package. Added some other models (including *.mixed models) in  plot.ci 

\item {\bf 3.1} (November 30, 2007): Stable release for R 2.6.0-2.6.1.
  Adding many new models such as {\tt aov}, {\tt chopit}, {\tt coxph},
  generalized linear mixed-effects models, and gee models.  Also,
  several bugs are fixed.  First version dependencies are as
  follows:\newline
\begin{tabular}{ll}\label{table.compat}
MASS      & 7.2-34 \\
nlme      & 3.1-86 \\
survival  & 2.34 \\
boot      & 1.2-30 \\
nnet      & 7.2-34 \\
zoo       & 1.4-0 \\
sandwich  & 2.0-2 \\
sna       & 1.4 \\
lme4      & 0.99875-9 \\
coda      & 0.12-1 \\
systemfit & 0.8-5 \\
VGAM      & 0.7-4 \\
MCMCpack  & 0.8-2 \\
mvtnorm   & 0.8-1 \\
gee       & 4.13-13 \\
mgcv      & 1.3-29 \\
anchors   & 2.0 
\end{tabular}

\item {\bf 3.0-1} -- {\bf 3.0-6}: Minor bug fixes. Stable release for
  R 2.5.0-2.5.1.

\item {\bf 3.0} (July 20, 2007): Stable release for R 2.5.0-2.5.1.
  Introducing vignettes for each model. Improving documentation in the
  Zelig web site, improving citation style, improving {\tt
    help.zelig()} function, adding gam models, social network methods,
  logit gee model, adding support for cross-validation procedures and
  diagnostics tools, etc.

\item {\bf 2.8-3} (May 29, 2007):  Stable release for R 2.4.0-2.5.0.  
Fixed bugs in {\tt help.zelig()}, and summary for multinomial logit, 
bivariate probit, and bivariate logit with multiple imputation.  (Thanks 
to Brant Inman and Javier Marquez.)  First version dependencies are as 
follows:\newline
\begin{tabular}{ll}
     MASS       &   7.2-34 \\
     boot        &  1.2-27 \\
     VGAM        &  0.7-1  \\
     MCMCpack    &  0.8-2  \\
     mvtnorm     &  0.7-5  \\
     survival    &  2.31  \\
     sandwich    &  2.0-0  \\
     zoo         &  1.2-1  \\
     coda        &  0.10-7  \\
     nnet        &  7.2-34 \\
     sna         &  1.4   
\end{tabular}

\item {\bf 2.8-2} (March 3, 2007):  Stable release for R 2.4.0-2.4.1.  
Fixed bug in ARIMA simulation process.  

\item {\bf 2.8-1} (February 21, 2007): Stable release for R
  2.4.0-2.4.1. Made {\tt setx()} compatible with ordred factor variables
  (thanks to Mike Ward and Kirill Kalinin).  First order dependencies as 
  in version 2.8-1.

\item {\bf 2.8-0} (February 12, 2007):  Stable release for R
2.4.0-2.4.1.  Released ARIMA models and network analysis models (least
squares and logit) for sociomatrices.  First level dependencies are as
follows:\newline
\begin{tabular}{ll}
    MASS       &   7.2-31 \\          
    boot       &   1.2-27   \\        
    VGAM       &   0.7-1      \\      
    MCMCpack   &   0.7-4      \\      
    mvtnorm    &   0.7-5       \\     
    survival   &   2.31        \\     
    sandwich   &   2.0-0        \\    
    zoo        &   1.2-1         \\   
    coda       &   0.10-7        \\   
    nnet       &   7.2-31         \\  
    sna        &   1.4  \\
\end{tabular}

\item {\bf 2.7-5} (December 25, 2006):  Stable release for R 2.4.0-2.4.1.  
Fixed bug related to {\tt names.default()}, summary for multiple 
imputation methods, and prediction for ordinal response models (thanks to 
Brian Ripley, Chris Lawrence, and Ian Yohai).  

\item {\bf 2.7-4} (November 10, 2006):  Stable release for R 2.4.0.  Fixed
bugs related to R check.

\item {\bf 2.7-3} (November 9, 2006):  Stable release for R 2.4.0.  Fixed 
bugs related to R check.  

\item {\bf 2.7-2} (November 5, 2006):  Stable release for R 2.4.0.  
Temporarily removed {\sc arima} models.  
  
\item {\bf 2.7-1} (November 3, 2006): Stable release for R 2.4.0. Made
  changes regarding the S4 classes in VGAM. The {\sc arima} ({\tt arima})
  model for time series data added by Justin Grimmer.  First level
  dependencies are as follows:\newline
\begin{tabular}{ll}
     MASS      &     7.2-29 \\
     boot      &    1.2-26 \\
     VGAM      &    0.7-1 \\
     MCMCpack  &    0.7-4 \\
     mvtnorm   &    0.7-5 \\
     survival  &    2.29 \\
     sandwich  &    2.0-0 \\
     zoo       &    1.2-1 \\
     coda      &    0.10-7 \\
\end{tabular} 


\item {\bf 2.6-5} (September 14, 2006): Stable release for R 2.3.0-2.3.1.  Fixed bugs
  in bivariate logit, bivariate probit, multinomial logit, and
  model.matrix.multiple (related to changes in version 2.6-4, but not
  previous versions, thanks to Chris Lawrence).  First level 
  dependencies are as follows:\newline
\begin{tabular}{ll}
     MASS      &     7.2-27.1 \\
     boot      &    1.2-26 \\
     VGAM      &    0.6-9 \\
     MCMCpack  &    0.7-1 \\
     mvtnorm   &    0.7-2 \\
     survival  &    2.28 \\
     sandwich  &    1.1-1 \\
     zoo       &    1.0-6 \\
     coda      &    0.10-5 \\
\end{tabular} 

\item {\bf 2.6-4} (September 8, 2006): Stable release for R
2.3.0-2.3.1.  Fixed bugs in {\tt setx()}, and bugs related to {\tt
multiple} and the multinomial logit model.  Added instructions for
installing Fortran tools for Intel macs.  Added the R$\times$C
ecological inference model.  (thanks to Kurt Hornik, Luke Keele, Joerg
Mueller-Scheessel, and B. Dan Wood)

\item {\bf 2.6-3} (June 19, 2006): Stable release for R 2.0.0-2.3.1.  
Fixed bug in {\sc vdc} interface functions, and {\tt parse.formula()}.  
(thanks to Micah Altman, Christopher N. Lawrence, and Eric Kostello)

\item {\bf 2.6-2} (June 7, 2006): Stable release for R 2.0.0-2.3.1.
Removed R $\times$ C {\sc ei}.  Changed {\tt data = list()} to {\tt
data = mi()} for multiply-imputed data frames.  First level version
compatibilities are as for version 2.6-1.  

\item {\bf 2.6-1} (April 29, 2006): Stable release for R 2.0.0-2.2.1.
Fixed major bug in ordinal logit and ordinal probit expected value
simulation procedure (does not affect Bayesian ordinal probit).
(reported by Ian Yohai) Added the following ecological inference {\sc
ei} models: Bayesian hierarchical {\sc ei}, Bayesian dynamic {\sc ei},
and R $\times$ C {\sc ei}.  First level version compatibilities (at
time of release) are as follows:\newline
\begin{tabular}{ll}
MASS      &    7.2-24 \\
boot      &    1.2-24  \\        
VGAM      &    0.6-8   \\       
MCMCpack  &    0.7-1   \\       
mvtnorm   &    0.7-2   \\       
survival  &    2.24    \\       
sandwich  &    1.1-1   \\       
zoo       &    1.0-6   \\       
coda      &    0.10-5  \\
\end{tabular}

\item {\bf 2.5-4} (March 16, 2006): Stable release for R 2.0.0-2.2.1.  Fixed bug 
related to windows build.  First-level dependencies are the same as in version 
2.5-1.

\item {\bf 2.5-3} (March 9, 2006): Stable release for R 2.0.0-2.2.1.  Fixed bugs 
related to VDC GUI.  First level dependencies are the same as in version 2.5-1.  

\item {\bf 2.5-2} (February 3, 2006): Stable release for R 2.0.0-2.2.1.  Fixed bugs 
related to VDC GUI.  First level dependencies are the same as in version 2.5-1.

\item {\bf 2.5-1} (January 31, 2006): Stable release 
for R 2.0.0-2.2.1.  Added methods for multiple equation models.  Added
tobit regression.  Fixed bugs related to robust estimation and upgrade of sandwich and 
zoo packages.  Revised {\tt setx()} to use environments.  Added 
{\tt current.packages()} to retrieve version of packages upon which Zelig 
depends.  First level version compatibilities (at time of release) 
are as follows: \newline
\begin{tabular}{ll}
     MASS       & 7.2-24 \\
     boot       & 1.2-24 \\
     VGAM       & 0.6-7 \\
     mvtnorm    & 0.7-2 \\ 
     survival   & 2.20  \\
     sandwich   & 1.1-0 \\
     zoo        & 1.0-4 \\
     MCMCpack   & 0.6-6 \\
     coda       & 0.10-3 \\
\end{tabular}

\item \textbf{2.4-7} (December 10, 2005): Stable release for R 
2.0.0-2.2.2.  Fixed the environment of {\tt eval()} called within {\tt
  setx.default()} (thanks to Micah Altman).

\item \textbf{2.4-6} (October 27, 2005): Stable release for R 
2.0.0-2.2.2.  Fixed bug related to simulation for Bayesian Normal regression.

\item \textbf{2.4-5} (October 18, 2005): Stable release for R 2.0.0-2.2.0.  
Fixed installation instructions.  

\item \textbf{2.4-4} (September 29, 2005): Stable release for R 
2.0.0-2.2.0.  Fixed {\tt help.zelig()} links.  

\item \textbf{2.4-3} (September 29, 2005): Stable release for R
2.0.0-2.2.0.  Revised {\tt matchit()} documentation.  

\item \textbf{2.4-2} (August 30, 2005): Stable release for R 2.0.0-2.1.1.  
  Fixed bug in {\tt setx()} related to {\tt as.factor()} and {\tt I()}.  
Streamlined {\tt qi.survreg()}.  

\item \textbf{2.4-1} (August 15, 2005): Stable release for R 2.0.0-2.1.1.  Added the
  following Bayesian models: factor analysis, mixed factor analysis,
  ordinal factor analysis, unidimensional item response theory,
  k-dimensional item response theory, logit, multinomial logit,
  normal, ordinal probit, Poisson, and tobit.  Also fixed minor bug in
formula (long variable names coerced to list).  

\item \textbf{2.3-2} (August 5, 2005): Stable release for R 2.0.0-2.1.1.  
Fixed bug in simulation procedure for lognormal model. 

\item \textbf{2.3-1} (August 4, 2005): Stable release for R 2.0.0-2.1.1.  
Fixed documentation errors related to model parameterization and code bugs 
related to first differences and conditional prediction for exponential, 
lognormal, and Weibull models.  (reported by Alison Post)

\item \textbf{2.2-4} (July 30, 2005):  Stable release for R
2.0.0-2.1.1.  Revised relogit, adding option for weighting in addition
to prior correction.  (reported by Martin Pl\"oderl)

\item \textbf{2.2-3} (July 24, 2005): Stable release for R 2.0.0-2.1.1.  
Fixed bug associated with robust standard errors for negative binomial.  

\item \textbf{2.2-2} (July 13, 2005): Stable release for R 2.0.0-2.1.1.  
Fixed bug in setx().  (reported by Ying Lu)

\item \textbf{2.2-1} (July 11, 2005):  Stable release for R 2.0.0-2.1.0.  
Revised ordinal probit to use MASS library.  Added robust standard errors 
for the following regression models: exponential, gamma, logit, lognormal, 
least squares, negative binomial, normal (Gaussian), poisson, probit, and 
weibull.

\item \textbf{2.1-4} (May 22, 2005):  Stable release for R 1.9.1-2.1.0.  Revised help.zelig()
  to deal with CRAN build of Windows version.  Added recode of slots to lists    
  in NAMESPACE.  Revised {\tt install.R} script to deal with changes to {\tt install.packages()}.  (reported by Dan Powers and 
Ying Lu)

\item \textbf{2.1-3} (May 9, 2005):  Stable release for R 1.9.1-2.1.0.  
Revised param.lm() function to work with bootstrap simulation.  (reported 
by Jens Hainmueller)

\item \textbf{2.1-2} (April 14, 2005):  Stable release for R 1.9.1-2.1.0.  
Revised summary.zelig(). 
\item \textbf{2.1-1} (April 7, 2005):  Stable release for R 1.9.1-2.1.0.  
Fixed bugs in 
  NAMESPACE and summary.vglm().
\item \textbf{2.0-14} (April 5, 2005): Stable release for R
  1.9.1-2.0.1. Added {\tt summary.vglm()} to ensure the compatibility with VGAM 0.6-2.
\item \textbf{2.0-13} (March 11, 2005): Stable release for R 1.9.1-2.0.1.  
Fixed bugs in {\tt NAMESPACE} and R-help file for {\tt rocplot()}.  
\item \textbf{2.0-12} (February 20, 2005): Stable release for R 
1.9.1-2.0.1.  Added {\tt plot = TRUE} option to {\tt rocplot()}.  
\item \textbf{2.0-11} (January 14, 2005): Stable release for R 1.9.1-2.0.1.  
Changed class name for subsettted models from {\tt "multiple"} to {\tt 
"strata"}, and modified affected functions.  
\item \textbf{2.0-10} (January 5, 2005): Stable release for R 1.9.1 and R 
2.0.0.  Fixed bug in ordinal logit simulation procedure.  (reported by Ian 
Yohai)
\item \textbf{2.0-9} (October 21, 2004): Stable release for R 1.9.1 
\emph{and} R 2.0.0 (Linux and Windows).  Fixed bug in NAMESPACE file.  
\item \textbf{2.0-8} (October 18, 2004): Stable release for R 1.9.1 
\emph{and} R 2.0.0 (Linux only).  Revised for submission to CRAN.  
\item \textbf{2.0-7} (October 14, 2004): Stable release for R 1.9.1 \emph{and} R 2.0.0 (Linux 
only).  Fixed bugs in {\tt summary.zelig()},  NAMESPACE, and assorted bugs related to new R 
release.  Revised syntax for multiple equation models.  
\item \textbf{2.0-6} (October 4, 2004): Stable release for R 1.9.1.
Fixed problem with NAMESPACE.  
\item \textbf{2.0-5} (September 25, 2004): Stable release for R 1.9.1.  
Changed installation procedure to source {\tt install.R} from Zelig 
website.
\item \textbf{2.0-4} (September 22, 2004): Stable release for R 1.9.1.  Fixed 
typo in installation directions, implemented NAMESPACE, rationalized {\tt 
summary.zelig()}, and tweaked documentation for least squares.  
\item \textbf{2.0-3} (September 1, 2004): Stable release for 
R 1.9.1.  Fixed bug in conditional prediction for survival models.  
\item \textbf{2.0-2} (August 25, 2004): Stable release for R 1.9.1.  
Removed predicted values from {\tt ls}.  
\item \textbf{2.0-1b} (July 16, 2004): Stable release for R 1.9.1.  MD5
  checksum problem fixed.  Revised {\tt plot.zelig()} command to be a
  generic function with methods assigned by the model.  Revised entire
  architecture to accept multiply imputed data sets with strata.
  Added functions to simplify adding models.  Completely restructured
  reference manual.  Fixed bugs related to conditional prediction in
  setx and summarizing strata in summary.zelig.
 \item \textbf{1.1-2} (June 24, 2004): Stable release for R 1.9.1 (MD5
  checksum problem not fixed, but does not seem to cause problems).
  Fixed bug in {\tt help.zelig()}.  (reported by Michael L. Levitan)
\item \textbf{1.1-1} (June 14, 2004): Stable release for R 1.9.0.
  Revised {\tt zelig()} procedure to use {\tt zelig2model()} wrappers,
  revised {\tt help.zelig()} to use a data file with extension {\tt
    .url.tab}, and revised {\tt setx()} procedure to take a list of
  {\tt fn} to apply to variables, and such that {\tt fn = NULL}
  returns the entire {\tt model.matrix()}.
 \item \textbf{1.0-8} (May 27, 2004): Stable release for R 1.9.0.
   Fixed bug in simulation procedure for survival models.  (reported
   by Elizabeth Stuart) 
 \item \textbf{1.0-7} (May 26, 2004): Stable release for R 1.9.0. Fixed 
   bug in relogit simulation procedure.  (reported by Tom Vanwellingham)
 \item \textbf{1.0-6} (May 11, 2004):  Stable release for R 1.9.0.  
   Fixed bug in setx.default, which had previously failed to ignore 
   extraneous variables in data frame.  (reported by Steve Purpura) 
 \item \textbf{1.0-5} (May 7, 2004): Replaced relogit procedure with 
   memory-efficient version. (reported by Tom Vanwellingham)
 \item \textbf{1.0-4} (April 19, 2004):  Stable release for R 1.9.0.  
   Added vcov.lm method; changed print for summary.relogit. 
 \item \textbf{1.0-2} (April 16, 2004): Testing distribution for R 1.9.0. 
 \item \textbf{1.0-1} (March, 23, 2004): Stable release for R 1.8.1. 
\end{itemize}

\section{What's Next?}

We have several plans for expanding and improving Zelig.  Major
changes slated for Version 3.0 (and beyond) include:  

\begin{itemize}
\item Hierarchical and multi-level models
\item Ecological inference models
\item GEE models
\item Neural network models
\item Average treatment effects for everyone (treated and control
units)
\item Time-series cross-sectional models (via {\tt nlme})
\item Generalized boosted regression model (via {\tt gbm})
\item Saving random seeds to ensure exact replication
\end{itemize}

If you have suggestions, or packages that you would like to contribute
to Zelig, please email our listserv at \hlink{zelig@lists.gking.harvard.edu}{mailto:zelig@lists.gking.harvard.edu}.


\input{faq}

\bibliographystyle{asa}
\bibliography{gk,gkpubs}

\end{document}

%%% Local Variables: 
%%% mode: LaTeX
%%% TeX-master: t
%%% End:


\include{commandsRd/setx}
\include{commandsRd/sim}
\section{{\tt summary}: Summarizing Zelig Output}
\label{ss:summary}

\subsubsection{Description}

Summarize R objects using using the generic function {\tt summary()}.
R automatically recognizes the type of object (lists, data frames,
variables, etc.) and selects the appropriate {\tt summary()} function.

\subsubsection{Syntax}

\begin{verbatim}
> summary(z.out, subset = NULL, ...)
> summary(s.out, subset = NULL, CI = 95, 
          stats = c("mean", "sd", "min", "max"), ...)
\end{verbatim}
To set the desired number of digits in the summary output, use {\tt
  options(digits = x)}, where {\tt x} is the desired number of digits,
prior to using the {\tt summary()} command.

\subsubsection{Arguments for multiply-imputed {\tt zelig()} output}
For {\tt zelig()} output created with $m$ multiply-imputed data sets,
the {\tt z.out} object contains one set of output for each imputed
data set.  Options for multiply-imputed {\tt zelig()} output include:
   \begin{itemize}
   \item {\tt subset}: specifies which of the $m$ sets of output to
     summarize.  Possible arguments include:
       \begin{itemize}
       \item {\tt NULL}: (default) which produces one summary by
         averaging the coefficients and standard errors from all the
         imputed data sets using the Rubin rules (\hlink{King,
           Honaker, Joseph, Scheve
           (2000)}{http://gking.harvard.edu/files/abs/evil-abs.shtml},
         p. 53).
       \item a numeric vector: consisting of any set of numbers in
         $[1,m]$.  For example, you may use {\tt summary(z.out,
           subset = 2:4)} to view the output from the second, third,
         and fourth imputed data sets only.  
       \end{itemize}
     \item{\dots}: additional options passed to {\tt print()}.  
    \end{itemize}

\subsubsection{Arguments for subsetted {\tt zelig()} output}
If you selected the {\tt by} option, {\tt zelig()} creates an {\tt
  z.out} object with one set of regression output for each of the
unique values in the {\tt by} variable.  Options for subsetted
  analyses are:  
\begin{itemize}
\item {\tt subset}: specifies which regression output to summarize.
  You cannot summarize multiple analyses generated using the {\tt
    "by"} option in one summary, as each of the levels in the {\tt by}
  variable may have a different number of associated observations.  (A
  weighted summary would also be inappropriate, as the weighted
  coefficients would be identical to the coefficients generated
  without subsetting.)
\begin{itemize}
\item numeric vector: specifying which sets of regression output to
  summarize.  By default, {\tt summary()} for subsetted output
  sequentially summarizes the regression output for each unique value
  in the {\tt by} variable.  You may choose to summarize just the
  fifteenth unique value by typing:  {\tt summary(z.out, subset =
    15)}.  
\end{itemize}
\item {\tt \dots}: Additional arguments passed to {\tt print()}.
\end{itemize}
   
\subsubsection{Arguments for {\tt sim()} output}
Summaries of {\tt sim()} output includes these additional options:
   \begin{itemize}
     \item {\tt subset}:  Only valid for {\tt sim()} output created
       using more than one observation in {\tt x}.  You may select to
       summarize the quantities of interest created by each
       observation in {\tt x} using one of the following options: 
       \begin{itemize}
       \item {\tt NULL}: (default) produces one summary per quantity
         of interest by combining the simulations from each
         observation into a single set and then summarizing.
         \item {\tt all}:  summarizes the quantity of interest
           produced by each observation separately.  
         \item a numeric vector: specifying the observations to summarize 
       \end{itemize}
     \item {\tt CI}: A value between 0 and 100, for the percentage
       confidence interval.  The default value is 95, for a 95 percent
       confidence interval.  
     \item {\tt stats}: A vector specifying the statistics to
       calculate for each quantity of interest.  The default values
       are {\tt c("mean", "sd", "min", "max")}.  
   \end{itemize}
Using {\tt summary()} on {\tt sim()} output from multiple analyses
will automatically weight the quantities of interest according to the
proportion of observations in each strata.  

\subsubsection{Output Values}
\begin{itemize}
\item For {\tt zelig()} output objects, {\tt summary()} returns a
  formatted summary of the regression output, including the usual
  table of coefficients, standard errors, and $t$-values.  Additional
  output depends on the statistical model, the names of which may
  be viewed using the {\tt names(summary(z.out))} command.
\item For {\tt sim()} output objects, {\tt summary()} returns summary
  statistics for each quantity of interest, including a default 95\%
  confidence interval.  Use {\tt names(summary(s.out))} to see the
  available output.  Note that {\tt qi.stats} contains a useful matrix
  of summary statistics for each of the quantities of interest.  For
  example, {\tt summary(s.out)\$qi.stats\$ev}, extracts the summary
  matrix of expected values.
\end{itemize}

\subsubsection{See Also}

Advanced users may wish to refer to {\tt help.zelig("print")},
as well as {\tt help(summary)}, {\tt help(summary.lm)}, {\tt
  help(summary.glm)}, and {\tt help(anova)} in the base package.

\subsubsection{Contributors}

Kosuke Imai, Gary King, and Olivia Lau added summary methods for
certain {\tt zelig()} models, and for {\tt sim()} output.


%%% Local Variables: 
%%% mode: latex
%%% TeX-master: t
%%% End: 


\include{commandsRd/plot.zelig}
\section{{\tt print}: Printing Quantities of Interest}
\label{ss:print}

\subsubsection{Description}
The {\tt print} command formats summary output and allows users to
specify the number of decimal places as an optional argument.  

\subsubsection{Syntax}
\begin{verbatim}
> print(x, digits = 3, print.x = FALSE)
\end{verbatim}

\subsubsection{Arguments}

\begin{itemize}
\item {\tt x}: the object to be printed may be {\tt z.out} output
  from {\tt zelig()}, {\tt x.out} output from {\tt setx()}, {\tt s.out}
  output from {\tt sim()}, or other R data structures.
\item {\tt digits}: the minimum number of significant digits to return
  for all elements of $x < 0$.  By default, {\tt print()} avoids
  scientific notation, but setting the number of digits to 1 will
  frequently force output in scientific notation.  The number of {\tt
    digits} is not the number of significant digits for all output
  values, but the minimum number of significant digits for the
  smallest value in {\tt x} between -1 and 1; this governs the number
  of significant digits in the rest of the values with decimal output.
\item {\tt print.x}: a logical value for {\tt sim()} output, which
  specifies whether to print a summary ({\tt print.x = FALSE}, the
  default) of the {\tt x} and {\tt x1} \emph{inputs} to {\tt sim()},
  or the complete set of inputs (optionally, {\tt print.x = TRUE}).
\end{itemize}

\subsubsection{Examples}
\begin{verbatim}
> print(summary(z.out), digits = 2)
> print(summary(s.out), digits = 3, print.x = TRUE)
\end{verbatim}

\subsubsection{See Also}

Advanced users may wish to refer to {\tt help(print)}.

\subsubsection{Contributors}

Kosuke Imai, Gary King, and Olivia Lau added print methods for {\tt
  sim()} output, and {\tt summary()} output for Zelig objects.   

%%% Local Variables: 
%%% mode: latex
%%% TeX-master: t
%%% End: 

\include{commandsRd/repl}

\chapter{Supplementary Commands}

\include{commands/matchit}
% author: Matt Owen
% documentation for using the 'mi' class
\documentclass[a4paper,11pt]{article}

% Packages
\usepackage{listings}
\lstset{language=R}


\begin{document}


\section*{Fitting Multiple Statistical Models Simultaneously}

It is often important, especially when computing data based on counterfactuals, to have a side-by-side comparison of the different fitted models and their \emph{quantities of interest}.  Typically, the statistician interested in comparing multiple models would:
\begin{itemize}
	\item{Prepare Several Different Datasets, by:}
	\begin{itemize}
		\item{Splitting a Dataset into Several Categorized by Some Particular Quality}
		\item{Preparing Counterfactual Data Based on a Pre-existing Dataset}
		\item{Collecting Several Different Data-sets Analyzing Similar Data}
	\end{itemize}
	\item{Compute a Fitted Model for Each Dataset}
	\item{Simulate Quantities of Interest for Each Fitted Model}
	\item{Organize the Results in a Logical Fashion}
\end{itemize}

This process can be quite arduous, and the programming syntax associated with these sorts of processes is often unintuitive and cumbersome.  Resulting from the necessity to simplify this process, the Zelig software package offers an intuitive and simple syntax to produce these results.

\section{The \emph{mi} class: An Easy Method for Multiply Imputed Data}

Zelig offers a simple way to fit models for multiply imputed data.  By submitting an \emph{mi} object to the data parameter of Zelig, the statistician can prepare analysis of multiply imputed data quickly and easily.

\subsection{Parameters and Return-values of the ``mi'' Function}


% Describe arguments/return-values
\begin{description}
	\item[@...:]{A series of \emph{data.frame} objects.  Note: if a non-\emph{data.frame} object is submitted, the \emph{mi} class will raise an error.}
	\item[Return Value:]{An \emph{mi} object that the \emph{zelig} function will use to run analysis on the datasets.}
\end{description}


% clear page, and begin the section on mi
\pagebreak
\subsection{Examples of Multiply Imputed Data}


% CODE EXAMPLE
% ---- -------
\begin{lstlisting}
library(Zelig)

data(turnout)

# split the turnout data into 2 sets
turnout1 <- turnout[1:1000,]
turnout2 <- turnout[1001:2000, ]

# create the mi object
split.turnout <- mi(turnout1, turnout2)

# fit the models
z.out <- zelig(vote ~ race + educate, model="logit", data=split.turnout)

# set a counterfactual on both data-sets
x.out <- setx(z.out, age=65)

# simulate quantities of interest for both models
s.out <- sim(z.out,  x = x.out)

# output results
summary(s.out)
\end{lstlisting}

% some spacing between sections
\vspace{4mm}

\subsection*{Important Notes}
\begin{itemize}
	\item{When \emph{data} is submitted to the zelig function via the \emph{mi} class, Zelig returns an object of both class \emph{MI} and \emph{zelig}.  This is in contrast to usually simply being of the class \emph{zelig}.}
	\item{...}
\end{itemize}

%
\subsection{Developing Models to Interact with \emph{mi} Objects}
For the most part, the Zelig developer should not need any additional code to have their model work with the \emph{mi} class.  However, an API exists for the managing and printing of results from this class.


\section{The \emph{by} Keyword: Subsetting Datasets by Factors}

For some models, a the significance of a factor variable - a variable whose range is a finite, specific number of values - is best understood by subsetting a dataset by each of its potential values.  For example, when trying to analyze voting patterns, it may make sense to fit your model separately by race, religion, or birth-state.  Typically, this would require creating a new data-set for each of the possible combinations of race, religion, and birth-state.  Zelig, via the emph{by} keyword, employs a straightforward method to create such models.

\subsection{How-to Use the \emph{by} Keyword}

\begin{itemize}
	\item{\emph{by} must be a column name of the dataset (or datasets) being passed to the model}
	\item{The formula must not contain any of the values passed into \emph{by}}
	\item{\emph{by} should be a column describing factors - not numerical data.  Otherwise, results may be too numerous or nonsensical}
\end{itemize}

\subsection{Code Example}

\begin{lstlisting}
library(Zelig)

data(turnout)

z.out <- zelig(
	vote ~ age + income + educate,
	model = "logit",
	data  = turnout,
	by    = "race"
	)

x.out <- setx(z.out)

s.out <- sim(z.out, x.out)

summary(s.out)
\end{lstlisting}

% some spacing between sections
\vspace{4mm}

\subsection{Explanation of the Above Code}

The above code individually fits, analyzes, and simulates a model for voter turnout based on age, income, and race.  By passing the \emph{by} keyword the value ``race'', our model knows to split the turnout data object into separate components: one for each different in the data column labeled ``race''.  The beauty in this script is the simplicity in which it does this process.  That is, it differs in no way from a typical call to \emph{zelig} with the exception that the \emph{by} keyword is specified.

\section{Using the \emph{mi} Class and the \emph{by} Keyword Together}

No special programming is necessary to use both the \emph{mi} class along with the \emph{by} keyword!  Just simply pass an \emph{mi} object to data, and pass the names of columns to subset the data by to the \emph{by} parameter, and \emph{zelig} will handle the rest.  In fact, the printout that Zelig will produce will automatically categorize the data according to which dataset was submitted and which factor was subsetted.  For complex operations an API is available for the savvy developer.

\end{document}






























\include{commandsRd/network}
\include{commandsRd/plot.ci}
\include{commandsRd/plot.surv}
\include{commandsRd/rocplot}
\include{commandsRd/ternaryplot}
\include{commandsRd/ternarypoints}

\chapter{Models Zelig Can Run}\label{s:model.details}

This section describes the mathematical components of the models
supported by Zelig, using whenever possible the classification and
notation of \cite{King89}.  Most models have a \emph{stochastic
  component} (probability density given certain parameters) and
a \emph{systematic component} (deterministic functional form that
specifies how one or more of the parameters varies over the observed
values $y_i$ as a function of the explanatory variables $x_i$).

Let $Y_i$ be a random outcome variable, realized as $i = 1, \dots, n$
observations $y_i$.  For the probability density $f(\cdot)$ with
systematic feature $\theta_i$ varying over $i$ and a scalar ancillary
parameter $\alpha$ (constant over $i$), the stochastic component is
given by
\begin{equation*}
Y_i \sim f(y_i \mid \theta_i, \alpha).
\end{equation*}

For a functional form $g(\cdot)$, $k$ explanatory variables $X_i$, and
effect parameters $\beta$, the systematic component is:  
\begin{equation*}
\theta_i = g(x_i, \beta).
\end{equation*}

Using the definitions of \hlink{King, Tomz, and Wittenberg,
    2000}{http://gking.harvard.edu/files/abs/making-abs.shtml},
  \nocite{KinTomWit00}
Zelig generates at least two quantities of interest:   
\begin{itemize}
\item The predicted value is a random draw from the stochastic component
  given random draws of $\beta$ and $\alpha$ from their sampling (or
  posterior) distribution.  
\item The expected value is the \emph{mean} of the stochastic component
  given random draws of $\beta$ and $\alpha$ from their sampling (or
  posterior) distributions.  For computational efficiency, Zelig
  deterministically calculates the expected values from the simulated
  parameters whenever possible. 
\end{itemize}

Both the predicted values and expected values produced by Zelig can be
displayed as histograms or density estimates (to summarize the full
sampling or posterior density), or summarized with confidence
intervals (by sorting the simulations and taking the 5th and 95th
percentile values for a 90\% confidence interval for example),
standard errors (by taking the standard deviation of the simulations),
or point estimates (by averaging the simulations).  The point estimate
of predicted and expected values are the same only in linear models.
In almost all situations, simulations from predicted values have more
variance than expected values.  As the number of simulations increases
the distribution of the expected values tends toward a constant; the
distribution of the predicted values does not collapse as the number
of simulations increases.

\include{aov}
\include{arima}
\include{blogit}
\include{bprobit}
\include{chopit}
\include{cloglog.net}
\include{coxph}
\include{ei.dynamic}
\include{ei.hier}
\include{ei.RxC}
\include{exp}
\include{factor.bayes}
\include{factor.mix}
\include{factor.ord}
\include{gamma}
\include{gamma.gee}
\include{gamma.mixed}
\include{gamma.net}
\include{gamma.survey}
\include{irt1d}
\include{irtkd}
\include{logit}
\include{logit.bayes}
\include{logit.gam}
\include{logit.gee}
\include{logit.gee}
\include{logit.mixed}
\include{logit.mixed}
\include{logit.net}
\include{logit.survey}
\include{lognorm}
\documentclass{article}

\title{
  ls: Least Squares Regression for Continuous
  Dependent Variables
}
\author{Matt Owen, Olivia Lau, Kosuke Imai, and Gary King}




\usepackage{bibentry}
\usepackage{graphicx}
\usepackage{natbib}
\usepackage{amsmath}
\usepackage{url}
\usepackage{Zelig}
\usepackage{Sweave}

%\VignetteIndexEntry{Least Squares Regression for Continuous Dependent Variables}
%\VignetteDepends{Zelig, stats}
%\VignetteKeyWords{model,least squares,continuous, regression}
%\VignettePackage{Zelig}

\begin{document}

\nobibliography*



\section{{\tt ls}: Least Squares Regression for Continuous
Dependent Variables}
\label{ls}

Use least squares regression analysis to estimate the best linear
predictor for the specified dependent variables.

\subsubsection{Syntax}

\begin{verbatim}
> z.out <- zelig(Y ~ X1 + X2, model = "ls", data = mydata)
> x.out <- setx(z.out)
> s.out <- sim(z.out, x = x.out)
\end{verbatim}

\subsubsection{Additional Inputs}  

In addition to the standard inputs, {\tt zelig()} takes the following
additional options for least squares regression:  
\begin{itemize}
\item {\tt robust}: defaults to {\tt FALSE}.  If {\tt TRUE} is
selected, {\tt zelig()} computes robust standard errors based on
sandwich estimators (see \cite{Zeileis04}, \cite{Huber81}, and
\cite{White80}).  The default type of robust standard error is
heteroskedastic consistent (HC), \emph{not} heteroskedastic and
autocorrelation consistent (HAC).  

In addition, {\tt robust} may be a list with the following options:  
\begin{itemize}
\item {\tt method}:  choose from 
\begin{itemize}
\item {\tt "vcovHC"}: (the default if {\tt robust = TRUE}), HC standard errors.
\item {\tt "vcovHAC"}: HAC standard errors without weights.  
\item {\tt "kernHAC"}: HAC standard errors using the weights given in
\cite{Andrews91}.   
\item {\tt "weave"}: HAC standard errors using the weights given in
\cite{LumHea99}.
\end{itemize} 
\item {\tt order.by}: only applies to the HAC methods above.  Defaults to
{\tt NULL} (the observations are chronologically ordered as in the
original data).  Optionally, you may specify a time index (either as
{\tt order.by = z}, where {\tt z} exists outside the data frame; or
as {\tt order.by = \~{}z}, where {\tt z} is a variable in the data
frame).  The observations are chronologically ordered by the size of
{\tt z}.
\item {\tt \dots}:  additional options passed to the functions
specified in {\tt method}.  See the {\tt sandwich} library and
\cite{Zeileis04} for more options.   
\end{itemize}
\end{itemize}

\subsubsection{Examples}\begin{enumerate}
\item Basic Example with First Differences

Attach sample data:
\begin{Schunk}
\begin{Sinput}
>  data(macro)
\end{Sinput}
\end{Schunk}
Estimate model:
\begin{Schunk}
\begin{Sinput}
>  z.out1 <- zelig(unem ~ gdp + capmob + trade, model = "ls", data = macro)
\end{Sinput}
\end{Schunk}
Summarize regression coefficients:
\begin{Schunk}
\begin{Sinput}
>  summary(z.out1)
\end{Sinput}
\end{Schunk}
Set explanatory variables to their default (mean/mode) values, with
high (80th percentile) and low (20th percentile) values for the trade variable:
\begin{Schunk}
\begin{Sinput}
>  x.high <- setx(z.out1, trade = quantile(macro$trade, 0.8))
\documentclass{article}

\title{
  ls: Least Squares Regression for Continuous
  Dependent Variables
}
\author{Matt Owen, Olivia Lau, Kosuke Imai, and Gary King}




\usepackage{bibentry}
\usepackage{graphicx}
\usepackage{natbib}
\usepackage{amsmath}
\usepackage{url}
\usepackage{Zelig}
\usepackage{Sweave}

%\VignetteIndexEntry{Least Squares Regression for Continuous Dependent Variables}
%\VignetteDepends{Zelig, stats}
%\VignetteKeyWords{model,least squares,continuous, regression}
%\VignettePackage{Zelig}

\begin{document}

\nobibliography*



\section{{\tt ls}: Least Squares Regression for Continuous
Dependent Variables}
\label{ls}

Use least squares regression analysis to estimate the best linear
predictor for the specified dependent variables.

\subsubsection{Syntax}

\begin{verbatim}
> z.out <- zelig(Y ~ X1 + X2, model = "ls", data = mydata)
> x.out <- setx(z.out)
> s.out <- sim(z.out, x = x.out)
\end{verbatim}

\subsubsection{Additional Inputs}  

In addition to the standard inputs, {\tt zelig()} takes the following
additional options for least squares regression:  
\begin{itemize}
\item {\tt robust}: defaults to {\tt FALSE}.  If {\tt TRUE} is
selected, {\tt zelig()} computes robust standard errors based on
sandwich estimators (see \cite{Zeileis04}, \cite{Huber81}, and
\cite{White80}).  The default type of robust standard error is
heteroskedastic consistent (HC), \emph{not} heteroskedastic and
autocorrelation consistent (HAC).  

In addition, {\tt robust} may be a list with the following options:  
\begin{itemize}
\item {\tt method}:  choose from 
\begin{itemize}
\item {\tt "vcovHC"}: (the default if {\tt robust = TRUE}), HC standard errors.
\item {\tt "vcovHAC"}: HAC standard errors without weights.  
\item {\tt "kernHAC"}: HAC standard errors using the weights given in
\cite{Andrews91}.   
\item {\tt "weave"}: HAC standard errors using the weights given in
\cite{LumHea99}.
\end{itemize} 
\item {\tt order.by}: only applies to the HAC methods above.  Defaults to
{\tt NULL} (the observations are chronologically ordered as in the
original data).  Optionally, you may specify a time index (either as
{\tt order.by = z}, where {\tt z} exists outside the data frame; or
as {\tt order.by = \~{}z}, where {\tt z} is a variable in the data
frame).  The observations are chronologically ordered by the size of
{\tt z}.
\item {\tt \dots}:  additional options passed to the functions
specified in {\tt method}.  See the {\tt sandwich} library and
\cite{Zeileis04} for more options.   
\end{itemize}
\end{itemize}

\subsubsection{Examples}\begin{enumerate}
\item Basic Example with First Differences

Attach sample data:
\begin{Schunk}
\begin{Sinput}
>  data(macro)
\end{Sinput}
\end{Schunk}
Estimate model:
\begin{Schunk}
\begin{Sinput}
>  z.out1 <- zelig(unem ~ gdp + capmob + trade, model = "ls", data = macro)
\end{Sinput}
\end{Schunk}
Summarize regression coefficients:
\begin{Schunk}
\begin{Sinput}
>  summary(z.out1)
\end{Sinput}
\end{Schunk}
Set explanatory variables to their default (mean/mode) values, with
high (80th percentile) and low (20th percentile) values for the trade variable:
\begin{Schunk}
\begin{Sinput}
>  x.high <- setx(z.out1, trade = quantile(macro$trade, 0.8))
\documentclass{article}

\title{
  ls: Least Squares Regression for Continuous
  Dependent Variables
}
\author{Matt Owen, Olivia Lau, Kosuke Imai, and Gary King}




\usepackage{bibentry}
\usepackage{graphicx}
\usepackage{natbib}
\usepackage{amsmath}
\usepackage{url}
\usepackage{Zelig}
\usepackage{Sweave}

%\VignetteIndexEntry{Least Squares Regression for Continuous Dependent Variables}
%\VignetteDepends{Zelig, stats}
%\VignetteKeyWords{model,least squares,continuous, regression}
%\VignettePackage{Zelig}

\begin{document}

\nobibliography*



\section{{\tt ls}: Least Squares Regression for Continuous
Dependent Variables}
\label{ls}

Use least squares regression analysis to estimate the best linear
predictor for the specified dependent variables.

\subsubsection{Syntax}

\begin{verbatim}
> z.out <- zelig(Y ~ X1 + X2, model = "ls", data = mydata)
> x.out <- setx(z.out)
> s.out <- sim(z.out, x = x.out)
\end{verbatim}

\subsubsection{Additional Inputs}  

In addition to the standard inputs, {\tt zelig()} takes the following
additional options for least squares regression:  
\begin{itemize}
\item {\tt robust}: defaults to {\tt FALSE}.  If {\tt TRUE} is
selected, {\tt zelig()} computes robust standard errors based on
sandwich estimators (see \cite{Zeileis04}, \cite{Huber81}, and
\cite{White80}).  The default type of robust standard error is
heteroskedastic consistent (HC), \emph{not} heteroskedastic and
autocorrelation consistent (HAC).  

In addition, {\tt robust} may be a list with the following options:  
\begin{itemize}
\item {\tt method}:  choose from 
\begin{itemize}
\item {\tt "vcovHC"}: (the default if {\tt robust = TRUE}), HC standard errors.
\item {\tt "vcovHAC"}: HAC standard errors without weights.  
\item {\tt "kernHAC"}: HAC standard errors using the weights given in
\cite{Andrews91}.   
\item {\tt "weave"}: HAC standard errors using the weights given in
\cite{LumHea99}.
\end{itemize} 
\item {\tt order.by}: only applies to the HAC methods above.  Defaults to
{\tt NULL} (the observations are chronologically ordered as in the
original data).  Optionally, you may specify a time index (either as
{\tt order.by = z}, where {\tt z} exists outside the data frame; or
as {\tt order.by = \~{}z}, where {\tt z} is a variable in the data
frame).  The observations are chronologically ordered by the size of
{\tt z}.
\item {\tt \dots}:  additional options passed to the functions
specified in {\tt method}.  See the {\tt sandwich} library and
\cite{Zeileis04} for more options.   
\end{itemize}
\end{itemize}

\subsubsection{Examples}\begin{enumerate}
\item Basic Example with First Differences

Attach sample data:
\begin{Schunk}
\begin{Sinput}
>  data(macro)
\end{Sinput}
\end{Schunk}
Estimate model:
\begin{Schunk}
\begin{Sinput}
>  z.out1 <- zelig(unem ~ gdp + capmob + trade, model = "ls", data = macro)
\end{Sinput}
\end{Schunk}
Summarize regression coefficients:
\begin{Schunk}
\begin{Sinput}
>  summary(z.out1)
\end{Sinput}
\end{Schunk}
Set explanatory variables to their default (mean/mode) values, with
high (80th percentile) and low (20th percentile) values for the trade variable:
\begin{Schunk}
\begin{Sinput}
>  x.high <- setx(z.out1, trade = quantile(macro$trade, 0.8))
\include{mlogit}
\include{mlogit.bayes}
\include{negbin}
\include{normal}
\include{normal.bayes}
\include{normal.gam}
\include{normal.gee}
\include{normal.net}
\include{normal.survey}
\include{ologit}
\include{oprobit}
\include{oprobit.bayes}
\include{poisson}
\include{poisson.bayes}
\include{poisson.gam}
\include{poisson.gee}
\include{poisson.mixed}
\include{poisson.net}
\include{poisson.survey}
\documentclass{article}

\title{
  probit: Probit Regression for Dichotomous
  Dependent Variables
}
\author{Matt Owen, Olivia Lau, Kosuke Imai, and Gary King}




\usepackage{bibentry}
\usepackage{graphicx}
\usepackage{natbib}
\usepackage{amsmath}
\usepackage{url}
\usepackage{Zelig}
\usepackage{Sweave}

%\VignetteIndexEntry{Probit Regression for Dichotomous Dependent Variables}
%\VignetteDepends{Zelig, stats}
%\VignetteKeyWords{model,least squares,continuous, regression}
%\VignettePackage{Zelig}

\begin{document}

\nobibliography*



\section{{\tt probit}: Probit Regression for Dichotomous Dependent Variables}\label{probit}

Use probit regression to model binary dependent variables
specified as a function of a set of explanatory variables.  

\subsubsection{Syntax}
\begin{verbatim}
> z.out <- zelig(Y ~ X1 + X2, model = "probit", data = mydata)
> x.out <- setx(z.out)
> s.out <- sim(z.out, x = x.out, x1 = NULL)
\end{verbatim}

\subsubsection{Additional Inputs} 

In addition to the standard inputs, {\tt zelig()} takes the following
additional options for probit regression:  
\begin{itemize}
\item {\tt robust}: defaults to {\tt FALSE}.  If {\tt TRUE} is
selected, {\tt zelig()} computes robust standard errors via the {\tt
sandwich} package (see \cite{Zeileis04}).  The default type of robust
standard error is heteroskedastic and autocorrelation consistent (HAC),
and assumes that observations are ordered by time index.

In addition, {\tt robust} may be a list with the following options:  
\begin{itemize}
\item {\tt method}:  Choose from 
\begin{itemize}
\item {\tt "vcovHAC"}: (default if {\tt robust = TRUE}) HAC standard
errors. 
\item {\tt "kernHAC"}: HAC standard errors using the
weights given in \cite{Andrews91}. 
\item {\tt "weave"}: HAC standard errors using the
weights given in \cite{LumHea99}.  
\end{itemize}  
\item {\tt order.by}: defaults to {\tt NULL} (the observations are
chronologically ordered as in the original data).  Optionally, you may
specify a vector of weights (either as {\tt order.by = z}, where {\tt
z} exists outside the data frame; or as {\tt order.by = \~{}z}, where
{\tt z} is a variable in the data frame).  The observations are
chronologically ordered by the size of {\tt z}.
\item {\tt \dots}:  additional options passed to the functions 
specified in {\tt method}.   See the {\tt sandwich} library and
\cite{Zeileis04} for more options.   
\end{itemize}
\end{itemize}

\subsubsection{Examples}
Attach the sample turnout dataset:
\begin{Schunk}
\begin{Sinput}
>  data(turnout)
\end{Sinput}
\end{Schunk}
Estimate parameter values for the probit regression:
\begin{Schunk}
\begin{Sinput}
>  z.out <- zelig(vote ~ race + educate,  model = "probit", data = turnout) 
\end{Sinput}
\end{Schunk}
\begin{Schunk}
\begin{Sinput}
>  summary(z.out)
\end{Sinput}
\end{Schunk}
Set values for the explanatory variables to their default values.
\begin{Schunk}
\begin{Sinput}
>  x.out <- setx(z.out)
\documentclass{article}

\title{
  probit: Probit Regression for Dichotomous
  Dependent Variables
}
\author{Matt Owen, Olivia Lau, Kosuke Imai, and Gary King}




\usepackage{bibentry}
\usepackage{graphicx}
\usepackage{natbib}
\usepackage{amsmath}
\usepackage{url}
\usepackage{Zelig}
\usepackage{Sweave}

%\VignetteIndexEntry{Probit Regression for Dichotomous Dependent Variables}
%\VignetteDepends{Zelig, stats}
%\VignetteKeyWords{model,least squares,continuous, regression}
%\VignettePackage{Zelig}

\begin{document}

\nobibliography*



\section{{\tt probit}: Probit Regression for Dichotomous Dependent Variables}\label{probit}

Use probit regression to model binary dependent variables
specified as a function of a set of explanatory variables.  

\subsubsection{Syntax}
\begin{verbatim}
> z.out <- zelig(Y ~ X1 + X2, model = "probit", data = mydata)
> x.out <- setx(z.out)
> s.out <- sim(z.out, x = x.out, x1 = NULL)
\end{verbatim}

\subsubsection{Additional Inputs} 

In addition to the standard inputs, {\tt zelig()} takes the following
additional options for probit regression:  
\begin{itemize}
\item {\tt robust}: defaults to {\tt FALSE}.  If {\tt TRUE} is
selected, {\tt zelig()} computes robust standard errors via the {\tt
sandwich} package (see \cite{Zeileis04}).  The default type of robust
standard error is heteroskedastic and autocorrelation consistent (HAC),
and assumes that observations are ordered by time index.

In addition, {\tt robust} may be a list with the following options:  
\begin{itemize}
\item {\tt method}:  Choose from 
\begin{itemize}
\item {\tt "vcovHAC"}: (default if {\tt robust = TRUE}) HAC standard
errors. 
\item {\tt "kernHAC"}: HAC standard errors using the
weights given in \cite{Andrews91}. 
\item {\tt "weave"}: HAC standard errors using the
weights given in \cite{LumHea99}.  
\end{itemize}  
\item {\tt order.by}: defaults to {\tt NULL} (the observations are
chronologically ordered as in the original data).  Optionally, you may
specify a vector of weights (either as {\tt order.by = z}, where {\tt
z} exists outside the data frame; or as {\tt order.by = \~{}z}, where
{\tt z} is a variable in the data frame).  The observations are
chronologically ordered by the size of {\tt z}.
\item {\tt \dots}:  additional options passed to the functions 
specified in {\tt method}.   See the {\tt sandwich} library and
\cite{Zeileis04} for more options.   
\end{itemize}
\end{itemize}

\subsubsection{Examples}
Attach the sample turnout dataset:
\begin{Schunk}
\begin{Sinput}
>  data(turnout)
\end{Sinput}
\end{Schunk}
Estimate parameter values for the probit regression:
\begin{Schunk}
\begin{Sinput}
>  z.out <- zelig(vote ~ race + educate,  model = "probit", data = turnout) 
\end{Sinput}
\end{Schunk}
\begin{Schunk}
\begin{Sinput}
>  summary(z.out)
\end{Sinput}
\end{Schunk}
Set values for the explanatory variables to their default values.
\begin{Schunk}
\begin{Sinput}
>  x.out <- setx(z.out)
\documentclass{article}

\title{
  probit: Probit Regression for Dichotomous
  Dependent Variables
}
\author{Matt Owen, Olivia Lau, Kosuke Imai, and Gary King}




\usepackage{bibentry}
\usepackage{graphicx}
\usepackage{natbib}
\usepackage{amsmath}
\usepackage{url}
\usepackage{Zelig}
\usepackage{Sweave}

%\VignetteIndexEntry{Probit Regression for Dichotomous Dependent Variables}
%\VignetteDepends{Zelig, stats}
%\VignetteKeyWords{model,least squares,continuous, regression}
%\VignettePackage{Zelig}

\begin{document}

\nobibliography*



\section{{\tt probit}: Probit Regression for Dichotomous Dependent Variables}\label{probit}

Use probit regression to model binary dependent variables
specified as a function of a set of explanatory variables.  

\subsubsection{Syntax}
\begin{verbatim}
> z.out <- zelig(Y ~ X1 + X2, model = "probit", data = mydata)
> x.out <- setx(z.out)
> s.out <- sim(z.out, x = x.out, x1 = NULL)
\end{verbatim}

\subsubsection{Additional Inputs} 

In addition to the standard inputs, {\tt zelig()} takes the following
additional options for probit regression:  
\begin{itemize}
\item {\tt robust}: defaults to {\tt FALSE}.  If {\tt TRUE} is
selected, {\tt zelig()} computes robust standard errors via the {\tt
sandwich} package (see \cite{Zeileis04}).  The default type of robust
standard error is heteroskedastic and autocorrelation consistent (HAC),
and assumes that observations are ordered by time index.

In addition, {\tt robust} may be a list with the following options:  
\begin{itemize}
\item {\tt method}:  Choose from 
\begin{itemize}
\item {\tt "vcovHAC"}: (default if {\tt robust = TRUE}) HAC standard
errors. 
\item {\tt "kernHAC"}: HAC standard errors using the
weights given in \cite{Andrews91}. 
\item {\tt "weave"}: HAC standard errors using the
weights given in \cite{LumHea99}.  
\end{itemize}  
\item {\tt order.by}: defaults to {\tt NULL} (the observations are
chronologically ordered as in the original data).  Optionally, you may
specify a vector of weights (either as {\tt order.by = z}, where {\tt
z} exists outside the data frame; or as {\tt order.by = \~{}z}, where
{\tt z} is a variable in the data frame).  The observations are
chronologically ordered by the size of {\tt z}.
\item {\tt \dots}:  additional options passed to the functions 
specified in {\tt method}.   See the {\tt sandwich} library and
\cite{Zeileis04} for more options.   
\end{itemize}
\end{itemize}

\subsubsection{Examples}
Attach the sample turnout dataset:
\begin{Schunk}
\begin{Sinput}
>  data(turnout)
\end{Sinput}
\end{Schunk}
Estimate parameter values for the probit regression:
\begin{Schunk}
\begin{Sinput}
>  z.out <- zelig(vote ~ race + educate,  model = "probit", data = turnout) 
\end{Sinput}
\end{Schunk}
\begin{Schunk}
\begin{Sinput}
>  summary(z.out)
\end{Sinput}
\end{Schunk}
Set values for the explanatory variables to their default values.
\begin{Schunk}
\begin{Sinput}
>  x.out <- setx(z.out)
\documentclass{article}

\title{
  probit: Probit Regression for Dichotomous
  Dependent Variables
}
\author{Matt Owen, Olivia Lau, Kosuke Imai, and Gary King}




\usepackage{bibentry}
\usepackage{graphicx}
\usepackage{natbib}
\usepackage{amsmath}
\usepackage{url}
\usepackage{Zelig}
\usepackage{Sweave}

%\VignetteIndexEntry{Probit Regression for Dichotomous Dependent Variables}
%\VignetteDepends{Zelig, stats}
%\VignetteKeyWords{model,least squares,continuous, regression}
%\VignettePackage{Zelig}

\begin{document}

\nobibliography*



\section{{\tt probit}: Probit Regression for Dichotomous Dependent Variables}\label{probit}

Use probit regression to model binary dependent variables
specified as a function of a set of explanatory variables.  

\subsubsection{Syntax}
\begin{verbatim}
> z.out <- zelig(Y ~ X1 + X2, model = "probit", data = mydata)
> x.out <- setx(z.out)
> s.out <- sim(z.out, x = x.out, x1 = NULL)
\end{verbatim}

\subsubsection{Additional Inputs} 

In addition to the standard inputs, {\tt zelig()} takes the following
additional options for probit regression:  
\begin{itemize}
\item {\tt robust}: defaults to {\tt FALSE}.  If {\tt TRUE} is
selected, {\tt zelig()} computes robust standard errors via the {\tt
sandwich} package (see \cite{Zeileis04}).  The default type of robust
standard error is heteroskedastic and autocorrelation consistent (HAC),
and assumes that observations are ordered by time index.

In addition, {\tt robust} may be a list with the following options:  
\begin{itemize}
\item {\tt method}:  Choose from 
\begin{itemize}
\item {\tt "vcovHAC"}: (default if {\tt robust = TRUE}) HAC standard
errors. 
\item {\tt "kernHAC"}: HAC standard errors using the
weights given in \cite{Andrews91}. 
\item {\tt "weave"}: HAC standard errors using the
weights given in \cite{LumHea99}.  
\end{itemize}  
\item {\tt order.by}: defaults to {\tt NULL} (the observations are
chronologically ordered as in the original data).  Optionally, you may
specify a vector of weights (either as {\tt order.by = z}, where {\tt
z} exists outside the data frame; or as {\tt order.by = \~{}z}, where
{\tt z} is a variable in the data frame).  The observations are
chronologically ordered by the size of {\tt z}.
\item {\tt \dots}:  additional options passed to the functions 
specified in {\tt method}.   See the {\tt sandwich} library and
\cite{Zeileis04} for more options.   
\end{itemize}
\end{itemize}

\subsubsection{Examples}
Attach the sample turnout dataset:
\begin{Schunk}
\begin{Sinput}
>  data(turnout)
\end{Sinput}
\end{Schunk}
Estimate parameter values for the probit regression:
\begin{Schunk}
\begin{Sinput}
>  z.out <- zelig(vote ~ race + educate,  model = "probit", data = turnout) 
\end{Sinput}
\end{Schunk}
\begin{Schunk}
\begin{Sinput}
>  summary(z.out)
\end{Sinput}
\end{Schunk}
Set values for the explanatory variables to their default values.
\begin{Schunk}
\begin{Sinput}
>  x.out <- setx(z.out)
\documentclass{article}

\title{
  probit: Probit Regression for Dichotomous
  Dependent Variables
}
\author{Matt Owen, Olivia Lau, Kosuke Imai, and Gary King}




\usepackage{bibentry}
\usepackage{graphicx}
\usepackage{natbib}
\usepackage{amsmath}
\usepackage{url}
\usepackage{Zelig}
\usepackage{Sweave}

%\VignetteIndexEntry{Probit Regression for Dichotomous Dependent Variables}
%\VignetteDepends{Zelig, stats}
%\VignetteKeyWords{model,least squares,continuous, regression}
%\VignettePackage{Zelig}

\begin{document}

\nobibliography*



\section{{\tt probit}: Probit Regression for Dichotomous Dependent Variables}\label{probit}

Use probit regression to model binary dependent variables
specified as a function of a set of explanatory variables.  

\subsubsection{Syntax}
\begin{verbatim}
> z.out <- zelig(Y ~ X1 + X2, model = "probit", data = mydata)
> x.out <- setx(z.out)
> s.out <- sim(z.out, x = x.out, x1 = NULL)
\end{verbatim}

\subsubsection{Additional Inputs} 

In addition to the standard inputs, {\tt zelig()} takes the following
additional options for probit regression:  
\begin{itemize}
\item {\tt robust}: defaults to {\tt FALSE}.  If {\tt TRUE} is
selected, {\tt zelig()} computes robust standard errors via the {\tt
sandwich} package (see \cite{Zeileis04}).  The default type of robust
standard error is heteroskedastic and autocorrelation consistent (HAC),
and assumes that observations are ordered by time index.

In addition, {\tt robust} may be a list with the following options:  
\begin{itemize}
\item {\tt method}:  Choose from 
\begin{itemize}
\item {\tt "vcovHAC"}: (default if {\tt robust = TRUE}) HAC standard
errors. 
\item {\tt "kernHAC"}: HAC standard errors using the
weights given in \cite{Andrews91}. 
\item {\tt "weave"}: HAC standard errors using the
weights given in \cite{LumHea99}.  
\end{itemize}  
\item {\tt order.by}: defaults to {\tt NULL} (the observations are
chronologically ordered as in the original data).  Optionally, you may
specify a vector of weights (either as {\tt order.by = z}, where {\tt
z} exists outside the data frame; or as {\tt order.by = \~{}z}, where
{\tt z} is a variable in the data frame).  The observations are
chronologically ordered by the size of {\tt z}.
\item {\tt \dots}:  additional options passed to the functions 
specified in {\tt method}.   See the {\tt sandwich} library and
\cite{Zeileis04} for more options.   
\end{itemize}
\end{itemize}

\subsubsection{Examples}
Attach the sample turnout dataset:
\begin{Schunk}
\begin{Sinput}
>  data(turnout)
\end{Sinput}
\end{Schunk}
Estimate parameter values for the probit regression:
\begin{Schunk}
\begin{Sinput}
>  z.out <- zelig(vote ~ race + educate,  model = "probit", data = turnout) 
\end{Sinput}
\end{Schunk}
\begin{Schunk}
\begin{Sinput}
>  summary(z.out)
\end{Sinput}
\end{Schunk}
Set values for the explanatory variables to their default values.
\begin{Schunk}
\begin{Sinput}
>  x.out <- setx(z.out)
\documentclass{article}

\title{
  probit: Probit Regression for Dichotomous
  Dependent Variables
}
\author{Matt Owen, Olivia Lau, Kosuke Imai, and Gary King}




\usepackage{bibentry}
\usepackage{graphicx}
\usepackage{natbib}
\usepackage{amsmath}
\usepackage{url}
\usepackage{Zelig}
\usepackage{Sweave}

%\VignetteIndexEntry{Probit Regression for Dichotomous Dependent Variables}
%\VignetteDepends{Zelig, stats}
%\VignetteKeyWords{model,least squares,continuous, regression}
%\VignettePackage{Zelig}

\begin{document}

\nobibliography*



\section{{\tt probit}: Probit Regression for Dichotomous Dependent Variables}\label{probit}

Use probit regression to model binary dependent variables
specified as a function of a set of explanatory variables.  

\subsubsection{Syntax}
\begin{verbatim}
> z.out <- zelig(Y ~ X1 + X2, model = "probit", data = mydata)
> x.out <- setx(z.out)
> s.out <- sim(z.out, x = x.out, x1 = NULL)
\end{verbatim}

\subsubsection{Additional Inputs} 

In addition to the standard inputs, {\tt zelig()} takes the following
additional options for probit regression:  
\begin{itemize}
\item {\tt robust}: defaults to {\tt FALSE}.  If {\tt TRUE} is
selected, {\tt zelig()} computes robust standard errors via the {\tt
sandwich} package (see \cite{Zeileis04}).  The default type of robust
standard error is heteroskedastic and autocorrelation consistent (HAC),
and assumes that observations are ordered by time index.

In addition, {\tt robust} may be a list with the following options:  
\begin{itemize}
\item {\tt method}:  Choose from 
\begin{itemize}
\item {\tt "vcovHAC"}: (default if {\tt robust = TRUE}) HAC standard
errors. 
\item {\tt "kernHAC"}: HAC standard errors using the
weights given in \cite{Andrews91}. 
\item {\tt "weave"}: HAC standard errors using the
weights given in \cite{LumHea99}.  
\end{itemize}  
\item {\tt order.by}: defaults to {\tt NULL} (the observations are
chronologically ordered as in the original data).  Optionally, you may
specify a vector of weights (either as {\tt order.by = z}, where {\tt
z} exists outside the data frame; or as {\tt order.by = \~{}z}, where
{\tt z} is a variable in the data frame).  The observations are
chronologically ordered by the size of {\tt z}.
\item {\tt \dots}:  additional options passed to the functions 
specified in {\tt method}.   See the {\tt sandwich} library and
\cite{Zeileis04} for more options.   
\end{itemize}
\end{itemize}

\subsubsection{Examples}
Attach the sample turnout dataset:
\begin{Schunk}
\begin{Sinput}
>  data(turnout)
\end{Sinput}
\end{Schunk}
Estimate parameter values for the probit regression:
\begin{Schunk}
\begin{Sinput}
>  z.out <- zelig(vote ~ race + educate,  model = "probit", data = turnout) 
\end{Sinput}
\end{Schunk}
\begin{Schunk}
\begin{Sinput}
>  summary(z.out)
\end{Sinput}
\end{Schunk}
Set values for the explanatory variables to their default values.
\begin{Schunk}
\begin{Sinput}
>  x.out <- setx(z.out)
\documentclass{article}

\title{
  probit: Probit Regression for Dichotomous
  Dependent Variables
}
\author{Matt Owen, Olivia Lau, Kosuke Imai, and Gary King}




\usepackage{bibentry}
\usepackage{graphicx}
\usepackage{natbib}
\usepackage{amsmath}
\usepackage{url}
\usepackage{Zelig}
\usepackage{Sweave}

%\VignetteIndexEntry{Probit Regression for Dichotomous Dependent Variables}
%\VignetteDepends{Zelig, stats}
%\VignetteKeyWords{model,least squares,continuous, regression}
%\VignettePackage{Zelig}

\begin{document}

\nobibliography*



\section{{\tt probit}: Probit Regression for Dichotomous Dependent Variables}\label{probit}

Use probit regression to model binary dependent variables
specified as a function of a set of explanatory variables.  

\subsubsection{Syntax}
\begin{verbatim}
> z.out <- zelig(Y ~ X1 + X2, model = "probit", data = mydata)
> x.out <- setx(z.out)
> s.out <- sim(z.out, x = x.out, x1 = NULL)
\end{verbatim}

\subsubsection{Additional Inputs} 

In addition to the standard inputs, {\tt zelig()} takes the following
additional options for probit regression:  
\begin{itemize}
\item {\tt robust}: defaults to {\tt FALSE}.  If {\tt TRUE} is
selected, {\tt zelig()} computes robust standard errors via the {\tt
sandwich} package (see \cite{Zeileis04}).  The default type of robust
standard error is heteroskedastic and autocorrelation consistent (HAC),
and assumes that observations are ordered by time index.

In addition, {\tt robust} may be a list with the following options:  
\begin{itemize}
\item {\tt method}:  Choose from 
\begin{itemize}
\item {\tt "vcovHAC"}: (default if {\tt robust = TRUE}) HAC standard
errors. 
\item {\tt "kernHAC"}: HAC standard errors using the
weights given in \cite{Andrews91}. 
\item {\tt "weave"}: HAC standard errors using the
weights given in \cite{LumHea99}.  
\end{itemize}  
\item {\tt order.by}: defaults to {\tt NULL} (the observations are
chronologically ordered as in the original data).  Optionally, you may
specify a vector of weights (either as {\tt order.by = z}, where {\tt
z} exists outside the data frame; or as {\tt order.by = \~{}z}, where
{\tt z} is a variable in the data frame).  The observations are
chronologically ordered by the size of {\tt z}.
\item {\tt \dots}:  additional options passed to the functions 
specified in {\tt method}.   See the {\tt sandwich} library and
\cite{Zeileis04} for more options.   
\end{itemize}
\end{itemize}

\subsubsection{Examples}
Attach the sample turnout dataset:
\begin{Schunk}
\begin{Sinput}
>  data(turnout)
\end{Sinput}
\end{Schunk}
Estimate parameter values for the probit regression:
\begin{Schunk}
\begin{Sinput}
>  z.out <- zelig(vote ~ race + educate,  model = "probit", data = turnout) 
\end{Sinput}
\end{Schunk}
\begin{Schunk}
\begin{Sinput}
>  summary(z.out)
\end{Sinput}
\end{Schunk}
Set values for the explanatory variables to their default values.
\begin{Schunk}
\begin{Sinput}
>  x.out <- setx(z.out)
\include{relogit}
\include{sur}
\include{threesls}
\include{tobit}
\include{tobit.bayes}
\include{twosls}

\documentclass[oneside,letterpaper,12pt]{book}
\usepackage{Rd}
%\usepackage{Sweave}
%\usepackage{/usr/lib64/R/share/texmf/Sweave}
%\usepackage{/usr/share/R/texmf/Sweave}
\usepackage{bibentry}
\usepackage{upquote}
\usepackage{graphicx}
\usepackage{natbib}
\usepackage[reqno]{amsmath}
\usepackage{amssymb}
\usepackage{amsfonts}
\usepackage{amsmath}
\usepackage{verbatim}
\usepackage{epsf}
\usepackage{url}
\usepackage{html}
\usepackage{dcolumn}
\usepackage{multirow}
\usepackage{fullpage}
\usepackage{lscape}
\usepackage[all]{xy}

\usepackage{csquotes}
% \usepackage[pdftex, bookmarksopen=true,bookmarksnumbered=true,
%   linkcolor=webred]{hyperref}
\bibpunct{(}{)}{;}{a}{}{,}
\newcolumntype{.}{D{.}{.}{-1}}
\newcolumntype{d}[1]{D{.}{.}{#1}}
\htmladdtonavigation{
  \htmladdnormallink{%
    \htmladdimg{http://gking.harvard.edu/pics/home.gif}}
  {http://gking.harvard.edu/}}
\newcommand{\MatchIt}{{\sc MatchIt}}
\newcommand{\hlink}{\htmladdnormallink}
\newcommand{\Sref}[1]{Section~\ref{#1}}
\newcommand{\fullrvers}{2.5.1}
\newcommand{\rvers}{2.5}
\newcommand{\rwvers}{R-2.5.1}
%\renewcommand{\bibentry}{\citealt}

\bodytext{ BACKGROUND="http://gking.harvard.edu/pics/temple.jpg"}
\setcounter{tocdepth}{2}

%\VignetteIndexEntry{Weibull Regression for Duration Dependent Variables}
%\VignetteDepends{Zelig, survival}
%\VignetteKeyWords{model, weibull,regression,bounded, duration}
%\VignettePackage{Zelig}
\usepackage{Sweave}
\begin{document}
\nobibliography*


\section{{\tt weibull}: Weibull Regression for Duration
Dependent Variables}\label{weibull}

Choose the Weibull regression model if the values in your dependent
variable are duration observations.  The Weibull model relaxes the
exponential model's (see \Sref{exp}) assumption of constant hazard,
and allows the hazard rate to increase or decrease monotonically with
respect to elapsed time.

\subsubsection{Syntax}

\begin{verbatim}
> z.out <- zelig(Surv(Y, C) ~ X1 + X2, model = "weibull", data = mydata)
> x.out <- setx(z.out)
> s.out <- sim(z.out, x = x.out)
\end{verbatim}
Weibull models require that the dependent variable be in the form {\tt
  Surv(Y, C)}, where {\tt Y} and {\tt C} are vectors of length $n$.
For each observation $i$ in 1, \dots, $n$, the value $y_i$ is the
duration (lifetime, for example), and the associated $c_i$ is a binary
variable such that $c_i = 1$ if the duration is not censored ({\it
  e.g.}, the subject dies during the study) or $c_i = 0$ if the
duration is censored ({\it e.g.}, the subject is still alive at the
end of the study).  If $c_i$ is omitted, all Y are assumed to be
completed; that is, time defaults to 1 for all observations.

\subsubsection{Input Values} 

In addition to the standard inputs, {\tt zelig()} takes the following
additional options for weibull regression:  
\begin{itemize}
\item {\tt robust}: defaults to {\tt FALSE}.  If {\tt TRUE}, {\tt
zelig()} computes robust standard errors based on sandwich estimators
(see \cite{Huber81} and \cite{White80}) based on the options in {\tt
cluster}.
\item {\tt cluster}:  if {\tt robust = TRUE}, you may select a
variable to define groups of correlated observations.  Let {\tt x3} be
a variable that consists of either discrete numeric values, character
strings, or factors that define strata.  Then
\begin{verbatim}
> z.out <- zelig(y ~ x1 + x2, robust = TRUE, cluster = "x3", 
                 model = "exp", data = mydata)
\end{verbatim}
means that the observations can be correlated within the strata defined by
the variable {\tt x3}, and that robust standard errors should be
calculated according to those clusters.  If {\tt robust = TRUE} but
{\tt cluster} is not specified, {\tt zelig()} assumes that each
observation falls into its own cluster.  
\end{itemize}  

\subsubsection{Example}

Attach the sample data: 
\begin{Schunk}
\begin{Sinput}
> data(coalition)
\end{Sinput}
\end{Schunk}
Estimate the model: 
\begin{Schunk}
\begin{Sinput}
> z.out <- zelig(Surv(duration, ciep12) ~ fract + numst2, model = "weibull", 
+     data = coalition)
\end{Sinput}
\end{Schunk}
View the regression output:  
\begin{Schunk}
\begin{Sinput}
> summary(z.out)
\end{Sinput}
\end{Schunk}
Set the baseline values (with the ruling coalition in the minority)
and the alternative values (with the ruling coalition in the majority)
for X:
\begin{Schunk}
\begin{Sinput}
> x.low <- setx(z.out, numst2 = 0)
> x.high <- setx(z.out, numst2 = 1)
\end{Sinput}
\end{Schunk}
Simulate expected values ({\tt qi\$ev}) and first differences ({\tt qi\$fd}):
\begin{Schunk}
\begin{Sinput}
> s.out <- sim(z.out, x = x.low, x1 = x.high)


\chapter{Commands for Programmers and Contributors}

\section{{\tt describe}: Describe a model's systematic and stochastic 
parameters}
\label{describe.mymodel}

\subsubsection{Description}

In order to use {\tt parse.formula()}, {\tt parse.par()}, and the {\tt
model.*.multiple()} commands, you must write a {\tt describe.mymodel()}
function where {\tt mymodel} is the name of your modeling function.
(Hence, if your function is called {\tt normal.regression()}, you need
to write a {\tt describe.normal.regression()} function.)  Note that
{\tt describe()} is \emph{not} a generic function, but is called by
{\tt parse.formula(\dots, model = "mymodel")} using a combination of
{\tt paste()} and {\tt exists()}.  You will never need to call {\tt
describe.mymodel()} directly, since it will be called from {\tt
parse.formula()} as that function checks the user-input formula or
list of formulas.  

\subsubsection{Syntax}
\begin{verbatim}
describe.mymodel()
\end{verbatim}

\subsubsection{Arguments}\label{categories}
The {\tt describe.mymodel()} function takes no arguments.  

\subsubsection{Output Values}
The {\tt describe.mymodel()} function returns a list with the
following information:  
\begin{itemize}
\item {\tt category}: a character string, consisting of one of the
following: 
\begin{itemize}
\item {\tt "continuous"}: the dependent variable is continuous, numeric, and
unbounded (e.g., normal regression), but may be censored with an associated censoring 
indicator (e.g., tobit regression).  
\item {\tt "dichotomous"}: the dependent variable takes two discrete integer
values, usually 0 and 1 (e.g., logistic regression).  
\item {\tt "ordinal"}: the dependent variable is an ordered factor
response, taking 3 or more discrete values which are arranged in an
ascending or descending manner (e.g., ordered logistic regression).  
\item {\tt "multinomial"}: the dependent variable is an unordered
factor response, taking 3 or more discrete values which are arranged
in no particular order (e.g., multinomial logistic regression).  
\item {\tt "count"}: the dependent variable takes integer values
greater than or equal to 0 (e.g., Poisson regression).  
\item {\tt "bounded"}: the dependent variable is a continuous numeric variable, that 
is restricted (bounded within) some range (e.g., $(0, \infty)$).  The variable may 
also be censored either on the left and/or right side, with an associated censoring 
indicator (e.g., Weibull regression).
\item {\tt "mixed"}: the dependent variables are a mix of the above
categories (usually applies to multiple equation models).  
\end{itemize}
Selecting the category is particularly important since it sets certain
interface parameters for the entire GUI.

\item {\tt package}: (optional) a list with the following elements

  \begin{itemize} 

   \item {\tt name}: a characters string with the name of the package
   containing the {\tt mymodel()} function.

   \item {\tt version}: the minimum version number that works with
   Zelig.

   \item {\tt CRAN}: if the package is not hosted on CRAN mirrors,
   provide the URL here as a character string.  You should be able
   to install your package from this URL using {\tt name}, {\tt
version}, and {\tt CRAN}:
\begin{verbatim}
install.packages(name, repos = CRAN, installWithVers = TRUE)
\end{verbatim}  
By default, {\tt CRAN = "http://cran.us.r-project.org/"}.  
\end{itemize}

\item {\tt parameters}: For each systematic and stochastic parameter
(or set of parameters) in your model, you should create a list (named
after the parameters as given in your model's notation, e.g., {\tt
mu}, {\tt sigma}, {\tt theta}, etc.; not literally {\tt myparameter})
with the following information:
\begin{itemize}

\item {\tt equations}: an integer number of equations for the
parameter.  For parameters that can take an undefined number of
equations (for example in seemingly unrelated regression), use {\tt
c(2, Inf)} or {\tt c(2, 999)} to indicate that the parameter can take
a minimum of two equations up to a theoretically infinite number of
equations.  

\item {\tt tagsAllowed}: a logical value ({\tt TRUE}/{\tt FALSE})
specifying whether a given parameter allows constraints.  If there is
only one equation for a parameter (for example, {\tt mu} for the
normal regression model has {\tt equations = 1}), then {\tt
tagsAllowed = FALSE} by default.  If there are two or more equations
for the parameter (for example, {\tt mu} for the bivariate probit
model has {\tt equations = 2}), then {\tt tagsAllowed = TRUE} by
default.  

\item {\tt depVar}: a logical value ({\tt TRUE}/{\tt FALSE})
specifying whether a parameter requires a corresponding dependent
variable.  

\item {\tt expVar}: a logical value ({\tt TRUE}/{\tt FALSE})
specifying whether a parameter allows explanatory variables.  If {\tt
depVar = TRUE} and {\tt expVar = TRUE}, we call the parameter a
``systematic component'' and {\tt parse.formula()} will fail if
formula(s) are not specified for this parameter.  If {\tt
depVar = FALSE} and {\tt expVar = TRUE}, the parameter is estimated as
a scalar ancillary parameter, with default formula \verb|~ 1|, if the
user does not specify a formula explicitly.  If {\tt depVar = FALSE}
and {\tt expVar = FALSE}, the parameter can only be estimated as a
scalar ancillary parameter.  

\item {\tt specialFunction}: (optional) a character string giving the
name of a function that appears on the left-hand side of the formula.
Options include {\tt "Surv"}, {\tt "cbind"}, and {\tt "as.factor"}. 

\item {\tt varInSpecial}: (optional) a scalar or vector giving the
number of variables taken by the {\tt specialFunction}.  For example,
{\tt Surv()} takes a minimum of 2 arguments, and a maximum of 4
arguments, which is represented as {\tt c(2, 4)}.   

\end{itemize}
If you have more than one parameter (or set of parameters) in
your model, you will need to produce a {\tt myparameter} list for each
one.  See examples below for details.  
\end{itemize}

\subsubsection{Examples}
For a Normal regression model with mean {\tt mu} and scalar variance
parameter {\tt sigma2}, the minimal {\tt describe.*()} function is as
follows:  
\begin{verbatim}
describe.normal.regression <- function() {
  category <- "continuous"
  mu <- list(equations = 1,              # Systematic component
             tagsAllowed = FALSE, 
             depVar = TRUE, 
             expVar = TRUE)
  sigma2 <- list(equations = 1,          # Scalar ancillary parameter
                 tagsAllowed = FALSE, 
                 depVar = FALSE, 
                 expVar = FALSE)
  pars <- list(mu = mu, sigma2 = sigma2)
  model <- list(category = category, parameters = pars)
}
\end{verbatim}
See \Sref{normal.regression} for full code to execute this model from
scratch in R with Zelig.  

Now consider a bivariate probit model with parameter vector {\tt mu} and
correlation parameter {\tt rho} (which may or may not take explanatory
variables).  Since the bivariate probit function uses the {\tt pmvnorm()} 
function from the mvtnorm library, we list this under {\tt package}.   
\begin{verbatim}
describe.bivariate.probit <- function() {
  category <- "dichotomous"
  package <- list(name = "mvtnorm", 
                  version = "0.7")
  mu <- list(equations = 2,               # Systematic component 
             tagsAllowed = TRUE,          
             depVar = TRUE, 
             expVar = TRUE) 
  rho <- list(equations = 1,              # Optional systematic component
             tagsAllowed = FALSE,         #   Estimated as an ancillary
             depVar = FALSE,              #   parameter by default
             expVar = TRUE) 
  pars <- list(mu = mu, rho = rho)
  list(category = category, package = package, parameters = pars)
}
\end{verbatim}  
See \Sref{bivariate.probit} for the full code to write this model from
scratch in R with Zelig. 

For a multinomial logit model, which takes an undefined number of
equations (corresponding to each level in the response variable):  
\begin{verbatim}
describe.multinomial.logit <- function() { 
  category <- "multinomial"
  mu <- list(equations = c(1, Inf), 
             tagsAllowed = TRUE, 
             depVAR = TRUE, 
             expVar = TRUE, 
             specialFunction <- "as.factor", 
             varInSpecial <- c(1, 1))
  list(category = category, parameters = list(mu = mu))
}
\end{verbatim}
(This example does not have corresponding executable sample code.)

\subsubsection{See Also}
\begin{itemize}
\item \Sref{s:new} for an overview of how the {\tt describe.*()}
function works with {\tt parse.formula()}.  
\item \Sref{parse.formula} for information on {\tt parse.formula()}.
\end{itemize}

\subsubsection{Contributors}

Kosuke Imai, Gary King, Olivia Lau, and Ferdinand Alimadhi.

%%% Local Variables: 
%%% mode: latex
%%% TeX-master: "~/zelig/docs/zelig"
%%% End: 

\include{commands/model.end}
\include{commands/model.frame.multiple}
\include{commands/model.matrix.multiple}
\documentclass{article}

\title{Specification for {\tt parse.formula}}
\author{Matt Owen}

\begin{document}

\maketitle


\section{Introduction}

\section{Types of Formula}

There are three basic ways to specify formulae in Zelig.

\begin{enumerate}

  \item A single formula with a single outcome term and one or more 
    response terms. For example, 
    \begin{itemize}
      \item {\tt y \~{} 1}
      \item {\tt y \~{} x1 + x2 + x3}
    \end{itemize}

  \item A single formula with a multiple outcome terms specified with 
    {\tt cbind} or {\tt list} and one or more response terms. For example,
    \begin{itemize}
      \item {\tt cbind(y1, y2) \~{} x1 + x2}
      \item {\tt list(y1, y2) \~{} x1 + x2}
    \end{itemize}

  \item A list of formulas of the first type - single outcome terms with one or
    more response terms. For example, 
    \begin{itemize}
      \item \begin{verbatim}
list(
     y1 ~ x1,
     y2 ~ x1 + x2
     )
        \end{verbatim}

      \item \begin{verbatim}
list(
     mu1 = y1 ~ x1,
     mu2 = y2 ~ 1
     )
        \end{verbatim}
    \end{itemize}

\end{enumerate}



% How to parse a formula
\section{{\tt parse.formula}}



% How a model matrix should be constructed
\section{{\tt model.matrix}}

This should be given a Zelig-style formula. As output, it will produce a valid
model matrix. Care should be taken that matrices of simulations have the same
column order of this model.matrix. Otherwise, simulations will produce invalid 
results.


\end{document}

\include{commands/parse.par}
\include{commands/put.start}
\include{commands/set.start}
\section{{\tt tag}: Constrain parameter effects across equations}
\label{tag}

\subsubsection{Description}
Use {\tt tag()} to identify parameters and constrain their effects
across equations in multiple-equation models.  
  
\subsubsection{Syntax}
\begin{verbatim}
tag(x, label)
\end{verbatim}

\subsubsection{Arguments}
\begin{itemize}
\item {\tt x}: the variable to be constrained.
\item {\tt label}: the name that the constrained variable takes.  
\end{itemize}

\subsubsection{Output Values}
While there is no specific output from {\tt tag()} itself, {\tt
parse.formula()} uses {\tt tag()} to identify parameter constraints
across equations, when a model takes more than one systematic
component.  

\subsubsection{Examples}

\subsubsection{See Also}
\begin{itemize}
\item \Sref{ui} for an overview of the multiple-equation user-interface.
\item \Sref{parse.formula} for more examples of acceptable uses for
{\tt tag()} in formulas.  
\end{itemize}

\subsubsection{Contributors}

Kosuke Imai, Gary King, Olivia Lau, and Ferdinand Alimadhi.


%%% Local Variables: 
%%% mode: latex
%%% TeX-master: t
%%% End: 















%%% Local Variables: 
%%% mode: latex
%%% TeX-master: "zelig"
%%% End: 
