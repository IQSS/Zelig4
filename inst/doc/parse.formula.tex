\documentclass{article}

\title{Specification for {\tt parse.formula}}
\author{Matt Owen}

\begin{document}

\maketitle


\section{Introduction}

\section{Types of Formula}

There are three basic ways to specify formulae in Zelig.

\begin{enumerate}

  \item A single formula with a single outcome term and one or more 
    response terms. For example, 
    \begin{itemize}
      \item {\tt y \~{} 1}
      \item {\tt y \~{} x1 + x2 + x3}
    \end{itemize}

  \item A single formula with a multiple outcome terms specified with 
    {\tt cbind} or {\tt list} and one or more response terms. For example,
    \begin{itemize}
      \item {\tt cbind(y1, y2) \~{} x1 + x2}
      \item {\tt list(y1, y2) \~{} x1 + x2}
    \end{itemize}

  \item A list of formulas of the first type - single outcome terms with one or
    more response terms. For example, 
    \begin{itemize}
      \item \begin{verbatim}
list(
     y1 ~ x1,
     y2 ~ x1 + x2
     )
        \end{verbatim}

      \item \begin{verbatim}
list(
     mu1 = y1 ~ x1,
     mu2 = y2 ~ 1
     )
        \end{verbatim}
    \end{itemize}

\end{enumerate}



% How to parse a formula
\section{{\tt parse.formula}}



% How a model matrix should be constructed
\section{{\tt model.matrix}}

This should be given a Zelig-style formula. As output, it will produce a valid
model matrix. Care should be taken that matrices of simulations have the same
column order of this model.matrix. Otherwise, simulations will produce invalid 
results.


\end{document}
