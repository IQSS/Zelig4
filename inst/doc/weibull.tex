
\documentclass[oneside,letterpaper,12pt]{book}
\usepackage{Rd}
%\usepackage{Sweave}
%\usepackage{/usr/lib64/R/share/texmf/Sweave}
%\usepackage{/usr/share/R/texmf/Sweave}
\usepackage{bibentry}
\usepackage{upquote}
\usepackage{graphicx}
\usepackage{natbib}
\usepackage[reqno]{amsmath}
\usepackage{amssymb}
\usepackage{amsfonts}
\usepackage{amsmath}
\usepackage{verbatim}
\usepackage{epsf}
\usepackage{url}
\usepackage{html}
\usepackage{dcolumn}
\usepackage{multirow}
\usepackage{fullpage}
\usepackage{lscape}
\usepackage[all]{xy}

\usepackage{csquotes}
% \usepackage[pdftex, bookmarksopen=true,bookmarksnumbered=true,
%   linkcolor=webred]{hyperref}
\bibpunct{(}{)}{;}{a}{}{,}
\newcolumntype{.}{D{.}{.}{-1}}
\newcolumntype{d}[1]{D{.}{.}{#1}}
\htmladdtonavigation{
  \htmladdnormallink{%
    \htmladdimg{http://gking.harvard.edu/pics/home.gif}}
  {http://gking.harvard.edu/}}
\newcommand{\MatchIt}{{\sc MatchIt}}
\newcommand{\hlink}{\htmladdnormallink}
\newcommand{\Sref}[1]{Section~\ref{#1}}
\newcommand{\fullrvers}{2.5.1}
\newcommand{\rvers}{2.5}
\newcommand{\rwvers}{R-2.5.1}
%\renewcommand{\bibentry}{\citealt}

\bodytext{ BACKGROUND="http://gking.harvard.edu/pics/temple.jpg"}
\setcounter{tocdepth}{2}

%\VignetteIndexEntry{Weibull Regression for Duration Dependent Variables}
%\VignetteDepends{Zelig, survival}
%\VignetteKeyWords{model, weibull,regression,bounded, duration}
%\VignettePackage{Zelig}
\usepackage{Sweave}
\begin{document}
\nobibliography*


\section{{\tt weibull}: Weibull Regression for Duration
Dependent Variables}\label{weibull}

Choose the Weibull regression model if the values in your dependent
variable are duration observations.  The Weibull model relaxes the
exponential model's (see \Sref{exp}) assumption of constant hazard,
and allows the hazard rate to increase or decrease monotonically with
respect to elapsed time.

\subsubsection{Syntax}

\begin{verbatim}
> z.out <- zelig(Surv(Y, C) ~ X1 + X2, model = "weibull", data = mydata)
> x.out <- setx(z.out)
> s.out <- sim(z.out, x = x.out)
\end{verbatim}
Weibull models require that the dependent variable be in the form {\tt
  Surv(Y, C)}, where {\tt Y} and {\tt C} are vectors of length $n$.
For each observation $i$ in 1, \dots, $n$, the value $y_i$ is the
duration (lifetime, for example), and the associated $c_i$ is a binary
variable such that $c_i = 1$ if the duration is not censored ({\it
  e.g.}, the subject dies during the study) or $c_i = 0$ if the
duration is censored ({\it e.g.}, the subject is still alive at the
end of the study).  If $c_i$ is omitted, all Y are assumed to be
completed; that is, time defaults to 1 for all observations.

\subsubsection{Input Values} 

In addition to the standard inputs, {\tt zelig()} takes the following
additional options for weibull regression:  
\begin{itemize}
\item {\tt robust}: defaults to {\tt FALSE}.  If {\tt TRUE}, {\tt
zelig()} computes robust standard errors based on sandwich estimators
(see \cite{Huber81} and \cite{White80}) based on the options in {\tt
cluster}.
\item {\tt cluster}:  if {\tt robust = TRUE}, you may select a
variable to define groups of correlated observations.  Let {\tt x3} be
a variable that consists of either discrete numeric values, character
strings, or factors that define strata.  Then
\begin{verbatim}
> z.out <- zelig(y ~ x1 + x2, robust = TRUE, cluster = "x3", 
                 model = "exp", data = mydata)
\end{verbatim}
means that the observations can be correlated within the strata defined by
the variable {\tt x3}, and that robust standard errors should be
calculated according to those clusters.  If {\tt robust = TRUE} but
{\tt cluster} is not specified, {\tt zelig()} assumes that each
observation falls into its own cluster.  
\end{itemize}  

\subsubsection{Example}

Attach the sample data: 
\begin{Schunk}
\begin{Sinput}
> data(coalition)
\end{Sinput}
\end{Schunk}
Estimate the model: 
\begin{Schunk}
\begin{Sinput}
> z.out <- zelig(Surv(duration, ciep12) ~ fract + numst2, model = "weibull", 
+     data = coalition)
\end{Sinput}
\end{Schunk}
View the regression output:  
\begin{Schunk}
\begin{Sinput}
> summary(z.out)
\end{Sinput}
\end{Schunk}
Set the baseline values (with the ruling coalition in the minority)
and the alternative values (with the ruling coalition in the majority)
for X:
\begin{Schunk}
\begin{Sinput}
> x.low <- setx(z.out, numst2 = 0)
> x.high <- setx(z.out, numst2 = 1)
\end{Sinput}
\end{Schunk}
Simulate expected values ({\tt qi\$ev}) and first differences ({\tt qi\$fd}):
\begin{Schunk}
\begin{Sinput}
> s.out <- sim(z.out, x = x.low, x1 = x.high)