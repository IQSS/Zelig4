\documentclass[10pt]{article}

\begin{document}

\title{Writing a \emph{qi} function}
\author{Matthew Owen}
\maketitle

\section{Introduction}
% Introduction Material
For any Zelig module, the \emph{qi} function is ultimately the most important piece of code that must be written; it describes the actual process which simulates the \emph{quantities of interest}.  Because of the nature of this process - and the gamut of statistical packages and their underlying statistical model - it is rare that the simulation process can be generalized for arbitrary fitted models.  Despite this, it is possible to break down the simulation process into smaller steps.


%
%
\section{Notable Features of \emph{qi} Function}
The typical \emph{qi} function has several basic procedures:
\begin{enumerate}
	\item{\emph{Call the param function}:  This is entirely optional but sometimes important for the clarity of your algorithm.  This step typically consists of taking random draws from the fitted model's underlying probability distribution.}
	\item{\emph{Compute the Quantity of Interest}: Depending on your model, there are several ways to compute necessary quantities of interest.  Typical methods for computing quantities of interest include:
		\begin{enumerate}
			\item{Using the `predict' method of your given linear model.}
			\item{Using the sample provided by `param' to generate simulations of the \emph{Quantities of Interest}.}
			\item{Using a Maximum-likelihood estimate on the fitted model.}
		\end{enumerate}}
	\item{\emph{Create a list of titles for your Quantities of Interest}:}
	\item{\emph{Generate the Quantity of Interest Object}: Finally, with the computed Quantities of Interest, you must }
\end{enumerate}


%
\section{Basic Layout of a \emph{qi} Function}
Now with the general outline of a \emph{qi} function defined, it is important to discuss the expected procedures and specifics of implementation.


\subsection{The Function's Signature}
% quick intro
The \emph{qi} function's signature accepts 4 parameters:


%
%
\begin{description}
	\item[@z:]{An object of type ``\emph{zelig}''.  This wraps the fitted model in the slot ``result''.}
	\item[@x:]{An object of type ``\emph{setx}''.  This object is used to compute important coefficients, parameters, and features of the data.frame passed to the function call.}
	\item[@x1:]{Also an object of type ``\emph{setx}''.  This object is used in a similar fashion, however its presence allows a variety of \emph{quantities of interest} to be computed.  Notably, this is a necessary parameter to compute first-differences.}
	\item[@num:]{The number of simulations to compute}
\end{description}


% code example
%
\subsection{Code Example: \emph{qi} Function Signature}
\begin{verbatim}
qi.model <- function(z, x=NULL, x1=NULL, num=1000) {
	# start typing your code here
	# ...
	# ...
\end{verbatim}


Note: In the above example, the function name ``qi.\_model'' is merely a placeholder.  In order to register a \emph{qi} function with zelig, the developer must follow the naming convention qi.\emph{your mode name}, where \emph{your\_model\_name} is the name of the developer's module.  For example, if a developer titled his or her zelig module ``logit'', then the corresponding \emph{qi} function is titled ``\emph{qi.logit}''.

\subsection{Call to the \emph{param} Function}
This step is common in many zelig models, however, its existence - though highly recommended - is purely optional.  Typically, during this step, samples are taken from the distribution governing the statistical model.  This is then used to simulate values for the \emph{quantities of interest}.

%
\subsection{The Function Body}
The function body of \emph{qi} function varies largely from model to model.  As a result, it is impossible to create general guidelines to simulate \emph{quantities of interest} - or even determine what the \emph{quantity of interest} is.  Typical methods for computing \emph{quantities of interest} include:
\begin{itemize}
	\item{Implementing sampling algorithms based on the underlying fitted model, or}
	\item{``Predicting'' a large number of values from the fitted model}
\end{itemize}


% return values
\subsection{The Return Value}
In order for Zelig to process the simulations, they must be returned in one of several formats:

\begin{itemize}
	% First Example
	\item{\begin{verbatim}
		list(
		     "TITLE OF QI 1" = val1,
		     "TITLE OF QI 2" = val2,
		     # any number of title-val pairs
		     # ...
		     "TITLE OF QI N" = val.n
		     )
	\end{verbatim}}

	% Second Example
	\item{\begin{verbatim}
		make.qi(
		        titles = list(title1, title2),
		        stats  = list(val1, val2)
		        )
	\end{verbatim}}
\end{itemize}


In the above example,\emph{val1, val2}are data.frames, matrices, or lists representing the simulations of the \emph{quantities of interests}, and \emph{title1, title2} - and any number of titles - are character-strings that will act as human-readable descriptions of the \emph{quantities of interest}.  Once results are returned in this format, Zelig will convert the results into a machine-readable format and summarize the simulations into a comprehensible format.

NOTE: Because of its readability, it is suggested that the first method is used when returning \emph{quantities of interest}.

% break
\pagebreak


% find better way to output this data
\section{Example \emph{qi} function (qi.logit.R)}


\begin{verbatim}
qi.ls <- function(z, x=NULL, x1=NULL, num=1000) {
  # error-catching
  if (missing(x))
    stop("x cannot be missing while computing the `ls' model")


  # get `parameters'
  # In this example, this amounts to sampling
  # a multivariate normal distribution
  coefs <- param(z, num=num)


  # compute expected values using X
  ev <- coefs %*% t(x$matrix)
  ev1 <- NA
  fd <- NA
  
  # if x1 exists:
  #   compute expected values using X1
  #   compute finite differences
  if (!is.null(x1)) {
    ev1 <- coefs %*% t(x1$matrix)
    fd <- ev1 - ev
  }

  # return
  list("Expected Value: E(Y|X)" = ev,
       "Expected Value (of X1): E(Y|X1)" = ev1,
       "First Difference in Expected Values: E(Y|X1) - E(Y|X)" = fd
       )
}
\end{verbatim}

\pagebreak


\section{The \emph{qi} API}
\emph{In Development}


\end{document}