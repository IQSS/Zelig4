\documentclass[a4paper,10pt]{article}

\begin{document}

\title{The \emph{describe} Function}
\author{Matt Owen}
\maketitle

\section{Introduction}
The {\tt describe} function serves two purposes:

\begin{enumerate}
	\item{to give correct citation information for the Zelig model}
	\item{to declare the type of data-sets that can be processed by the Zelig model}
\end{enumerate}

The developer can accomplish these two things simply by writing the {\tt describe} function for their model.

\section{Form of a {\tt describe} Function}
The {\tt describe} function should - in almost all cases - simply return a list or character-vector specifying the author, year or publication, and description of the developed model.  That is, 

\section{Example of a {\tt describe} Function}
\begin{verbatim}
describe.logit <- function(zelig.obj)
  list(
       author   = c("Kosuke Imai", "Gary King"),
       year     = 2008,
       describe = "Logistic Regression for Dichotomous Dependent Variables"
       )
\end{verbatim}

\section{Resulting Citation from the Above Example}
\begin{verbatim}
How to cite this model in Zelig:
  Kosuke Imai, Gary King, and Olivia Lau. 2008.
  "logit: Logistic Regression for Dichotomous Dependent Variables"
  in Kosuke Imai, Gary King, and Olivia Lau,
  "Zelig: Everyone's Statistical Software,"
  http://gking.harvard.edu/zelig
\end{verbatim}

\section{Explanation of the Above Example}
The above example is an actual copy of the {\tt describe} function for the ``logit'' model in Zelig's core package.  It specifies the author, the year of publication, and the description text in a clear and concise manner.  All Zelig models can have citation information generated for the exact same fashion that the logit model does.
\end{document}